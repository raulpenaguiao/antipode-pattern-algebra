\documentclass[12pt, reqno]{amsart}

\usepackage{graphicx}
\usepackage{amssymb}
\usepackage{amsthm}
\usepackage{listings}
\usepackage{lineno}
\usepackage[margin=3cm]{geometry}
\usepackage[all,cmtip, color,matrix,arrow]{xy}

%\usepackage{natbib}
%\usepackage{graphicx}
\usepackage{amsmath}%To use \text 
%\usepackage{amssymb}
%\usepackage{amsthm}
\usepackage[utf8]{inputenc}
%\usepackage[english]{babel}
%\usepackage{biblatex}
\usepackage{hyperref}
\usepackage[capitalize]{cleveref}  
\crefname{thm}{Theorem}{Theorems}
\usepackage{bbold}
\usepackage[export]{adjustbox}
%\usepackage{tikz-cd}
%\usepackage{xr}
%\usetikzlibrary{babel}
\usepackage{todonotes}
\usepackage{bm}
\usepackage{wrapfig}
\usepackage{bbold}
\usepackage{float}
\usepackage{mathtools}
\usepackage{aliascnt}
\newaliascnt{eqfloat}{equation}
\newfloat{eqfloat}{h}{eqflts}
\floatname{eqfloat}{Equation}
\usepackage{dirtytalk}

\newcommand*{\ORGeqfloat}{}
\let\ORGeqfloat\eqfloat
\def\eqfloat{%
  \let\ORIGINALcaption\caption
  \def\caption{%
    \addtocounter{equation}{-1}%
    \ORIGINALcaption
  }%
  \ORGeqfloat
}
\newcommand{\raul}[1]{\todo[color=green!30,inline]{CH, #1}}


\theoremstyle{definition}
\newtheorem{thm}{Theorem}[section]
\newtheorem{prop}[thm]{Proposition}
\newtheorem{lm}[thm]{Lemma}
\newtheorem{cor}[thm]{Corollary}
\newtheorem{obs}[thm]{Observation}
\newtheorem{defin}[thm]{Definition}
\newtheorem{smpl}[thm]{Example}
\newtheorem{quest}[thm]{Question}
\newtheorem{prob}[thm]{Problem}
\newtheorem{conj}[thm]{Conjecture}
\newtheorem{rem}[thm]{Remark}
\crefname{lm}{Lemma}{Lemmas}
\crefname{thm}{Theorem}{Theorems}
\crefname{prop}{Proposition}{Propositions}
\crefname{defin}{Definition}{Definitions}
\crefname{rem}{Remark}{Remarks}

\newcommand{\oPi}{\mathbf{C}}
\newcommand{\opi}{\vec{\boldsymbol{\pi}}}
\newcommand{\otau}{\vec{\boldsymbol{\tau}}}
\newcommand{\olambda}{\vec{\boldsymbol{\gamma}}}
\newcommand{\msum}{\sideset{^M}{}\sum}
\newcommand{\nestohedra}{nestohThisedra}
\newcommand{\gpHa}{\mathbf{GP}}
\newcommand{\gHa}{\mathbf{G}}
\newcommand{\citestan}{\cite[Proposition 3.2]{gebhard99}}
\newcommand{\parpi}{\boldsymbol{\pi}}
\newcommand{\partau}{\boldsymbol{\tau}}
\newcommand{\makepar}{\boldsymbol{\lambda}}
\newcommand{\uhsm}{\boldsymbol{\Psi}}
\newcommand{\uHsm}{\boldsymbol{\Psi}}
\newcommand{\sqbinom}{\genfrac{[}{]}{0pt}{}}


\newcommand{\bfpsigrf}{$\mathbf{\Psi}_{\mathbf{Q}}$}

\newcommand{\pper}{marked permutation }
\newcommand{\ppers}{marked permutations }
\newcommand{\pperp}{marked permutation.}
\newcommand{\ppersp}{marked permutations.}

\newcommand{\stirling}[2]{\genfrac{[}{]}{0pt}{}{#1}{#2}}%binomial coefficients
\newcommand{\III}{\vec{\mathbf{I}}}
\newcommand{\JJJ}{\vec{\mathbf{J}}}


\DeclareMathOperator{\w}{\mathcal{W}}
\DeclareMathOperator{\pos}{\mathrm{pos}}
\DeclareMathOperator{\s}{\mathrm{mset}}
\DeclareMathOperator{\ms}{\mathrm{mset}}
\DeclareMathOperator{\Alg}{\mathrm{Alg}}
\DeclareMathOperator{\im}{im}
\DeclareMathOperator{\id}{id}
\DeclareMathOperator{\comu}{comu}
\DeclareMathOperator{\rest}{\mathbf{res}}
\DeclareMathOperator{\Orth}{Orth}
\DeclareMathOperator{\cano}{cano}
\DeclareMathOperator{\Ppatp}{\mathbf{Ppat}^*}
\DeclareMathOperator{\dpt}{\mathbf{depth}}
\DeclareMathOperator{\pat}{\mathbf{pat}}
\DeclareMathOperator{\Pat}{\mathrm{Pat}}
\DeclareMathOperator{\Func}{\mathrm{Func}}
\DeclareMathOperator{\spn}{\mathrm{span}}
\DeclareMathOperator{\inc}{\mathrm{inc}}
\DeclareMathOperator{\Ch}{\mathrm{Ch}}
\DeclareMathOperator{\tvs}{\textvisiblespace}
\DeclareMathOperator{\sFunc}{\mathrm{SurFunc}}
%\usepackage{lipsum}

\newcommand{\bfa}{\mathbf{a}}

%combinatorial concepts
\newcommand{\Sr}{\mathrm{S}} %symmetric group
\newcommand{\opp}[1]{\overline{#1}} %for opposite
\newcommand{\len}{l} % for length (degree) of a composition or partition
%\newcommand{\maxflat}{\hat{1}}
\newcommand{\maxflat}{\widehat{I}}
\newcommand{\minflat}{\{I\}}
\newcommand{\ifac}{
\begin{picture}(3,5)(0,0)
\put(0,0){\textup{!}}\put(1.5,4.8){\circle{3}}
\end{picture}
} %cyclic factorial
\newcommand{\acyc}[1]{#1 !} %acyclic orientations

%linear operators


%categories
\newcommand{\Fset}{\mathsf{Set^{\times}}}
\newcommand{\Veck}{\mathsf{Vec}_\Kb} 
\newcommand{\Vect}{\mathsf{Vect}} 
\newcommand{\Set}{\mathsf{Set}}
\newcommand{\Ss}{\mathsf{Sp}} % set species
\newcommand{\Ssk}{\mathsf{Sp}_\Kb} %vector species
\newcommand{\Spr}{\mathsf{Spr}} % set species with restrictions

\newcommand{\Mo}[1]{\mathsf{Mon}(#1)} %monoids
\newcommand{\Co}[1]{\mathsf{Comon}(#1)} %comonoids
\newcommand{\Bo}[1]{\mathsf{Bimon}(#1)} %bimonoids
\newcommand{\Kb}{\mathbb{K}} 



%generic set species
\newcommand{\rP}{\mathrm{P}}
\newcommand{\rQ}{\mathrm{Q}}
\newcommand{\rH}{\mathrm{H}}
\newcommand{\rR}{\mathrm{R}}


%generic species with restrictions
\newcommand{\prP}{\mathtt{P}}
\newcommand{\prQ}{\mathtt{Q}}
\newcommand{\prH}{\mathtt{H}}
\newcommand{\prR}{\mathtt{R}}


%set species
\newcommand{\rE}{\mathrm{E}}
\newcommand{\rL}{\mathrm{L}}
%\newcommand{\rX}{\mathrm{X}}
\let\rPi=\Pi %flats
\let\rSig=\Sigma
\newcommand{\rSigh}{\widehat{\rSig}} %decompositions = weak compositions
\newcommand{\rG}{\mathrm{G}} %graphs
\newcommand{\rGP}{\mathrm{GP}} %generalized permutahedra
\newcommand{\SP}{\mathfrak{p}} %standard permutahedron

%generic species
\newcommand{\thh}{\mathbf{h}} 
\newcommand{\tg}{\mathbf{g}}
\newcommand{\tk}{\mathbf{k}}
\newcommand{\tp}{\mathbf{p}} 
\newcommand{\tq}{\mathbf{q}}
\newcommand{\tr}{\mathbf{r}}
\newcommand{\ta}{\mathbf{a}}
\newcommand{\tb}{\mathbf{b}} 
\newcommand{\tc}{\mathbf{c}}
\newcommand{\td}{\mathbf{d}}
\newcommand{\trr}{\mathbf{r}} 
\newcommand{\tm}{\mathbf{m}} %module
\newcommand{\brac}{\nu} %Lie bracket

%examples of species
\newcommand{\tone}{\mathbf{1}}
\newcommand{\wU}{\mathbf{U}}
\newcommand{\wX}{\mathbf{X}}
\newcommand{\wE}{\mathbf{E}} %exponential species
\newcommand{\wL}{\mathbf{L}} %linear orders
\newcommand{\wI}{\mathbf{I}}%used in macro \tLL
\newcommand{\tLL}{\wI\hspace*{-2pt}\wL} %pairs of chambers
\newcommand{\tLie}{\mathbf{Lie}} %Lie species
\newcommand{\tPi}{\mathbf{\Pi}} %flats
\newcommand{\tSig}{\mathbf{\Sigma}} %faces
\newcommand{\tSigh}{\mathbf{\widehat{\Sigma}}} % decompositions
\newcommand{\te}{\mathbf{e}} %elements
\newcommand{\tG}{\mathbf{G}} %graphs
\newcommand{\tcG}{\mathbf{cG}} %connected graphs
\newcommand{\tGP}{\mathbf{GP}} 

%some isomorphisms
\let\isoflat=\psi
\let\isolinear=\psi
\let\isograph=\varphi


%Fock functors
\newcommand{\contra}[1]{#1^{\vee}} 
\newcommand{\Kc}{\mathcal{K}}
\newcommand{\Kcb}{\overline{\Kc}}
\newcommand{\cKc}{\contra{\Kc}}
\newcommand{\cKcb}{\contra{\Kcb}}

%functors
\newcommand{\Tc}{\mathcal{T}}
\newcommand{\Tcq}{\Tc_q}
\newcommand{\Sc}{\mathcal{S}}
\newcommand{\Pc}{\mathcal{P}} %primitive element functor
\newcommand{\Qc}{\mathcal{Q}} %indecomposables
\newcommand{\Uc}{\mathcal{U}} %universal enveloping algebra



%% natbib.sty is loaded by default. However, natbib options can be
%% provided with \biboptions{...} command. Following options are
%% valid:

%%   round  -  round parentheses are used (default)
%%   square -  square brackets are used   [option]
%%   curly  -  curly braces are used      {option}
%%   angle  -  angle brackets are used    <option>
%%   semicolon  -  multiple citations separated by semi-colon
%%   colon  - same as semicolon, an earlier confusion
%%   comma  -  separated by comma
%%   numbers-  selects numerical citations
%%   super  -  numerical citations as superscripts
%%   sort   -  sorts multiple citations according to order in ref. list
%%   sort&compress   -  like sort, but also compresses numerical citations
%%   compress - compresses without sorting
%%
%% \biboptions{comma,round}

% \biboptions{}


%\usepackage[backend=bibtex]{biblatex}
%\addbibresource{biblio.bib}

\usepackage{amsaddr}

\begin{document}

%% Title, authors and addresses
\title{Antipode formulas for pattern Hopf algebras} % Subtitle


%\author{Raul Penaguiao\footnote{\href{mailto:raulpenaguiao@sfsu.edu}{raulpenaguiao@sfsu.edu}}\footnote{Institute of Mathematics, University of Zurich, Winterthurerstrasse 190, Zurich, CH - 8057.}\footnote{{\bf Keywords:} marked permutations, presheaves, species, Hopf algebras, free algebras}\footnote{2010 AMS Mathematics Subject Classification 2010: 05E05, 16T05, 18D10}}

\author{Raul Penaguiao, Yannic Vargas}
\email{raulpenaguiao@sfsu.edu}
\email{vargas@wias-berlin.de}
\address{San Francisco State University}
\address{Weierstrass Institute for Applied Analysis and Stochastics}
\keywords{permutations, presheaves, species, Hopf algebras, free algebras, antipode}
\subjclass[2010]{05E05, 16T05, 18D10}
\date{\today} % Date

\begin{abstract}
The permutation pattern Hopf algebra is a commutative filtered and connected Hopf algebra.
Its product structure stems from counting patterns of a permutation, and the Hopf algebra was shown to be free and to fit into a general framework of pattern Hopf algebras.

In this paper we discuss a polynomial invariant on permutations that arises from this Hopf algebra structure.
Because this Hopf algebra is not graded, a ready made polynomial invariant does not arise from the work of Aguiar, Bergeron and Sotille.
We are still however able to obtain this polynomail invariant in full generality for pattern Hopf algebras.

Leveraging the motivating work on pattern Hopf algebras, the resulting polynomial invariant is a Hopf algebra morphism.
On the particular case of permutations, we explore the consequences of the recently discovered cancellation-free and grouping-free antipode formula to obtain reciprocity results.
\end{abstract}


\maketitle

\tableofcontents

\section{Introduction}


\begin{defin}[Polynomial invariant on a species with restrictions $\mathtt{h}$]
Let $\mathrm{y}$ be an object in the species with restrictions $\mathtt{h}$.
Then we define the following invariant on $\mathtt{h}$: let $\mathrm{x}$ be an object, then
$$\chi^{\mathrm{x}}(\pat_{\mathrm{y}})(n) \coloneqq \pat_{\mathrm{y}}(\mathrm{x}^{\star n})\, . $$
\end{defin}

\section{Properties of the invariant}

\begin{lm}[Coefficients]
Let $x$ be an object in $\mathrm{x}$ with factorization into irreducibles $\mathrm{x} = x_1 \star \dots \star x_m$.
For a composition $\alpha \models m$

\end{lm}

\begin{thm}
Let $\mathtt{h}$ be a associative species with restrictions, and $\mathrm{x}$ an object in it.
Then the following defines a Hopf algebra morphism $\chi^{\mathrm{x}}: \mathcal{A}(\mathtt{h}) \to \mathbb{K}[n]$:
$$\chi^{\mathrm{x}}(\pat_{\mathrm{y}})(n) \coloneqq \pat_{\mathrm{y}}(\mathrm{x}^{\star n})\, . $$
\end{thm}



\begin{lm}
In the polynomial Hopf algebra $\mathbb{K}[x]$, one has
$$\Delta \binom{x}{l} = \sum_{a+b=l} \binom{x}{a}\otimes \binom{x}{b}\, .$$
\end{lm}

\begin{proof}
We recall that we have $\Delta x = 1 \otimes x + x \otimes 1$.
We act by strong induction, with the base case $l = 0$ being dealt by the reader.

Then
\begin{equation*}
\begin{split}
\Delta \binom{x}{l+1}&= \Delta \binom{x}{l} \frac{x-l-1}{l+1}\\
                     &= \left(\sum_{a+b=l} \binom{x}{a}\otimes \binom{x}{b}\right) \Delta \frac{x-l-1}{l+1}\\
                     &= \sum_{a+b=l} \binom{x}{a}\otimes \binom{x}{b}\left(\frac{1}{l+1}(1\otimes x + x\otimes 1) - 1\otimes 1\right)\\
                     &= \sum_{a+b=l} 
\end{split}
\end{equation*}

\end{proof}

\begin{proof}
We need to show that $\chi^{\mathrm{x}}$ is a polynomial, that it preserves products and coproducts.
We start with the polynomiality, by describing 

For that, we appeal to the following 



Furthermore, we need to show that $\chi$ preserves the product structure.
Indeed, for each $n$, we have that
\end{proof}


\section{Polynomial invariant in some species with restrictions}

\subsection{In graphs}

\subsection{In permutations}

\subsection{In marked permutations}


\section{Reciprocity results}

In this section we answer the following question:
What is the combinatorial interpretation of $\chi^{\tau}_{\pi}(-1)$?

\bibliographystyle{alpha}
\bibliography{Bibliography}



\end{document}
