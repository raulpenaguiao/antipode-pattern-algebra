\documentclass[12pt, reqno]{amsart}

\usepackage{graphicx}
\usepackage{amssymb}
\usepackage{amsthm}
\usepackage{listings}
\usepackage{lineno}
\usepackage[margin=3cm]{geometry}
\usepackage[all,cmtip, color,matrix,arrow]{xy}

%\usepackage{natbib}
%\usepackage{graphicx}
\usepackage{amsmath}%To use \text 
%\usepackage{amssymb}
%\usepackage{amsthm}
\usepackage[utf8]{inputenc}
%\usepackage[english]{babel}
%\usepackage{biblatex}
\usepackage{hyperref}
\usepackage[capitalize]{cleveref}  
\crefname{thm}{Theorem}{Theorems}
\usepackage{bbold}
\usepackage[export]{adjustbox}
%\usepackage{tikz-cd}
%\usepackage{xr}
%\usetikzlibrary{babel}
\usepackage{todonotes}
\usepackage{bm}
\usepackage{wrapfig}
\usepackage{bbold}
\usepackage{float}
\usepackage{mathtools}
\usepackage{aliascnt}
\newaliascnt{eqfloat}{equation}
\newfloat{eqfloat}{h}{eqflts}
\floatname{eqfloat}{Equation}
\usepackage{dirtytalk}

\newcommand*{\ORGeqfloat}{}
\let\ORGeqfloat\eqfloat
\def\eqfloat{%
  \let\ORIGINALcaption\caption
  \def\caption{%
    \addtocounter{equation}{-1}%
    \ORIGINALcaption
  }%
  \ORGeqfloat
}
\newcommand{\raul}[1]{\todo[color=green!30,inline]{Raul: #1}}

\newcommand{\yannic}[1]{\todo[color=violet!30,inline]{YV, #1}}


\theoremstyle{definition}
\newtheorem{thm}{Theorem}[section]
\newtheorem{prop}[thm]{Proposition}
\newtheorem{lm}[thm]{Lemma}
\newtheorem{cor}[thm]{Corollary}
\newtheorem{obs}[thm]{Observation}
\newtheorem{defin}[thm]{Definition}
\newtheorem{smpl}[thm]{Example}
\newtheorem{quest}[thm]{Question}
\newtheorem{prob}[thm]{Problem}
\newtheorem{conj}[thm]{Conjecture}
\newtheorem{rem}[thm]{Remark}
\crefname{lm}{Lemma}{Lemmas}
\crefname{thm}{Theorem}{Theorems}
\crefname{prop}{Proposition}{Propositions}
\crefname{defin}{Definition}{Definitions}
\crefname{rem}{Remark}{Remarks}

\newcommand{\oPi}{\mathbf{C}}
\newcommand{\opi}{\vec{\boldsymbol{\pi}}}
\newcommand{\otau}{\vec{\boldsymbol{\tau}}}
\newcommand{\olambda}{\vec{\boldsymbol{\gamma}}}
\newcommand{\msum}{\sideset{^M}{}\sum}
\newcommand{\nestohedra}{nestohThisedra}
\newcommand{\gpHa}{\mathbf{GP}}
\newcommand{\gHa}{\mathbf{G}}
\newcommand{\citestan}{\cite[Proposition 3.2]{gebhard99}}
\newcommand{\parpi}{\boldsymbol{\pi}}
\newcommand{\partau}{\boldsymbol{\tau}}
\newcommand{\makepar}{\boldsymbol{\lambda}}
\newcommand{\uhsm}{\boldsymbol{\Psi}}
\newcommand{\uHsm}{\boldsymbol{\Psi}}
\newcommand{\sqbinom}{\genfrac{[}{]}{0pt}{}}


\newcommand{\bfpsigrf}{$\mathbf{\Psi}_{\mathbf{Q}}$}

\newcommand{\pper}{marked permutation }
\newcommand{\ppers}{marked permutations }
\newcommand{\pperp}{marked permutation.}
\newcommand{\ppersp}{marked permutations.}

\newcommand{\stirling}[2]{\genfrac{[}{]}{0pt}{}{#1}{#2}}%binomial coefficients
\newcommand{\III}{\vec{\mathbf{I}}}
\newcommand{\JJJ}{\vec{\mathbf{J}}}


\DeclareMathOperator{\w}{\mathcal{W}}
\DeclareMathOperator{\pos}{\mathrm{pos}}
\DeclareMathOperator{\s}{\mathrm{mset}}
\DeclareMathOperator{\ms}{\mathrm{mset}}
\DeclareMathOperator{\Alg}{\mathrm{Alg}}
\DeclareMathOperator{\im}{im}
\DeclareMathOperator{\id}{id}
\DeclareMathOperator{\comu}{comu}
\DeclareMathOperator{\rest}{\mathbf{res}}
\DeclareMathOperator{\Orth}{Orth}
\DeclareMathOperator{\cano}{cano}
\DeclareMathOperator{\Ppatp}{\mathbf{Ppat}^*}
\DeclareMathOperator{\dpt}{\mathbf{depth}}
\DeclareMathOperator{\pat}{\mathbf{pat}}
\DeclareMathOperator{\Pat}{\mathrm{Pat}}
\DeclareMathOperator{\Func}{\mathrm{Func}}
\DeclareMathOperator{\spn}{\mathrm{span}}
\DeclareMathOperator{\inc}{\mathrm{inc}}
\DeclareMathOperator{\Ch}{\mathrm{Ch}}
\DeclareMathOperator{\tvs}{\textvisiblespace}
\DeclareMathOperator{\sFunc}{\mathrm{SurFunc}}
%\usepackage{lipsum}

\newcommand{\bfa}{\mathbf{a}}

%combinatorial concepts
\newcommand{\Sr}{\mathrm{S}} %symmetric group
\newcommand{\opp}[1]{\overline{#1}} %for opposite
\newcommand{\len}{l} % for length (degree) of a composition or partition
%\newcommand{\maxflat}{\hat{1}}
\newcommand{\maxflat}{\widehat{I}}
\newcommand{\minflat}{\{I\}}
\newcommand{\ifac}{
\begin{picture}(3,5)(0,0)
\put(0,0){\textup{!}}\put(1.5,4.8){\circle{3}}
\end{picture}
} %cyclic factorial
\newcommand{\acyc}[1]{#1 !} %acyclic orientations

%linear operators


%categories
\newcommand{\Fset}{\mathsf{Set^{\times}}}
\newcommand{\Veck}{\mathsf{Vec}_\Kb} 
\newcommand{\Vect}{\mathsf{Vect}} 
\newcommand{\Set}{\mathsf{Set}}
\newcommand{\Ss}{\mathsf{Sp}} % set species
\newcommand{\Ssk}{\mathsf{Sp}_\Kb} %vector species
\newcommand{\Spr}{\mathsf{Spr}} % set species with restrictions

\newcommand{\Mo}[1]{\mathsf{Mon}(#1)} %monoids
\newcommand{\Co}[1]{\mathsf{Comon}(#1)} %comonoids
\newcommand{\Bo}[1]{\mathsf{Bimon}(#1)} %bimonoids
\newcommand{\Kb}{\mathbb{K}} 



%generic set species
\newcommand{\rP}{\mathrm{P}}
\newcommand{\rQ}{\mathrm{Q}}
\newcommand{\rH}{\mathrm{H}}
\newcommand{\rR}{\mathrm{R}}


%generic species with restrictions
\newcommand{\prP}{\mathtt{P}}
\newcommand{\prQ}{\mathtt{Q}}
\newcommand{\prH}{\mathtt{H}}
\newcommand{\prR}{\mathtt{R}}


%set species
\newcommand{\rE}{\mathrm{E}}
\newcommand{\rL}{\mathrm{L}}
%\newcommand{\rX}{\mathrm{X}}
\let\rPi=\Pi %flats
\let\rSig=\Sigma
\newcommand{\rSigh}{\widehat{\rSig}} %decompositions = weak compositions
\newcommand{\rG}{\mathrm{G}} %graphs
\newcommand{\rGP}{\mathrm{GP}} %generalized permutahedra
\newcommand{\SP}{\mathfrak{p}} %standard permutahedron

%generic species
\newcommand{\thh}{\mathbf{h}} 
\newcommand{\tg}{\mathbf{g}}
\newcommand{\tk}{\mathbf{k}}
\newcommand{\tp}{\mathbf{p}} 
\newcommand{\tq}{\mathbf{q}}
\newcommand{\tr}{\mathbf{r}}
\newcommand{\ta}{\mathbf{a}}
\newcommand{\tb}{\mathbf{b}} 
\newcommand{\tc}{\mathbf{c}}
\newcommand{\td}{\mathbf{d}}
\newcommand{\trr}{\mathbf{r}} 
\newcommand{\tm}{\mathbf{m}} %module
\newcommand{\brac}{\nu} %Lie bracket

%examples of species
\newcommand{\tone}{\mathbf{1}}
\newcommand{\wU}{\mathbf{U}}
\newcommand{\wX}{\mathbf{X}}
\newcommand{\wE}{\mathbf{E}} %exponential species
\newcommand{\wL}{\mathbf{L}} %linear orders
\newcommand{\wI}{\mathbf{I}}%used in macro \tLL
\newcommand{\tLL}{\wI\hspace*{-2pt}\wL} %pairs of chambers
\newcommand{\tLie}{\mathbf{Lie}} %Lie species
\newcommand{\tPi}{\mathbf{\Pi}} %flats
\newcommand{\tSig}{\mathbf{\Sigma}} %faces
\newcommand{\tSigh}{\mathbf{\widehat{\Sigma}}} % decompositions
\newcommand{\te}{\mathbf{e}} %elements
\newcommand{\tG}{\mathbf{G}} %graphs
\newcommand{\tcG}{\mathbf{cG}} %connected graphs
\newcommand{\tGP}{\mathbf{GP}} 

%some isomorphisms
\let\isoflat=\psi
\let\isolinear=\psi
\let\isograph=\varphi


%Fock functors
\newcommand{\contra}[1]{#1^{\vee}} 
\newcommand{\Kc}{\mathcal{K}}
\newcommand{\Kcb}{\overline{\Kc}}
\newcommand{\cKc}{\contra{\Kc}}
\newcommand{\cKcb}{\contra{\Kcb}}

%functors
\newcommand{\Tc}{\mathcal{T}}
\newcommand{\Tcq}{\Tc_q}
\newcommand{\Sc}{\mathcal{S}}
\newcommand{\Pc}{\mathcal{P}} %primitive element functor
\newcommand{\Qc}{\mathcal{Q}} %indecomposables
\newcommand{\Uc}{\mathcal{U}} %universal enveloping algebra

%objects
\newcommand{\xx}{\mathrm{x}}
\newcommand{\yy}{\mathrm{y}}
\newcommand{\zz}{\mathrm{z}}



%% natbib.sty is loaded by default. However, natbib options can be
%% provided with \biboptions{...} command. Following options are
%% valid:

%%   round  -  round parentheses are used (default)
%%   square -  square brackets are used   [option]
%%   curly  -  curly braces are used      {option}
%%   angle  -  angle brackets are used    <option>
%%   semicolon  -  multiple citations separated by semi-colon
%%   colon  - same as semicolon, an earlier confusion
%%   comma  -  separated by comma
%%   numbers-  selects numerical citations
%%   super  -  numerical citations as superscripts
%%   sort   -  sorts multiple citations according to order in ref. list
%%   sort&compress   -  like sort, but also compresses numerical citations
%%   compress - compresses without sorting
%%
%% \biboptions{comma,round}

% \biboptions{}


%\usepackage[backend=bibtex]{biblatex}
%\addbibresource{biblio.bib}

\usepackage{amsaddr}

\begin{document}

%% Title, authors and addresses
\title{Antipode formulas for pattern Hopf algebras} % Subtitle


%\author{Raul Penaguiao\footnote{\href{mailto:raulpenaguiao@sfsu.edu}{raulpenaguiao@sfsu.edu}}\footnote{Institute of Mathematics, University of Zurich, Winterthurerstrasse 190, Zurich, CH - 8057.}\footnote{{\bf Keywords:} marked permutations, presheaves, species, Hopf algebras, free algebras}\footnote{2010 AMS Mathematics Subject Classification 2010: 05E05, 16T05, 18D10}}

\author{Raul Penaguiao, Yannic Vargas}
\email{raul.penaguiao@mis.mpg.de}
\email{vargas@wias-berlin.de}
\address{Max Planck Institute for Mathematics in the Sciences, Leipzig}
\address{Weierstrass Institute for Applied Analysis and Stochastics}
\keywords{permutations, presheaves, species, Hopf algebras, free algebras, antipode, polynomials}
\subjclass[2010]{05E05, 16T05, 18D10}
\date{\today} % Date

\begin{abstract}
The permutation pattern Hopf algebra is a commutative filtered and connected Hopf algebra.
Its product structure stems from counting patterns of a permutation, and the Hopf algebra was shown to be free and to fit into a general framework of pattern Hopf algebras.

In this paper we discuss a polynomial invariant on permutations that arises from this Hopf algebra structure.
Because this Hopf algebra is not graded, a ready made polynomial invariant does not arise from the work of Aguiar, Bergeron and Sotille \yannic{The Hopf algebra of quasisymmetric functions is also the initial object in the category of combinatorial Hopf algebras, without assumption of the graduation (see Remark 4.2 in \cite{ABS2006}). I will ask to Marcelo about this result.}.
We are still however able to obtain this polynomial invariant in full generality for pattern Hopf algebras.

Leveraging the motivating work on pattern Hopf algebras, the resulting polynomial invariant is a Hopf algebra morphism.
On the particular case of permutations, we explore the consequences of the recently discovered cancellation-free and grouping-free antipode formula to obtain reciprocity results.
\end{abstract}


\maketitle

\tableofcontents

\section{Introduction}

Polynomial invariants are pervasive in the study of combinatorial objects.
Central figure is the chromatic polynomial on graphs, first defined in \cite{}.

Our central invariant in this paper is the following polynomial invariant:

\begin{defin}[Polynomial invariant on a species with restrictions $\mathtt{R}$]
Let $\mathrm{y}$ be an object in the connected associative species with restrictions $\mathtt{R}$.
Then we define the following invariant on $\mathtt{R}$: fix $\mathrm{x}$ an object, then
$$\chi^{\mathrm{x}}(\pat_{\mathrm{y}})(n) \coloneqq \pat_{\mathrm{y}}(\mathrm{x}^{\star n})\, . $$
\end{defin}

One of the main results in this paper is the following:
\begin{thm}[Polynomial invariants in pattern algebras]\label{thm:polynomiality}
If $\mathtt{R}$ is a connected associative species with restrictions and $\mathrm{x}$ on of its objects, then $\chi^{\mathrm{x}}$ is a Hopf algebra morphism $\mathcal A(\mathtt{R}) \to \mathbb{K}[x]$.
\end{thm}


This article is organized as follows: in \cref{sec:}, we introduce relevant notation.
In \cref{sec:proof}, we present a proof of the polynomiality of $\chi^{\mathrm{x}}_{\mathrm{y}}$.
We also show that this is a Hopf algebra morphism.
In \cref{sec:recip} we explore some particular cases in permutations, where we present some significant consequences of the antipode formula presented in \cite{penaguiao2022antipode}.


\section{Properties of the invariant\label{sec:proof}}


We recall from \cite{Penaguiao2020} that for each object $\yy$ in a connected associative species with restrictions, any factorization into $\ast$-indecomposible elements has the same multiset of factors and, therefore, the same size, which we call $\ell(\yy)$.

\begin{lm}\label{lm:polynomial}
The invariant $\chi^{\xx}_{\yy}(n)$ is a polynomial in $n$.
The degree of this polynomial is at most $\ell(\yy)$, the number of $\ast$-indecomposible factors of $\yy$ (with repetition).
\end{lm}


\begin{proof}
We act by induction on $|\yy|$.
If $|\yy| = 0$, recall that $\mathtt{R}$ is connected, so $\chi^{\xx}_{\yy}(n) = 1$ for all $n$, which is a polynomial of degree zero.

For the induction hypothesis, observe that 
$$\Delta \pat_{\yy} = 1\otimes \pat_{\yy} + \pat_{\yy} \otimes 1 + \sum_{\substack{a \ast b = y\\\ell(a), \ell(b) < \ell(\yy)}} \pat_a \otimes \pat_b \, ,$$
thus we have for $n > 1$ 

\begin{equation*}
    \begin{split}
        \pat_{\yy}(\xx^{\ast n}) &= \pat_{\yy}(\xx^{\ast n-1}) + \pat_{\yy}(\xx) + \sum_{\substack{a \ast b = y\\|a|, |b| < |\yy|}} \pat_a (\xx) \otimes \pat_b (\xx^{\ast n-1}) \\
         \pat_{\yy}(\xx^{\ast n}) &- \pat_{\yy}(\xx^{\ast n-1}) = \pat_{\yy}(\xx) + \sum_{\substack{a \ast b = y\\|a|, |b| < |\yy|}} \pat_a (\xx) \chi^{\xx}_b ( n-1)
    \end{split}
\end{equation*}

The right hand side is, by induction hypothesis, a polynomial of degree at most $\ell(\yy) -1$.
The left hand side is $\chi^{\xx}_{\yy}(n) - \chi^{\xx}_{\yy}(n-1)$, which shows that $\chi^{\xx}_{\yy}(n)$ has degree at most $\ell(\yy)$.
\end{proof}


\begin{lm}\label{lm:HA_morphism}
The map $\pat_{\yy} \mapsto \chi^{\xx}_{\yy}$ is a Hopf algebra morphism.
\end{lm}

\begin{proof}
   Observe that if $\pat_{\yy_1} \pat_{\yy_2} = \sum_{\zz} \binom{\zz}{\yy_1, \yy_2}_{\mathtt{R}} \pat_{\zz}$, then passing in the argument $x^{\ast n}$ of both sides yields $\chi^{\xx}_{\yy_1}(n) \chi^{\xx}_{\yy_2}(n) = \sum_{\zz} \binom{\zz}{\yy_1, \yy_2}_{\mathtt{R}} \chi^{\xx}_{\zz}(n)  $.
    This is precisely what we want from the product structure.

    For the coproduct, we want to show that $\Delta \chi^{\xx}_{\yy} = \chi^{\xx} \otimes \chi^{\xx}(\Delta \pat_{\yy})$.
    It is enough to show that $\Delta \chi^{\xx}_{\yy} (m \otimes n) = \chi^{\xx} \otimes \chi^{\xx}(\Delta \pat_{\yy})(m\otimes n) $ for any non-negative integers $m, n$.

    Write $\chi^{\xx}_{\yy} = \sum_{k\geq 0 } a_k x^k$.
    On the left hand side, we get $\Delta \chi^{\xx}_{\yy} (m \otimes n) = \sum_{k \geq 0} a_k \sum_{j=0}^k \binom{k}{j} x^j\otimes x^{k-j} (m \otimes n) = \sum_{k \geq 0} a_k (m+n)^k = \chi^{\xx}_{\yy}(m+n) $.

    On the right hand side we have 
    
    
    \begin{equation*}
        \begin{split}
        \chi^{\xx} \otimes \chi^{\xx}\Delta \pat_{\yy}(m\otimes n) =& \sum_{a \ast b = \yy} \chi^{\xx}_a(m) \otimes \chi^{\xx}_a(n)\\
        =& \sum_{a \ast b = \yy} \pat_a(x^{\ast m})\pat_b(x^{\ast n})\\
        =& \sum_{a \ast b = \yy} \pat_a\otimes \pat_b ( x^{\ast m} \otimes x^{\ast n})\\
        =& \Delta \pat_{\yy}(x^{\ast n} \ast x^{\ast b}) = \Delta \pat_{\yy}(x^{\ast m+n}) = \chi^{\xx}_{\yy}(n+m)\, .
        \end{split}
    \end{equation*}
    This is exactly what we obtained earlier, so this map preserves coproducts.
\end{proof}


\begin{proof}[Proof of \cref{thm:polynomiality}]
This is a consequence of \cref{lm:polynomial} and \cref{lm:HA_morphism}.
\end{proof}


\section{Reciprocity results\label{sec:recip}}

In this section we answer the following question:
What is the combinatorial interpretation of $\chi^{\tau}_{\pi}(-1)$?


\begin{thm}
Let $\rho$ be a packed word, and let $\pi = \pi_1 \oplus\dots \oplus \pi_n$ be the decomposition of a packed word $\pi$ into $\oplus$-indecomposibles, then
    $$\chi^{\rho}_{\pi}(-x) = (-1)^n \sum_{\sigma}
    \begin{bmatrix}
    \sigma \\ \pi_1, \dots, \pi_n
    \end{bmatrix}
     \chi^{\rho}_{\omega}(x).$$
\end{thm}

\bibliographystyle{alpha}
\bibliography{Bibliography}



\end{document}