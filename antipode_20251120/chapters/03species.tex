

\section{Combinatorial species\label{sec:species}}


In this section we give the preliminaries on monoids in species.
This will follow closely \cite{AM2010} and \cite{Schmitt1993}.
Specifically, we introduce species with vector spaces and sets.
We also introduce species with restrictions, and we will clarify the meaning of a monoid, comonoid, bimonoid and Hopf monoid in each of these monoidal categories.
We finally present some examples of species with restrictions that will be important in the remaining paper.


\begin{figure}
\centering
\includegraphics[scale=.6]{../images/comuting_diagram_algebras.png}
\caption{Diagram reflecting the context of the pattern Hopf algebra. 
Given a restriction species $\tr$, its linearisation is denoted $\mathbf{r} = \mathbb{K}\tr$. 
The product $\ast$ in the top-right corner corresponds to the pointwise product of functions in the dual space $\overline{\mathcal{K}}(\mathbf{r})$. The coproduct $\Delta_{\square} $ of the pattern Hopf algebra corresponds to the dual of the concatenation product $\square$, after applying the Fock functor $\overline{\mathcal{K}}$.}
\end{figure}

\

\subsection{Species}
In this section we recall the basic definitions of the general theory of \emph{combinatorial species}. Following \cite{AM2010}, we will focus first in \emph{vectorial species} and \emph{set species}.

\

Let $\mathbb{K}$ be a field of arbitrary characteristic. Let $\Fset$ be the category of finite sets and bijections between finite sets, and $\Vect_{\mathbb{K}}$ be the category of $\mathbb{K}$-vector spaces and linear maps between vector spaces. A {\bf vector species}, or simply a \textbf{species}, is a functor $\tp: \Fset \to \Vect_{\mathbb{K}}$. A morphism between species $\tp$ and $\tq$ is a natural transformation between the functors $\tp$ and $\tq$.
For clarity, we denote the vector species with a bold lowercase Latin letter, with few exceptions.

\

A species $\tp$ is said to be {\bf positive} if $\tp[\emptyset]=0$. The {\bf positive part} of a species $\tq$ is the positive species $\tq_+$ given by
\[\tq_+[I]=\begin{cases}
\tq[I],& \text{ if } I\neq \emptyset\\
\emptyset, & \text{ otherwise}
\end{cases}.\]

Given a vector space $V$, let ${\bf 1}_V$ be the vector species defined by
\[{\bf 1}_V[I]=\begin{cases}
V,& \text{ if } I= \emptyset\\
\emptyset, & \text{ otherwise}
\end{cases}.\]

\

We write $\Ss_{\mathbb{K}}$ for the category of vector species over the field $\mathbb{K}$.
We consider two monoidal structures on this category: the {\bf Cauchy} and {\bf Hadamard} products $\cdot$ and $\times$, respectively: for any finite set $I$,
\[(\tp \cdot \tq)[I]:=\bigoplus_{I = S \sqcup T}\tp[S]\otimes \tq[T];\]
\[(\tp \times \tq)[I]:=\tp[I] \otimes \tq[I].\]

We denote by $(\Ssk, \cdot)$ the resulting monoidal category obtained from the Cauchy operation.

\

We can also consider {\bf set species}, these are functors $\rP: \Fset \to \Set$, where $\Set$ is the category of arbitrary sets and arbitrary maps between sets. Given a set species $\rP$, the notions of {\bf positive part} $\rP_+$ of $\rP$, {\bf positive set species} are defined analogously as for vector species. If $C$ is a set, let ${\bf 1}_C$ be the set species defined by
\[{\bf 1}_C[I]=\begin{cases}
C,& \text{ if } I= \emptyset\\
\emptyset, & \text{ otherwise}
\end{cases},\]
for any finite set $I$.
For clarity, we denote a set species with a capital Latin letter, with few exceptions.

\

The Cauchy and Hadamard products of vector species have their analogues in this context. For instance, if $\rP$ and $\rQ$ are two set species, let
\[(\rP \cdot \rQ)[I]:=\bigsqcup_{I = S \sqcup T}\rP[S]\times \rQ[T];\]
\[(\rP \times \rQ)[I]:=\rP[I]\times  \rQ[I],\]
on any finite set $I$, where the $\times$ symbol in the right-hand sides refers to the Cartesian product.

\

It is possible to relate set species to vector species via the \emph{linearisation functor} $\mathbb{K}(-): \Set \to \Vect_{\mathbb{K}}$, which sends a set to the vector space generated by the given set. Composing a set species $\rP$ with the linearisation functor gives a vector species, denoted by $\mathbb{K}\rP$. A {\bf linearized species} is a vector species $\tp$ of the form $\tp=\mathbb{K}\rP$, for some set species $\rP$. We have natural isomorphisms
 \[\mathbb{K}(\rP \cdot \rQ)\simeq \mathbb{K}\rP \cdot \mathbb{K}\rQ \qquad , \qquad \mathbb{K}(\rP \times \rQ)\simeq \mathbb{K}\rP \times \mathbb{K}\rQ.\]

\

\subsection{Algebraic structures on vector species}


\subsubsection{Monoids}
A {\bf monoid} in $(\Ssk, \cdot)$ consists of a species $\tp$ equipped with morphisms of species
\begin{equation*}
    \mu: \tp \cdot \tp \to \tp \qquad \text{ and } \qquad \iota: \mathrm{1}_{\mathbb{K}} \to \tp.
\end{equation*}

That is, for each finite set $I$ and for each decomposition $I=S \sqcup T$, we have a linear map 
\begin{equation*}
    \mu_{S,T}: \tp[S] \otimes \tp[T]\to \tp[I] \text{ and } \iota_\emptyset: \mathbb{K} \to \tp[\emptyset].
\end{equation*}

If $x \in \tp[S]$, $y \in \tp[T]$,  let 
\[x \cdot y \in \tp[I]\]
denote the image of $x\otimes y$ under $\mu_{S,T}$. 

\

The collection of linear maps $\mu=(\mu_{S,T})$, called the {\bf product} of the monoid, must satisfy the following axioms.

\

\begin{itemize}
    \item[(i)] Naturality axiom: for finite sets $I,J$, a bijection $\sigma: I \to J$, a decomposition $I=S \sqcup T$ and elements $x \in \tp[S]$ and $y \in \tp[T]$, we have
\begin{equation*}
\tp[\sigma](x \cdot y)=\tp[\sigma(S)](x) \cdot \tp[\sigma(T)](y) \, .
\end{equation*}

\item[(ii)] Associativity axiom: for finite set $I$, a decomposition $I=R \sqcup S \sqcup T$ and for elements $x \in \tp[R]$, $y \in \tp[S]$ and $z \in \tp[T]$, we have
\begin{equation}\label{eq:axiomii}
    (x \cdot y)\cdot z=x \cdot (y \cdot z).
\end{equation}

\item[(iii)] Unit axiom: for each finite set $I$ and $x \in \tp[I]$, we have
\begin{equation*}
 x \cdot \iota_\emptyset(1) = x =  \iota_\emptyset(1) \cdot x,
\end{equation*}
where $1\in \mathbb{K}$ is the unit of the field $\mathbb{K}$.

\end{itemize}


\

A monoid $(\tp, \mu, \iota)$ in $(\Ssk, \cdot)$ is {\bf commutative} if
\begin{equation*}
    x\cdot y=y\cdot x,
\end{equation*}
for all $I=S \sqcup T$, $x \in \tp[S]$ and $y \in \tp[T]$.

\

Let $(\tp, \mu, \iota)$ be a monoid. From the associativity axiom, for any decomposition $I=S_1 \sqcup \cdots \sqcup S_k$ with $k \geq 2$ there is a unique map
\begin{equation}
    \mu_{S_1, \hdots, S_k}: \tp[S_1]\otimes \cdots \otimes \tp[S_k] \to \tp[I],
\end{equation}
called the {\bf higher product map} of $\tp$, obtained by iterating the product maps in any meaningful way. 
This is well defined from \eqref{eq:axiomii}.
We can extend the definition of higher product map for all $k\geq 0$: for $k=1$, $\mu_I$ is defined as the identity map of $\tp[I]$ ($I$ is the only decomposition of itself in one block); if $k=0$, then $\mu_{\emptyset }:=\iota_{\emptyset }$.
%Note how in this case $I = \emptyset $.
\

Monoids are closed under the Cauchy product. Also, if $(\tp, \mu, \iota)$ is a monoid, then $\tp[\emptyset]$ is an algebra with product $\mu_{\emptyset, \emptyset}$ and unit $\iota_\emptyset(1)$ (see \cite{AM2013}, section 2.3).


\

\subsubsection{Comonoids}
A {\bf comonoid} in $(\Ssk, \cdot)$ corresponds to the dual of a monoid in $(\Ssk, \cdot)$.
Specifically, a comonoid consists of a species $\tp$ equipped with morphisms of species
\begin{equation*}
    \Delta: \tp \to \tp \cdot \tp \qquad \text{ and } \qquad \varepsilon: \tp \to 1_{\mathbb{K}}.
\end{equation*}

That is, for each finite set $I$ and for each decomposition $I=S \sqcup T$, we have a linear map 
\begin{equation*}
    \Delta_{S,T}: \tp[I] \to \tp[S] \otimes \tp[T] \text{ and } \varepsilon_\emptyset: \tp[\emptyset] \to \mathbb{K}.
\end{equation*}

\

The collection of linear maps $\Delta=(\Delta_{S,T})$, called the {\bf coproduct} of the monoid, must satisfy the following axioms.

\

\begin{itemize}
    \item[(i)] Naturality axiom: for finite sets $I, J$, bijection $\sigma: I \to J$, a decomposition $I = S \sqcup T$ and an element $x\in \tp [I]$.
\begin{equation*}
(\tp[\sigma|_S] \otimes \tp[\sigma|_T])\circ\Delta_{S,T}(x)=\Delta_{\sigma(S), \sigma(T)} \circ\tp[\sigma](x).
\end{equation*}

\item[(ii)] Coassociativity axiom: for a finite set $I$, a decomposition $I=R \sqcup S \sqcup T$ and for each $x \in \tp[I]$, we have
\begin{equation}\label{eq:coaxiomii}
    (\Delta_{R,S}\otimes \text{id}_{\tp[T]})\circ \Delta_{R \sqcup S, T}(x)=(\text{id}_{\tp[R]} \otimes \Delta_{S,T})\circ \Delta_{R, S \sqcup T}(x).
\end{equation}
\vspace{.1in}
\item[(iii)] Counit axiom: for each finite set $I$ and $x \in \tp[I]$, we have
\begin{equation*}
(\varepsilon_\emptyset \otimes \text{id}_{\tp[I]})\circ\Delta_{\emptyset, I}(x)= x = (\text{id}_{\tp[I]} \otimes \varepsilon_\emptyset)\circ\Delta_{I,\emptyset}(x).
\end{equation*}

\end{itemize}


\

A comonoid $(\tp, \mu, \iota)$ in $(\Ssk, \cdot)$ is {\bf cocommutative} if for any finite disjoint sets $S, T$, and element $x\in \tp[S\sqcup T]$, we have
\begin{equation*}
    \Delta_{S,T}(x)=\Delta_{T,S}(x).
\end{equation*}

\

Let $(\tp, \mu, \iota)$ be a comonoid. Dually to the case of monoids, every decomposition $I=S_1 \sqcup \cdots \sqcup S_k$ with $k \geq 0$ gives rises to a unique linear map
\begin{equation}
    \Delta_{S_1, \hdots, S_k}: \tp[I] \to \tp[S_1]\otimes \cdots \otimes \tp[S_k],
\end{equation}
called the {\bf higher coproduct map} of $\tp$, obtained by iterating the coproducts map $\Delta_{S,T}$. 
This is well defined because of \cref{eq:coaxiomii}.
We extend this definition to $k=1$ as the identity of $\tp[I]$ and for $k=0$, this map is the counit map $\varepsilon_\emptyset$.

\

Comonoids are closed under the Cauchy product. 
If $(\tp, \Delta, \varepsilon)$ is a comonoid, then $\tp[\emptyset]$ is a coalgebra with coproduct $\Delta_{\emptyset, \emptyset}$ and counit $\varepsilon_\emptyset$ (see \cite{AM2013}, section 2.4).

\

\subsubsection{Bimonoids  and Hopf monoids}
A {\bf bimonoid} $(\thh, \mu, \Delta, \iota, \varepsilon)$ in $(\Ssk, \cdot)$ is a monoid and comonoid such that the diagram
\[\xymatrix{
\thh[A] \otimes \thh[B] \otimes \thh[C] \otimes \thh[D] \ar[rr]^-{\cong} && \thh[A] \otimes \thh[C] \otimes \thh[B] \otimes \thh[D] \ar[dd]^-{\mu_{A,C}\otimes\mu_{B,D}}\\
&&\\
\thh[S_1]\otimes \thh[S_2]\ar[uu]^-{\Delta_{A,B}\otimes \Delta_{C,D}} \ar[r]_-{\mu_{S_1, S_2}} & \thh[I] \ar[r]_-{\Delta_{T_1, T_2}} & \thh[T_1]\otimes \thh[T_2]
}\]
commutes, where $I=S_1\sqcup S_2=T_1 \sqcup T_2$ are two decompositions of a finite set $I$ with the following resulting pairwise intersections:
\[A:=S_1\cap T_1 \quad , \quad B:=S_1 \cap T_2 \quad , \quad C:=S_2 \cap T_1 \quad , \quad D:=S_2 \cap T_2.\]
This is also schematically presented in \cref{fig:bimonoid}.


\begin{figure}
\includegraphics[scale=0.5]{../images/diagram_bialgebra}
\caption{The bimonoid compatibility axiom.\label{fig:bimonoid}}
\end{figure}

\


A morphism of species $s: \thh \to \thh$, is called and {\bf antipode} of $\thh$, if $\thh[\emptyset]$ is a Hopf algebra with antipode $s_\emptyset: \thh[\emptyset] \to \thh[\emptyset]$, and for each nonempty set $I$, we have
\begin{equation}\label{eq:antipode_species}
    \sum_{S \sqcup T = I} \mu_{S,T}(\text{id}_S \otimes s_T)\Delta_{S,T} = 0 = \sum_{S \sqcup T = I} \mu_{S,T}(s_S \otimes \text{id}_T)\Delta_{S,T}.
\end{equation}


We define the convolution algebra $\text{End}_{\Ssk}(\thh)$ as the monoid of natural transformations $l: \thh \to \thh$ with the product $\star$.
Note how $\iota \circ \epsilon $ is the identity.
In this case, \eqref{eq:antipode_species} can be rephrased as $s = \id_{\thh}^{\star -1}$.

A {\bf Hopf monoid} in $(\Ssk, \cdot)$ is a bimonoid along with an antipode $s:\thh \to \thh $.
Recall that this is also the case in the classical Hopf algebras.

\

\subsection{Algebraic structures on set species}

The notions of monoid, comonoid, bimonoid and Hopf monoid for set species can be described in terms similar to those in the previous section. 

\

\subsubsection{Monoids}
A {\bf monoid in set species} is a set species $\rP$ equipped with morphisms of species
\begin{equation}
    \mu: \rP \cdot \rP \to \rP \qquad \text{ and } \qquad \iota: \mathrm{1}_{\{\emptyset \} } \to \rP.
\end{equation}

That is, for each decomposition $I=S \sqcup T$, we have maps 
\begin{equation}
    \mu_{S,T}: \rP[S] \times \rP[T]\to \rP[I] \text{ and } \iota_\emptyset: \{\emptyset\} \to \rP[\emptyset].
\end{equation}

If $x \in \rP[S]$, $y \in \rP[T]$,  let 
\[x \cdot y \in \rP[I]\]
denote the image of $x$ under $\mu_{S,T}$. Also, let $e\in \rP[\emptyset]$ denote the image of $\emptyset$ under $\iota_\emptyset$.

\

The collection of maps $\mu=(\mu_{S,T})$, called the {\bf product} of the comonoid, must satisfy naturality, associativity and unit axioms analogue to the ones defined for monoids in vector species.

\

Note that, for any monoid on a set species $(\rP, \mu, \iota)$, then $(\rP[\emptyset], \mu_{\emptyset, \emptyset}, \iota_\emptyset)$ is a set theoretical monoid.

\

A monoid in set species $(\rP, \mu, \iota)$ is {\bf commutative} if
\[x\cdot y=y\cdot x,\]
for all $I=S \sqcup T$, $x \in \rP[S]$ and $y \in \rP[T]$.

\

\subsubsection{Comonoids}
A {\bf comonoid in set species} consist of a species $\rP$ equipped with morphisms of species
\[\Delta:\rP \to \rP \cdot \rP \qquad \text{ and } \qquad \varepsilon: \rP \to \mathrm{1}_{\{\emptyset \} }. \]
That is, for each decomposition $I=S \sqcup T$ we have maps $\Delta_{S,T}: \rP[I]\to \rP[S] \times \rP[T]$, and $\varepsilon_\emptyset: \rP[\emptyset] \to \{\emptyset\}$. If $x \in \rP[I]$,  let 
\[(x|_S, x /_S)\in \rP[S]\times \rP[T]\]
denote the image of $(x,y)$ under $\Delta_{S,T}$. The map $x \mapsto x|_S$ can be thought as a ``restriction'' of the structure $x$ from $I$ to $S$, while $x \mapsto x/_S$ can be associated to a ``contraction'' of $S$ from $x$, resulting in a structure on $T$.

\

The collection of maps $\Delta=(\Delta_{S,T})$, called the {\bf coproduct} of the monoid, must satisfy naturality, coassociativity and counit axioms analogues to the ones defined for monoids in vector species.
These axioms can be described explicitly:
\begin{itemize}
    \item Naturality axiom: for each bijection $\sigma: I \to J$,  we have
\[{\Big (}\rP[\sigma](x){\Big )}\Big|_{\sigma(S)}=\rP[\sigma|_S](x|_S) \qquad , \qquad {\Big (}\rP[\sigma](x){\Big )}\Big/_{\sigma(S)}=\rP[\sigma|_T](x/_S),\]
for all $x \in \rP[I]$.
\vspace{.1in}
\item Coassociativity axiom: for all decomposition $I=R \sqcup S \sqcup T$, \[(x|_{R\sqcup S})|_R=x|_R \qquad , \qquad (x|_{R\sqcup S})/_R=(x/_R)|_S \qquad , \qquad x/_{R \sqcup S}=(x/_R)/_S, \]
for all $x \in \rP[I]$
\vspace{.1in}
\item Counit axiom: we have
\[x|_I= x = x/_\emptyset,\]
for each finite set $I$ and for each $x \in \rP[I]$. In particular, $\Delta_{\emptyset, \emptyset}(x)=(x,x)$, for each $x \in \rP[\emptyset]$.
\end{itemize}

\


A comonoid in set species $(\rP, \Delta, \varepsilon)$ is {\bf cocommutative} if
\[x|_S=x/_T,\]

for any disjoint finite sets $S, T$ and $x \in \rP[S\sqcup T]$.

\

\subsubsection{Bimonoids}
A bimonoid in set species $(\rH, \mu, \Delta, \iota, \varepsilon)$ is a monoid and comonoid in set species such that the diagram
\[\xymatrix{
\rH[A] \times \rH[B] \times\rH[C] \times \rH[D] \ar[rr]^-{\cong} && \rH[A] \times \rH[C] \times \rH[B] \times \rH[D] \ar[dd]^-{\mu_{A,C}\times\mu_{B,D}}\\
&&\\
\rH[S_1]\times \rH[S_2]\ar[uu]^-{\Delta_{A,B}\times \Delta_{C,D}} \ar[r]_-{\mu_{S_1, S_2}} & \rH[I] \ar[r]_-{\Delta_{T_1, T_2}} & \rH[T_1]\times \rH[T_2]
}\]
commutes, where $I=S_1\sqcup S_2=T_1 \sqcup T_2$ are two decompositions of a finite set $I$ with the following resulting pairwise intersections:
\[A:=S_1\cap T_1 \quad , \quad B:=S_1 \cap T_2 \quad , \quad C:=S_2 \cap T_1 \quad , \quad D:=S_2 \cap T_2.\]

The compatibility axiom in the definition of bimonoid can be reformulated as
\[x|_A \cdot y|_C= (x\cdot y)|_{T_1} \qquad , \qquad x/_A \cdot y/_C=(x \cdot y)/_{T_1},\]
for any disjoint sets $S_1, S_2$, for elements $x \in \rH[S_1]$, $y\in \rH[S_2]$ and for any set $T_1\subseteq S_1 \sqcup S_2$, by letting $A= S_1 \cap T_1$ and $C= S_2\cap T_1$.

\

A {\bf Hopf monoid} in $(\Ss, \cdot)$ $\rH$ is a bimonoid in set species such that the monoid $\rH[\emptyset]$ is a group.
Its antipode on $\emptyset $ is the map $s_\emptyset: \rH[\emptyset]\to \rH[\emptyset]$ given by $s_\emptyset(x):=x^{-1}$.
We define an antipode map $s: H\to H$ via the Takeuchi formula, adapted to set species.


\

\subsection{Fock functor}
In \cite{AM2010} (Part III), a construction is presented allowing to produce a (graded) Hopf algebra from a Hopf monoid. This is a categorical approach of a construction due to Stover (\cite{Stover}, Section 14), studied later by Patras, Schocker and Reutenauer in \cite{PR2004}, \cite{PS2006} and \cite{PS2008}.

\

We recall briefly this construction. Let $\mathbb{K}$ be a field of characteristic zero. If $\tp \in \Ssk$, then there is an action of the symmetric group $\mathfrak{S}_n$ on $\tp[n]$ by relabeling, for each $n \geq 0$. We denote by $\tp[n]_{\mathfrak{S}_n}$ the \emph{space of $\mathfrak{S}_n$-coinvariants of $\tp[n]$}:
\begin{equation}
  \tp[n]_{\mathfrak{S}_n}:=\tp[n]/\left\langle \, x- \tp[\alpha](x) \, | \, \alpha \in \mathfrak{S}_n, x\in \tp[n] \,\right\rangle.
\end{equation}

\


Consider $\gVec$ be the category of graded vector spaces over $\mathbb{K}$. The functors $\Kc, \Kcb: \Ssk \to \gVec$ given by 
\begin{equation}
    \Kc(\tp):=\bigoplus_{n \geq 0}\tp[n] \qquad , \qquad \Kcb(\tp):= \bigoplus_{n \geq 0}\tp[n]_{\mathfrak{S}_n}
\end{equation}
are referred in \cite{AM2010} as \emph{full Fock funtor} and \emph{bosonic Fock functor}, respectively. From any monoid (resp. comonoid, Hopf monoid) $\tp$, it is possible to obtain algebras (resp. coalgebras, Hopf algebras) $\Kc(\tp)$ and $\Kcb(\tp)$ from those of $\tp$, together with certain canonical transformations (see \cite{AM2010}, section 15.2).

\
