
\section{The antipode formula on special cases\label{sec:formula_pp}}
\subsection{The antipode formula for the pattern algebra on packed words\label{sec:formula_packed}}
%Let $\rho = (\leq_P, \leq_V)$ be a packed word.
For a partial order $\leq$ on a set $I$, and two sets $A, B \subseteq I$, recall that we say that $A \lneq B$ if $A\cap B = \emptyset$ and $a \leq b$ for any $a \in A$ and $b \in B$.
We recall the definition of non-interlacing QSS on packed words.


\begin{defin}[Interlacing QSS on packed words]
Let $\rho, \omega_1, \dots, \omega_n$ be packed words, where $\rho = (\leq_P, \leq_V)$ is a packed word on $I$.
Let $\III = (I_1, \dots, I_n)$ be a QSS of $\rho$ from $\omega_1, \dots, \omega_n$.
We say that this QSS is \textbf{non-interlacing} if there exists some $i = 1, \dots, n-1$ such that $I_i \lneq_P I_{i+1}$ and $I_i \lneq_V I_{i+1}$.
If no such $i$ exists, we say that the QSS is \textbf{interlacing}.

\

Additionally, let $ \bigl[\!\begin{smallmatrix} \rho  \\ \omega_1, \dots, \omega_n \end{smallmatrix}\!\bigr]$ be the number of interlacing QSS of $\rho$ from $\omega_1, \dots, \omega_n$.
\end{defin}

\

Our goal in this section is to analyse $\mathcal I^{\omega, \rho}_{\III}$, show that it always has a unique minimal element, and that it is precisely $\mathbb{1}$ whenever $\III$ is interlacing.
Recall that $\mu_i = (\underbrace{1, \dots , 1}_\text{$i-1$ times}, 2, 1, \dots, 1)$.

\

\begin{lm}\label{lm:QSSpackedWords}
Let $\omega, \rho = (\leq_P, \leq_V)$ be packed words, such that $\omega = \omega_1 \oplus \dots \oplus \omega_n$ is its factorisation into $\oplus$-indecomposibles.
Let $\III$ be a QSS of $\rho$ from $\omega_1, \dots , \omega_n$.
Then $\III$ is $\mu_i$-stable if and only if $I_i \lneq_P I_{i+1}$ and $I_i \lneq_V I_{i+1}$.
\end{lm}


\begin{proof}
Let us take care of the \textbf{forward direction} first.
From \cref{obs:pwords-characterisation}, we have that $I_i \lneq_P I_{i+1}$, therefore $I_i$ and $I_{i+1}$ are disjoint and $I_i$ is the unique ideal of $(\leq_P)|_{I_i \cup I_{i+1}}$ of size $|I_i|$.
We have that $\rho|_{I_i \cup I_{i+1}} \sim \omega_i \oplus \omega_{i+1}$, so the unique ideal of $(\leq_V)|_{I_i \cup I_{i+1}}$ of size $|I_i|$ is also an ideal of $(\leq_P)|_{I_i \cup I_{i+1}}$, so it has to be $I_i$.
We conclude that $I_i$ is an ideal of $(\leq_V)|_{I_i \cup I_{i+1}}$, so $I_i \lneq I_{i+1}$, concluding this part of the proof.

\

For the \textbf{backwards direction}, if $I_i \lneq_P I_{i+1}$ then $I_i$ and $I_{i+1}$ are disjoint.
Therefore, by the definition of $\ast $ in partial orders, 
$$((\leq_P)|_{I_i \cup I_{i+1}}) = ((\leq_P)|_{I_i} \ast (\leq_P)|_{I_{i+1}})\, . $$
Similarly for $\leq_V$.
So we conclude that 
$$\rho|_{I_i \cup I_{i+1}} = ((\leq_P)|_{I_i} \ast (\leq_P)|_{I_{i+1}}, (\leq_V)|_{I_i} \ast (\leq_V)|_{I_{i+1}}) = \rho|_{I_i} \oplus \rho|_{I_{i+1}}\, .  $$
If $\omega_i \sim \omega_{i+1}$, we also have that $I_i \lneq_P I_{i+1}$, so $\III$ is $\mu_i$-stable.
\end{proof}

\

\begin{lm}\label{lm:minpacked}
Let $\omega$ be  packed word, along with $\omega = \omega_1 \oplus \dots \oplus \omega_n$, its unique factorisation into $\oplus$-indecomposible packed words.
Let $\rho$ be another packed word, and $\III$ a QSS of $\rho$ from $\omega_1, \dots , \omega_n$. Then $\mathcal I^{\omega, \rho}_{\III}$ has a unique minimal element and, therefore, it is an interval in $\mathcal C_n$.
\end{lm}

\begin{proof}
%Recall that $\rho$ is a pair of orders in $I$.
%Notice that from the factorisation $\omega = \omega_1 \oplus \dots \oplus \omega_n$
We give a concrete description of $\mathcal I^{\omega, \rho}_{\III}$.
Consider the following set:
$$J = \{i \in [n-1] | I_i \lneq_P I_{i+1} \text{ and } I_i \lneq_V I_{i+1} \} = \{i \in [n-1] | \, \III \text{ is $\mu_i$-stable } \} \, ,$$
where the second equality is due to \cref{lm:QSSpackedWords}.
Let $\beta = \mathbf{CS}^{-1}( J)$. 
We claim that $\beta \in \mathcal I^{\omega, \rho}_{\III}$ and that this is its smallest element.

\
\begin{itemize}
\item {\bf That $\beta$ is in $\mathcal I^{\omega, \rho}_{\III}$} we prove now.\\
Indeed, we just need to establish that for each $I^{\beta}_i$ we have that $\rho|_{I^{\beta}_i} = x^{\beta}_i$.
Because $I^{\beta}_i = I_{f_{i-1} + 1} \cup \dots \cup I_{f_i}$, and $I_{f_{i-1} + 1} \lneq_P \dots \lneq_P I_{f_i}$, $I_{f_{i-1} + 1} \lneq_V \dots \lneq_V I_{f_i}$, we get that 
$$\rho|_{I^{\beta}_i} = \rho|_{I_{f_{i-1} + 1}} \oplus \dots \oplus \rho|_{I_{f_i}} = x_{f_{i-1} + 1} \oplus \dots\oplus x_{f_i} = x^{\beta}_i\,  ,$$

\

    \item {\bf That $\beta$ is the smallest element in $\mathcal I^{\omega, \rho}_{\III}$} we prove now. \\
Let $\alpha $ be a composition of $n$ such that $\alpha \not \geq \beta$.
Assume that $\alpha \neq \beta$. 
Then $J^c \cap \mathbf{CS}(\alpha)\neq \emptyset $, so pick some $i \in  J^c \cap \mathbf{CS}(\alpha) $.
Because $i \not\in J$, is such that $ I_i \not\lneq_P I_{i+1} \text{ or } I_i \not\lneq_V I_{i+1} $.

\

If $\omega_i \not\sim \omega_{i+1}$, one can see that $\rho|_{I_i \cup I_{i+1}} \neq \omega_i \oplus \omega_{i+1}$.
If $\omega_i \sim \omega_{i+1}$, stability would require $I_i \lneq_P I_{i+1}$, so $I_i \not\lneq_V I_{i+1}$, and we conclude again that $\rho|_{I_i \cup I_{i+1}} \neq \omega_i \oplus \omega_{i+1}$.

\

So we can never have $\rho|_{I^{\alpha}_j} = \omega^{\alpha}_j$ for $j$ such that $\omega^{\alpha}_j$ includes both $\omega_i \oplus \omega_{i+1}$ in its factorisation.
If $\omega_i \sim \omega_{i+1}$, stability would require $I_i \lneq_P I_{i+1}$, so $I_i \not\lneq_V I_{i+1}$, and we conclude again that $\rho|_{I_i \cup I_{i+1}} \neq \omega_i \oplus \omega_{i+1}$.
This is the contradiction that we are aiming for.
\end{itemize}
With the construction of the minimal element, we have that $\mathcal I^{\omega, \rho}_{\III}$ is an interval in $\mathcal{C}_n$.
\end{proof}

\

\begin{thm}\label{thm:antipode_packed}
Let $\omega $ be a packed word, and $\omega = \omega_1 \oplus \dots \oplus \omega_n$ be its decomposition into $\oplus$-indecomposible packed words.
Then, on the pattern Hopf algebra of packed words, we have the following cancellation free and grouping free formula:
$$S(\pat_{\omega}) = (-1)^n  \sum_{\rho} \bigl[\!\begin{smallmatrix} \rho  \\ \omega_1, \dots, \omega_n \end{smallmatrix}\!\bigr] \pat_{\rho}  \, .$$
\end{thm}


\begin{proof}
From \cref{thm:general_antipode}, we only need to establish that

\begin{equation}\label{eq:packed_alternating_sum}
\sum_{\alpha\in \mathcal I^{\rho, \omega}_{\III}} (-1)^{\ell (\alpha)} = (-1)^n \mathbb{1}[\III \text{ is interlacing QSS of $\rho$ from } \omega_1, \dots, \omega_n]\, .      
\end{equation}

\

Further, from \cref{prop:filter_structure_I} we know that the sum $\sum_{\alpha\in \mathcal I^{\rho, \omega}_{\III}}  (-1)^{\ell (\alpha)} $ vanishes whenever $\mathcal I^{\rho, \omega}_{\III}$ is an interval with more than one element.
From \cref{lm:minpacked}, we know that $\mathcal I^{\rho, \omega}_{\III}$ is indeed an interval.
The minimal interval is $\mathbb{1}$ if and only if $\III$ is an interlacing QSS from \cref{lm:minpacked}.
This concludes the proof.
\end{proof}

\

%From \cref{lm:minpacked}, we know that $\mathcal I^{\rho, \omega}_{\III}$ is indeed an interval, so the sum on the LHS of \eqref{eq:packed_alternating_sum} is zero except when $\mathbb{1} = \mathbf{CS}^{-1}([n-1]\setminus J)$, that is, when 
%$$\{i \in [n-1] | I_i <_P I_{i+1} \text{ and } I_i <_V I_{i+1} \} = \emptyset \, .$$

%This is precisely when $\III$ is an interlacing QSS.

\

Notice that this proof hides a sign-reversing involution in it.
Specifically, it was used in establishing \cref{prop:filter_structure_I}.

\

\subsection{The antipode formula for the pattern algebra on permutations\label{sec:formula_permutation}}
We start by recalling the definition of interlaced QSS on permutations.

\begin{defin}[Interlacing QSS on permutations]
Let $\sigma, \pi_1, \dots, \pi_n$ be permutations, where $\sigma = (\leq_P, \leq_V)$ is a permutation on $I$.
Let $\III = (I_1, \dots, I_n)$ be a QSS of $\sigma$ from $\pi_1, \dots, \pi_n$.
We say that $\III$ is \textbf{non-interlacing} if there exists $i = 1, \dots, n-1$ such that $I_i \lneq_P I_{i+1}$ and $I_i \lneq_V I_{i+1}$.
If no such $i$ exists, we say that the QSS is \textbf{interlacing}.

\

Additionally, let 
$ \bigl[\!\begin{smallmatrix} \sigma \\ \pi_1, \dots, \pi_n \end{smallmatrix}\!\bigr]$ be the number of interlacing QSS of $\sigma$ from $\pi_1, \dots, \pi_n$
\end{defin}

\

\begin{thm}\label{thm:antipode_perms}
Let $\pi $ be a permutation, and $\pi = \pi_1 \oplus \dots \oplus \pi_n$ be its decomposition into irreducible permutations.
Then, on the pattern Hopf algebra of permutations, we have the following cancellation free and grouping free formula:
$$S(\pat_{\pi}) = (-1)^n \sum_{\sigma} \bigl[\!\begin{smallmatrix} \sigma \\ \pi_1, \dots, \pi_n \end{smallmatrix}\!\bigr] \pat_{\sigma} \, .$$
\end{thm}

\

Although we can obtain the antipode formula by showing that a relevant poset of compositions is an interval (see \cref{lm:minpacked}), we present here a different proof.
Specifically, we will be leveraging a map $\mathcal A[\mathrm{inc}]: \mathcal A(\mathtt{PW}) \to \mathcal A(\mathtt{Per})$ and the previous result on packed words.

\

First, observe that any permutation is a packed word, because any pair of total orders $(\leq_P, \leq_V)$ is a packed word, that is, a pair of partial orders such that $\leq_P$ is a total order and $\leq_V$ is partial linear order.
This gives us an inclusion map $\mathrm{inc} : \mathtt{Per} \to \mathtt{PW}$ that preserves restrictions and the monoidal structure.
Therefore, this gives us a surjective Hopf algebra morphism $\mathcal A[\mathrm{inc}]: \mathcal A(\mathtt{PW}) \to \mathcal A(\mathtt{Per})$.


\begin{proof}
From \cref{thm:antipode_packed}, we know that for any permutation $\pi$ seen as a packed word we have that
$$S(\pat_{\pi}) = (-1)^n \sum_{\omega} \bigl[\!\begin{smallmatrix} \omega \\ \pi_1, \dots, \pi_n \end{smallmatrix}\!\bigr] \pat_{\omega}\, . $$

If $\omega$ is a packed word, we compute $\mathcal A[\,\mathrm{inc}] (\pat_{\omega}) = \pat_{\omega}\mathbb{1}[\omega \text{ is a permutation }]$.
Thus, applying $\mathcal A[\mathrm{inc}]$ to both sides of the equation above, we get the desired result.
Note that $\mathcal A[\mathrm{inc}]$ is a Hopf algebra morphism from \cref{thm:functoriality}, so it commutes with the antipode.
\end{proof}
%
%Observe that we also have a map $\mathrm{st}: \mathtt{PW} \to \mathtt{Per}$.
%This gives us an injective map $\mathcal A[\mathrm{st}]: \mathcal A[\mathtt{Per}] \to  \mathcal A[\mathtt{PW}] $ defined as follows
%$$\mathcal A[\mathrm{st}] (\pat_{\pi} ) = \sum_{\mathrm{st}(\omega ) = \pi} \pat_{\omega}.$$
%We can also use this map to find an antipode formula for the permutation pattern Hopf algebra.
%\todo[inline]{Extend this remark.}
%
%\todo[inline]{what follows is the old proof, kept here for legacy purposes, and because some definitions needed above are still buried in this section, a keen eye will need to be used to parse them}
%
%
%Fix a permutation $\pi $, such that $\pi = \pi_1 \oplus \dots \oplus \pi_k $ is its $\oplus$-factorisation into $\oplus$-indecomposable permutations.
%We will assume this notation for the remaining of this section.
%
%The intermediate step to conclude \cref{thm:cancellationformAPer} is \cref{lm:interlacingcefsformula}, which is a description of the interlacing coefficients via an inclusion-exclusion sum.
%In order to establish this, we define a classical poset structure in compositions, which we will see is isomorphic to a Boolean poset.
%This allows us to simplify sums quite drastically, which leads us to the desired formula.
%
%\begin{defin}[Order on compositions]\label{defin:ordercomps}
%We can endow the set of compositions of $n$ with an order $\leq $ as follows:
%We say that $\alpha \leq \beta $ if $\beta $ results from $\alpha $ by merging two consecutive entries.
%By taking the transitive closure, we have a poset in $\mathcal C_n$.
%Note that $(n)$ is the unique minimal element, whereas $(1, \dots , 1)$ is the unique maximal element.
%\end{defin}
%
%%\begin{rem}
%%This results in the \textbf{opposite} order of the coarsening order, defined in \cref{defin:coarseningorder} for set partitions and set compositions, and extended in \cref{lm:ordercomps} to an order $\leq'$ in compositions.
%%\end{rem}
%
%We now describe the bijection between $\mathcal C_n$, the set of compositions of $n$, and the Boolean poset $[k-1]$.
%
%\begin{defin}[Cumulative sum map]
%Assume $n \geq 1$.
%Then we define an explicit bijection between subsets of $[n-1]$ and the set $\mathcal C_n$: if $\alpha=(\alpha_1, \dots ,\alpha _j)$ is a set composition, then we define
%$$ \mathbf{CS}(\alpha ) = \{\alpha_1, \alpha_1 + \alpha_2, \dots , \alpha_1 +\dots + \alpha_{j-1}\}\, . $$
%This is a subset of $[n-1]$.
%That this is a bijection is immediate, as an inverse can be readily constructed.
%
%We further define the integers $ \mathbf{CS}(\alpha )_1 < \mathbf{CS}(\alpha )_2 < \dots < \mathbf{CS}(\alpha )_{j-1}$ such that 
%$$ \mathbf{CS}(\alpha ) =\{ \mathbf{CS}(\alpha )_1 ,  \mathbf{CS}(\alpha )_2 ,  \dots ,  \mathbf{CS}(\alpha )_{j-1} \}\, . $$
%and set $\mathbf{CS}(\alpha )_0 = 0 $ and $ \mathbf{CS}(\alpha )_j = n $.
%
%Furthermore, this map is a poset isomorphism, where it maps $\leq $, the partial order in $\mathcal C_n$ introduced in \cref{defin:ordercomps}, to the inclusion of sets.
%Details on this map are given in \cite{stanley00}.
%
%Recall that we fix a permutation $\pi $ with $\pi = \pi_1 \oplus \dots \oplus \pi_k $.
%Given a composition $\alpha \models n$, and an integer $i \in \{ 1 , \dots , l(\alpha )\}$, we write 
%$$\pi^{(i)}_{\alpha } = \pi_{\mathbf{CS}(\alpha )_{i-1}+1} \oplus \dots \oplus \pi_{\mathbf{CS}(\alpha )_{i}}\, . $$
%This notation can be further extended to weak compositions $\alpha $ by setting $\pi^{(i)}_{\alpha } = \pi^{(i)}_{\beta }$, where $\beta$ is the composition resulting from erasing the zeros from $\alpha $.
%\end{defin}
%
%
%
%
%\begin{lm}\label{lm:interlacingcefsformula}
%$$\bigl[\!\begin{smallmatrix} \sigma  \\ \pi_1, \dots ,\pi_k \end{smallmatrix}\!\bigr] =  \sum_{\alpha\models n} (-1)^{l(\alpha )} \binom{\sigma }{\pi^{(1)}_{\alpha }, \dots ,\pi^{(l(\alpha ))}_{\alpha }}\, . $$
%\end{lm}
%
%We now see that this result suffices to establish the main result of this section.
%
%\begin{proof}[Proof of \cref{thm:cancellationformAPer}]
%We simply apply Takeuchi's formula and use \cref{lm:interlacingcefsformula} when needed:
%\begin{equation}
%\begin{split}
%S(\pat_{\pi}) &= \sum_{j=0}^k (-1)^j \mu^{ \circ j-1} \circ ( \id_{\mathcal A ( \mathtt{Per} )} - \iota \circ \varepsilon)^{ \otimes j}\circ \Delta^{\circ j-1}( \pat_{\pi}) \\
%&= \sum_{\alpha\models^0 k} (-1)^{l(\alpha )} \mu^{\circ l(\alpha )-1 } \circ ( \id_{\mathcal A ( \mathtt{Per} )} - \iota \circ \varepsilon)^{ \otimes l(\alpha )}  ( \pat_{\pi^{(1)}_{\alpha }} \otimes \dots \otimes \pat_{\pi^{(l(\alpha ))}_{\alpha } }) \\
%&= \sum_{\alpha\models k} (-1)^{l(\alpha )} \mu^{\circ l(\alpha )-1 }  ( \pat_{\pi^{(1)}_{\alpha }} \otimes \dots \otimes \pat_{\pi^{(l(\alpha ))}_{\alpha } })= \sum_{\alpha\models k} (-1)^{l(\alpha )}  \pat_{\pi^{(1)}_{\alpha }}  \dots  \pat_{\pi^{(l(\alpha ))}_{\alpha } } \\
%&= \sum_{\alpha\models k} (-1)^{l(\alpha )} \sum_{\sigma } \pat_{\sigma} \binom{\sigma}{ \pat_{\pi^{(1)}_{\alpha }} ,  \dots  , \pat_{\pi^{(l(\alpha ))}_{\alpha } } } \\
%&=  \sum_{\sigma } \pat_{\sigma} \sum_{\alpha\models k} (-1)^{l(\alpha )} \binom{\sigma}{ \pat_{\pi^{(1)}_{\alpha }} ,  \dots  , \pat_{\pi^{(l(\alpha ))}_{\alpha } } } =  \sum_{\sigma } \pat_{\sigma} \bigl[\!\begin{smallmatrix} \sigma  \\ \pi_1, \dots ,\pi_k \end{smallmatrix}\!\bigr] \, ,
%\end{split}
%\end{equation}
%where in the last equality we used \cref{lm:interlacingcefsformula}.
%\end{proof}
%
%In order to establish \cref{lm:interlacingcefsformula}, we proceed as follows:
%For a composition $\alpha$, we introduce a notion of $\alpha$-QSS of a permutation $\sigma$.
%We see that any interlacing QSS of $\sigma$ cannot be \say{extended} to a non-trivial $\alpha$-QSS of a permutation $\sigma$.
%Finally, we see that in \cref{lm:posetsumcancelation}, all the QSS of $\sigma$ that can be extended will cancel its contribution to the interlacing quasi-shuffle coefficient.
%
%\begin{defin}
%Fix a permutation $\sigma $ and a composition $\alpha $.
%We say that $\III$ is an $\alpha$-QSS of $\sigma$ from $\pi_1, \dots ,\pi_k$ if $\III $ is a QSS of $\sigma $ from $\pi^{(1)}_{\alpha }, \dots ,\pi^{(l(\alpha ))}_{\alpha } $.
%When the permutations $\pi_1, \dots ,\pi_k$ are clear from context, we simply write that $\III $ is an $\alpha$-QSS of $\sigma$.
%
%Assume that $\III$ is an $\alpha$-QSS of $\sigma$, then we can construct a canonical QSS of $\sigma $, which we write  $\JJJ = \sigma(\III)$, as follows:
%For each $i= 1 , \dots , l(\alpha )$, let $J^{(i)}_1, \dots , J^{(i)}_{\alpha_i} $ be the unique choice of sets such that 
%\begin{itemize}
%\item $\biguplus_{j=1}^{\alpha_i} J_j^{(i)} = I_i$;
%
%\item $J_j^{(i)}< J_{j+1}^{(i)} $ for any $j=1 , \dots , \alpha_i-1 $;
%
%\item $\sigma|_{J_j^{(i)}} = \pi_{\mathbf{CS}(\alpha)_{i-1} + j}$ for any $j=1 , \dots , \alpha_i $.
%\end{itemize}
%
%This is possible, because $\sigma|_{I_i} = \pi_{\alpha }^{(i)} = \pi_{\mathbf{CS}(\alpha )_{i-1}+1} \oplus \dots \oplus \pi_{\mathbf{CS}(\alpha )_{i}} $.
%Then, by letting $\JJJ = (J^{(1)}_1, \dots , J^{(1)}_{\alpha_1},  J^{(2)}_1, \dots ) $ we obtain a QSS of $\sigma $.
%
%
%
%Finally, fix a permutation $\sigma $, and let $\JJJ $ be a QSS of $\sigma $.
%Then, define 
%$$\mathcal I^{\sigma, \pi }_{\JJJ} = \{\alpha \models k \, \,  | \, \,  \exists \, \III \text{ QSS of } \sigma \text{ s.t. } \sigma(\III ) = \JJJ  \}\, . $$
%\end{defin}
%
%
%
%\begin{lm}
%Fix a permutation $\pi$ and $\sigma $, and let $\pi_1, \dots ,\pi_k$ be as above.
%Given $\JJJ$ a QSS of $\sigma$, and $\alpha$ a composition of $k$, then there is at most one $\III$ that is an $\alpha$-QSS of $\sigma$ such that $\sigma(\III ) = \JJJ$.
%
%Furthermore, $\JJJ $ is a $(1, \dots , 1)$-QSS of $\sigma$ and $\sigma(\JJJ ) = \JJJ$.
%\end{lm}
%
%\begin{proof}
%That $\JJJ $ is a $(1, \dots , 1)$-QSS of $\sigma$ and $\sigma(\JJJ ) = \JJJ$ is immediate by definition.
%The fact that there is at most one such $\III $, follows from the fact that the procedure $\sigma ( \III ) = \JJJ $ is invertible if we know $\alpha $: if $\sigma ( \III ) = \JJJ $, then $\III $ results from $\JJJ =(J_1, \dots , J_k)$ by taking the union of consecutive sets $J_i$ according to $\alpha $.
%Thus it follows that such QSS $\III $ is unique.
%\end{proof}
%
%\begin{rem}
%Having established the uniqueness, one can wonder if we can also guarantee the existence.
%That is in fact not the case, because the procedure of \say{taking the union of consecutive sets $J_i$ according to $\alpha $} does not guarantee that this union will satisy the properties of QSS, namely that $\sigma|_{I_i}= \pi^{(i)}_{\alpha }$.
%
%In fact, we will see now that if $\JJJ $ is an interlacing QSS of $\sigma $, then there is no such $\alpha $ and such $\III$, except the trivial choice $\III= \JJJ$ and $\alpha = (1, \dots , 1)$.
%\end{rem}
%
%The following lemma is crucial in establishing \cref{lm:interlacingcefsformula}.
%
%\begin{lm}\label{lm:noninterlacingcrit}
%Fix a permutation $\pi$ and $\sigma $, and let $\pi_1, \dots ,\pi_k$ be as above.
%Fix $\JJJ$ a QSS of $\sigma $.
%Then $|\mathcal I^{\sigma, \pi }_{\JJJ} | = 1 $ if and only if $\JJJ $ is interlacing QSS of $\sigma$.
%\end{lm}
%
%\begin{proof}
%Assume that $\JJJ = (J_1, \dots , J_k ) $ is non-interlacing QSS of $\sigma$.
%Then, there is some $i= 1, \dots , k-1 $ such that $J_i< J_{i+1}$ and $\sigma(J_i) < \sigma (J_{i+1})$.
%We claim that $\alpha = (\underbrace{1, \dots ,1 }_{i-1 \text{ times}}, 2, \underbrace{1, \dots ,1 }_{k-i-1 \text{ times}}) \in \mathcal I^{\sigma, \pi }_{\JJJ}$.
%
%
%In fact, we have that $\III = (J_1, \dots J_{i-1}, J_i\cup J_{i+1}, J_{i+2}, \dots , J_k ) $ is precisely the desired $\alpha$-QSS, since we have by construction that $\sigma (\III ) = \JJJ$.
%Further, the non-interlacing condition gives us that $\sigma|_{ J_i\cup J_{i+1}} = \pi_i\oplus\pi_{i+1}$.
%
%Thus $|\mathcal I^{\sigma, \pi }_{\JJJ} | \neq 1 $, as desired.
%
%\todo[inline]{from this part until the YVY I still don't like how it looks}
%
%On the other hand, if $\JJJ $ is interlacing, suppose that there is some non-trivial $\alpha \in \mathcal I^{\sigma, \pi }_{\JJJ}$, and consider $\III $ its corresponding $\alpha$-QSS of $\sigma$.
%Then $\JJJ = \sigma (\III )$, and this contradicts the assumption that $\JJJ $ is interlacing by construction of $\sigma (\III)$.
%\end{proof}
%
%\begin{lm}\label{lm:principalideal}
%The set $\mathcal I^{\sigma, \pi }_{\JJJ} $ is an ideal with a unique minimum.
%\end{lm}
%
%Because we know that $(1, \cdots , 1) \in \mathcal I^{\sigma, \pi }_{\JJJ} $, this means that $\mathcal I^{\sigma, \pi }_{\JJJ} $  is in fact an interval.
%
%\begin{proof}
%Let $P=\{i \in [k-1]| J_i < J_{i+1} \text{ and } \sigma(J_i) < \sigma( J_{i+1}\}\}$.
%We claim that $\mathcal I^{\sigma, \pi }_{\JJJ} = \{ \alpha \models [k-1] | \mathbf{CS}(\alpha ) \cup P = [k-1] \}$.
%That is, $\mathbf{CS}^{-1}([k-1]\setminus P ) $ is the unique minimal element in $\mathcal I^{\sigma, \pi }_{\JJJ} $
%
%First, let $\alpha \geq \mathbf{CS}^{-1}([k-1]\setminus P ) $.
%We observe that it is straightforward to construct an $\alpha$-QSS of $\sigma$, say $\III $, such that $\sigma(\III) = \JJJ$.
%The definition of $P$ gives us that we indeed obtain an $\alpha$-QSS of $\sigma$.
%
%On the other hand, suppose that $\JJJ = \sigma (\III )$, where $\III$ is an $\alpha$-QSS of $\sigma$ such that $\mathbf{CS}(\alpha ) \cup P \neq [k-1] $.
%Say $j\in [k-1] $ is such that $j \not \in\mathbf{CS}(\alpha ) \cup P$.
%Then either $J_j \not< J_{j+1}$ or $\sigma(J_j) \not< \sigma (J_{j+1})$.
%
%However, from $j\not\in \mathbf{CS}(\alpha )$, we know that there is some $i\in \{1, \dots , l(\alpha ) \}$ such that $J_j, J_{j+1}\subseteq I_i$.
%From $\JJJ = \sigma (\III )$ we know that $J_j, J_{j+1}$ are obtained from the $\oplus$-decomposition of $\sigma|_{I_i}$.
%This contradicts the assumption that either $J_j \not< J_{j+1}$ or $\sigma(J_j) \not< \sigma (J_{j+1})$.
%\end{proof}
%
%\begin{lm}\label{lm:posetsumcancelation}
%Fix a permutation $\pi$ and $\sigma $, and let $\pi_1, \dots ,\pi_k$ be as above.
%Fix $\JJJ$ a QSS of $\sigma $.
%Then
%$$\sum_{\alpha \in \mathcal I^{\sigma, \pi }_{\JJJ}}(-1)^{ l(\alpha )} = \begin{cases}
%(-1)^k, \text{ if } |\mathcal I^{\sigma, \pi }_{\JJJ} | = 1, \\
%0, \text{ otherwise }
%\end{cases} \, . $$
%\end{lm}
%
%\begin{proof}
%The case where $|\mathcal I^{\sigma, \pi }_{\JJJ} | = 1$ is trivial, because in this case we have $\mathcal I^{\sigma, \pi }_{\JJJ} = \{\underbrace{(1, \dots ,1 )}_{k \text{ times}}\}$.
%Otherwise, from \cref{lm:principalideal} we have that $\mathcal I^{\sigma, \pi }_{\JJJ} $ is an ideal with a unique minimum that contains $\{\underbrace{(1, \dots ,1 )}_{k \text{ times}}\}$.
%Thus, $\mathbf{CS} ( \mathcal I^{\sigma, \pi }_{\JJJ}  ) $ is an ideal with a unique minimum, so there is a set $M \subsetneq [k-1] $ such that 
%$$\mathbf{CS} ( \mathcal I^{\sigma, \pi }_{\JJJ}  ) = \{K \subseteq [k-1] |  M \subseteq K \} \, . $$
%
%Thus, we have that
%\begin{equation}
%\begin{split}
%\sum_{\alpha \in \mathcal I^{\sigma, \pi }_{\JJJ}}(-1)^{ l(\alpha )} =& \sum_{\substack{M\subseteq \mathbf{CS} (\alpha )\\ \mathbf{CS} (\alpha ) \subseteq [k-1]}} (-1)^{l(\alpha )}\\
%=& \sum_{\substack{M\subseteq K\\K \subseteq [k-1]}} (-1)^{|K |+1} = 0\, , 
%\end{split}
%\end{equation}
%where the last equality is a simple binomial identity.
%\end{proof}
%
%With this we can finally prove \cref{lm:interlacingcefsformula}, which concludes the proof of the cancellation-free formula of the antipode of $\mathcal A (\mathtt{Per})$.
%
%\begin{proof}[Proof of \cref{lm:interlacingcefsformula}]
%
%We use \eqref{eq:qscoefasQSS} to develop the sum at hand, yielding:
%\begin{equation}
%\begin{split}
%\sum_{\alpha\models n} (-1)^{l(\alpha )} &\binom{\sigma }{\pi^{(1)}_{\alpha }, \dots ,\pi^{(l(\alpha ))}_{\alpha }} =  \sum_{\alpha\models n}  (-1)^{l(\alpha )} \sum_{\III \text{ is } \alpha-\text{QSS of $\sigma$}} 1  \\
% =&  \sum_{\alpha\models n} (-1)^{l(\alpha )} \sum_{\JJJ \text{ is QSS of $\sigma$}} \mathbb{1}[\exists \, \, \III \, \alpha-\text{QSS s.t. } \sigma(\III) = \JJJ]   \\
% =& \sum_{\JJJ \text{ is QSS of $\sigma$}} \sum_{\substack{\alpha\models n\\ \alpha \in \mathcal I^{\sigma, \pi}_{\JJJ}}}  (-1)^{l(\alpha )}  \\
%  =& \sum_{\substack{\JJJ \text{ is QSS of $\sigma$}\\ |\mathcal I^{\sigma, \pi}_{\JJJ}| = 1}} (-1)^k\, , \\
%\end{split}
%\end{equation}
%where in the last sum we used \cref{lm:posetsumcancelation}.
%
%Thus, we are to compute
%$$|\{ \JJJ \text{ is QSS of $\sigma$ s.t.} |\mathcal I^{\sigma, \pi}_{\JJJ}| = 1 \} |\, , $$
%which is precisely $ \bigl[\!\begin{smallmatrix} \sigma  \\ \pi_1, \dots ,\pi_k \end{smallmatrix}\!\bigr]$, according to \cref{lm:noninterlacingcrit}.
%This concludes the proof.
%\end{proof}
%


\
