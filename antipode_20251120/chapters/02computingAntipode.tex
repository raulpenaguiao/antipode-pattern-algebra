
\section{Computing the antipode\label{sec:antipode_computing}}

In this section we explore the antipode of a Hopf algebra using Takeuchi's formula.
We explore examples on the polynomial algebra and on the permutation pattern Hopf algebra.
We start by recalling Takeuchi's formula, in the form that is presented in \cite{GrinbergReiner}, as well as some convenient notation.
%This is the beginning of any cancellation-free formula.
Let us define the $\star$ notation on linear maps $a, b: C \to A$.
Whenever $A$ is an algebra, and $C$ is a coalgebra, we define:
$$a \star b \coloneqq \mu_A \circ (a \otimes b) \circ \Delta_C\, ,$$
which is an associative and unitary product on linear maps from $C$ to $A$. We will be focused on the case when $C=A=H$ is a Hopf algebra, so this defines a convolution operation on $\End(H)$.

The following result is from \cite[Lemma 14]{Takeuchi1971}.

\begin{prop}[Takeuchi's formula]\label{lm:takeuchi}
If $H = (H, \mu, \iota, \Delta, \epsilon, S)$ is a Hopf algebra such that $(\iota\circ \epsilon - \id_H)$ is $\star$-nilpotent (such as in any filtered Hopf algebra), then 
\begin{equation}\label{eq:eq1}
S = \sum_{k\geq 0 }  ( \iota  \circ\epsilon- \id_H)^{\star k} = \sum_{k\geq 0} (-1)^k \mu^{\circ (k-1)} \circ (\id_{H} - \iota \circ \epsilon)^{\otimes k} \circ \Delta^{\circ (k-1)}\, ,
\end{equation}
where we use the convention that $\Delta^{\circ (-1)} = \epsilon $ and $\mu^{\circ (-1)} = \iota$.
Notice that the $\star$-nilpotent property ensures that this sum is finite.
\end{prop}

Note that, any pattern algebra is a filtered Hopf algebra, so for our objects of study we can always apply Takeuchi's formula.
%This is summarized in \cref{thm:conHopfalgebra}.

\

In the Hopf algebra of polynomials, this gives us the following:
$$S(x^3) = \underbrace{0}_{k = 0} - \underbrace{x^3}_{k = 1} + \underbrace{3 x^2 \cdot x + 3 x \cdot x^2}_{k=2} - \underbrace{6 x \cdot x \cdot x}_{k = 3} = - x^3 \, .$$


We now present another example, this time on the permutation pattern Hopf algebra $\mathcal A(\mathtt{Per})$.
Consider $\pi = 132 = 1 \oplus 21$. Then Takeuchi's formula gives us:
\begin{align*}
S(\pat_{132}) =& \sum_{k=0}^2 (-1)^k \mu^{\circ (k-1)} \circ (\id_{\mathcal A(\mathtt{Per})} - \iota \circ \epsilon)^{\otimes k} \circ \Delta^{\circ (k-1)}(\pat_{132})\\
=& -(\id_{\mathbb{K}[x]} - \iota \circ \epsilon)(\pat_{132}) + \mu \circ (\id_{\mathbb{K}[x]} - \iota\circ\epsilon)^{\otimes 2}(\pat_1 \otimes \pat_{12}) \\
=& - \underbrace{\pat_{132}}_{k=1} + \underbrace{\pat_1 \pat_{21}}_{k=2} \\
=& 3 \pat_{321} + 2 \pat_{231} + 2 \pat_{312} + \pat_{213} + 2 \pat_{21} \, .
\end{align*}

\

These coefficients can be seen as enumerating quasi-shuffle signatures of $132$ from $1$ and $12$ that are \textbf{interlacing}, according to \cref{thm:antipode_perms_intro}.



\

\subsection{The sign-reversing involution method}

The application of the sign-reversing involution method to compute antipodes of Hopf algebras was first presented in \cite{BS2017}.
This is a method to find cancellation-free formulas for the antipode of a Hopf algebra.
It starts in the formula given by Takeuchi, and results in a $\pm1$ sum that runs over a collection of objects, say $\mathcal O$.
An involution $\zeta $ is an endomorphism such that $\zeta \circ \zeta$ is the identity.
A sign-reversing involution $\zeta $, is an involution on $\mathcal O$ such that if $\zeta(x) \neq x$, then $x$ and $\zeta(x)$ contribute with opposite signs to the sum over $\mathcal O$ that results in the antipode.
As a consequence, when applying Takeuchi's formula, we can cancel terms that are not fixed points of $\zeta$.

%keeps track of all the terms to be summed, usually by means of compositions, that arise in this formula.
%Thus, the sum obtained runs over a collection of objects, say $\mathcal O$, that is partitioned into families indexed by compositions.
%These compositions play an important role in the sum, as their length determines the sign of the corresponding objects.

%\
%
%Recall that an involution $\zeta $ is an endomorphism such that $\zeta \circ \zeta$ is the identity.
%We describe an involution in $\mathcal O$, call it $\zeta $, in such a way that if $\zeta(x) \neq x$, then $x$ and $\zeta(x)$ contribute with opposite signs to the antipode.

\

%We give an example of how this method is applied in the Hopf algebra $\mathbb{K}[x]$.
%The following is a computation from \cite{BS2017}.
%There are easier ways to obtain an antipode formula for the polynomial Hopf algebra: the interested reader can find some for instance in \cite{GrinbergReiner}.
%However, this example shows the power of sign-reversing involutions, as the whole process can be done with little recourse to intuition after picking the right involution.

An example of how this method is applied in the Hopf algebra $\mathbb{K}[x]$ is given in \cite{BS2017}.


\begin{thm}[The antipode formula for the polynomial Hopf algebra]\label{thm:polyHA}
The antipode $S$ for $\mathbb{K}[x] $ is 
$$ S(x^n) =(-x)^n\, . $$
\end{thm}
%
%To prove this formula, we first introduce \textit{weak compositions}.
%A \textbf{weak composition} $\alpha$ of an integer $n$ is a list of non-negative integers $\alpha = (\alpha_1, \dots , \alpha_l)$ such that $\sum_i \alpha_i = n$.
%We denote the length of the weak composition by $\ell = \ell(\alpha)$, and use the shorthand notation $\alpha \models^0 n$.
%If $\alpha $ has no zero entries, we say that $\alpha$ is a \textbf{composition}, and write $\alpha \in \mathcal C_n$.
%
%\
%
%A \textbf{weak set composition} $\opi$ of a set $A$ is a list of pairwise disjoint sets $\opi = (A_1, \dots , A_{\ell})$ such that $\bigcup_i A_i = A$.
%Note that some sets may be empty.
%We denote the length of the weak composition by $\ell = \ell(\opi)$, and use the shorthand notation $\opi \models^0 A$ to indicate that $\opi$ is a weak set composition of the set $A$.
%If the weak set composition does not contain any empty set, we say that this is a \textbf{set composition}, and we denote $\opi \models A$ to indicate that $\opi$ is a set composition of the set $A$.
%There are finitely many set compositions of a given set $A$, whereas there are infinitely many weak set compositions of $A$.
%Let $\mathbf{C}_A$ be the collection of set composition of $A$.
%We abbreviate to $\mathbf{C}_n$ when $A = [n]$.
%
%\begin{proof}[Proof of \cref{thm:polyHA}]
%We use Takeuchi formula, given in \eqref{eq:eq1}, which holds for graded Hopf algebras,
%$$S(x^n) = \sum_{k = 0} (-1)^k \mu^{\circ (k-1)} \circ (\id_{\mathbb K[x]} - \iota \circ \epsilon )^{\otimes k} \circ \Delta^{\circ (k-1)} (x^n) \, .$$
%
%The following results from induction on $k$:
%$$ \Delta^{\circ (k-1) } (x^n) = \sum_{(A_1, \dots , A_k) \models^0 [n] } x^{|A_1|} \otimes \dots \otimes x^{|A_k|} \, ,$$
%where the sum runs over weak set compositions of the set $[n]$ with lenght $k$.
%Thus, the antipode formula can be rewritten as
%
%\begin{align*}
%S(x^n)&= \sum_{k=0}^n(-1)^k \sum_{(A_1, \dots , A_k) \models^0 [n]} \mu^{\circ (k-1)} \circ (\id_{\mathbb K[x]} - \iota \circ \epsilon )^{\otimes k} (x^{|A_1|} \otimes \dots \otimes x^{|A_k|})   \\
%	  &= \sum_{k=0}^n(-1)^k \sum_{(A_1, \dots , A_k) \models [n]} \mu^{\circ (k-1)} (x^{|A_1|} \otimes \dots \otimes x^{|A_k|})   \\
%	  &= \sum_{k=0}^n(-1)^k \sum_{(A_1, \dots , A_k) \models [n]} x^n   \\
%	  &= x^n \sum_{\opi \models [n]} (-1)^{\ell(\opi)}\, .
%\end{align*}
%
%Consider the following involution $\zeta: \oPi_{n} \to \oPi_{n} $.
%For $\opi = (A_1, \dots , A_k) $, let $j_{\opi}$ be the smallest index such that $|A_{j_{\opi}}| \neq 1$ or $\max A_{j_{\opi}} > \max A_{j_{\opi}+1} $.
%Then, there are three cases:
%
%\begin{enumerate}
%
%\item The set $A_{j_{\opi}} $ is a singleton with $j_{\opi}\leq k-1$, then let $\zeta(\opi ) $ be the set composition resulting from merging $A_{j_{\opi}} $ and $A_{j_{\opi}+1}$.
%Note how, in this case, $\max ( A_{j_{\opi}} \cup A_{j_{\opi} + 1} ) $ is the only element in $A_{j_{\opi}}$.
%
%\item The set $A_{j_{\opi}} $ is not a singleton, then we define $\zeta(\opi ) $ to be the set composition resulting from splitting $A_{j_{\opi}} $ into $\{ \max A_{j_{\opi}} \} $ and $A_{j_{\opi}} \setminus \{\max A_{j_{\opi}} \}$, in this order.
%
%\item There is no such $j_{\opi}$. Then $\opi = (\{1\}, \dots , \{n\})$ and we define $\zeta(\opi )= \opi $.
%
%\end{enumerate}
%
%It is a direct observation that $\zeta $ is an involution.
%In fact, the only fixed point is $\opi = (\{1\}, \dots , \{n\})$, and for any other set composition $\opi$, whenever the index $j_{\opi}$ in $\opi $ behaves as described in case 1, then the index $j_{\zeta(\opi)}$ in $\zeta ( \opi ) $ behaves as described in case 2, in which case we have $l(\opi) = 1+\ell(\zeta (\opi))$ and we can easily see that $\zeta(\zeta(\opi )) = \opi$.
%%We also have the converse statement. 
%Thus, we have that 
%\[x^n \sum_{\opi \in \oPi_n} (-1)^{\ell(\opi)} = x^n (-1)^{\ell(\{1\}, \dots, \{n\} )}= (-x)^n,\] 
%as desired.
%\end{proof}
%
%
%In this way we see that a formula for the antipode depends simply on an understanding of the structure of the length of compositions of $[n]$.
%This is a general feature whenever we apply Takeuchi's formula.
%
%



\