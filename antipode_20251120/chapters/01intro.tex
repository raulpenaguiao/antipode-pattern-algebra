
\section{Introduction\label{sec:intro}}

A central problem in combinatorics is to extract combinatorial interpretations of algebraic invariants.
Hopf algebras provide a powerful framework for this, encoding combinatorial structures with algebraic operations.
However, computing antipodes in Hopf algebras---the algebraic analogue of inversion---remains notoriously difficult.
Takeuchi's celebrated formula (see \cite{Takeuchi1971}, Lemma 14) offered the first general approach, applying to any filtered Hopf algebra.
Yet this formula is computationally unwieldy: it produces intermediate cancellations and requires grouping terms to simplify the final result.

In this paper, we develop economical antipode formulas for \textbf{pattern Hopf algebras}---a rich family of Hopf algebras constructed from combinatorial structures via species with restrictions.
These algebras generalize classical objects like the permutation pattern Hopf algebra, and connect combinatorics directly to algebra.
Our main contribution is a systematic method to derive \textbf{cancellation-free and grouping-free} antipode formulas, which not only simplify computation but reveal hidden combinatorial structure.

\

An antipode formula is \textbf{cancellation-free} if every term in its expansion contributes fully to the final answer, with no intermediate cancellations.
Intuitively, if you sum contributions from many combinatorial objects, each with a $\pm 1$ sign, you want every term to ``survive'' in the final sum---no term should be negated by another, wasting computational effort.
Similarly, a formula is \textbf{grouping-free} if combinatorial objects are counted independently: you never need to aggregate or combine multiple objects to simplify the computation.
These properties are not just aesthetically pleasing; they reveal that the formula directly reflects the underlying combinatorial structure without algebraic noise.

Economical formulas for antipodes in Hopf algebras in combinatorics have played an important role in extracting old and new combinatorial equations, see \cite{Schmitt1993, humpert2012incidence, BS2017, aguiar2017hopf, xu2022cancellation}.
In particular, we argue Humpert and Martin stand out, as they were able to explain, in \cite{humpert2012incidence}, an elusive \textit{reciprocity relation} on graphs first presented by Stanley in \cite{stanley1975combinatorial}, where the number of acyclic orientations plays a role.

\

A fundamental insight in enumerative combinatorics is that algebraic morphisms often correspond to important combinatorial invariants.
For instance, the chromatic polynomial $\chi_G$ of a graph counts proper vertex colorings; evaluating it at $x=-1$ yields the number of acyclic orientations---a nontrivial result discovered by Stanley.
Such evaluations at negative integers are called \textbf{reciprocity relations}, and they often reveal surprising combinatorial structure.

More generally, a polynomial $\chi$ that is a Hopf algebra morphism commutes with the antipode: $\chi(-x) = \chi(S(h))$.
This means that if we have a combinatorial interpretation of the antipode $S(h)$, we automatically obtain a combinatorial interpretation of $\chi$ evaluated at negative integers.
Hence, developing cancellation-free antipode formulas directly yields new reciprocity results.
This is the engine driving our application in permutation patterns: the formula we develop reveals a new polynomial invariant and its reciprocity interpretation.

\

%On a graph $G$, we define its chromatic function by counting, for each $n\geq 1$, the number of \textit{stable colourings} of its vertices.
%That is, functions $f:V(G) \to [n]$ such that each edge is not monochromatic.
%This function, $\chi_G$, turns out to be a polynomial, and its degree is the number of vertices of $G$.
%It makes sense to explore the specialisation $\chi_G(-1)$, and Stanley proved that this counts the \textit{acyclic orientations} of $G$.
%The observation of Humpert and Martin is that $\chi$ is in fact a Hopf algebra morphism from the \textbf{incidence Hopf algebra on graphs} to the polynomial Hopf algebra, so it commutes with the antipodes of each Hopf algebra.
%The antipode of the polynomial Hopf algebra is $S(x) = -x$, so combinatorial interpretations of the chromatic polynomial on negative numbers can be elicited, via
%$$\chi_G(-x) = \chi_{S(G)}(x)\, .$$

%A cancellation-free formula for the antipode $S(G)$ on the incidence Hopf algebra on graphs was also presented by Humpert and Martin, where the number of acyclic orientations plays a role and explains the previous result from Stanley.
%This antipode formula was also used to obtain new relations involving the Tutte polynomial.


\

The \textit{sign-reversing involution method} (introduced by Sagan and Benedetti in \cite{BS2017}) offers a systematic approach: design a combinatorial involution that pairs most terms with opposite signs, causing them to cancel; the surviving terms yield a cancellation-free formula.
This elegant technique has succeeded for several Hopf algebras: the shuffle Hopf algebra, the incidence Hopf algebra on graphs, and quasisymmetric functions (see \cite{BS2017}).
However, the method is far from automatic.
It requires deep understanding of the specific algebra's structure, and constructing the involution is often highly nontrivial.

Notably, computing the antipode for permutation patterns has resisted these methods.
While Foissy's bialgebra of cointeraction approach yielded results \cite{Foissy}, and Xu achieved partial results for the Malvenuto–Reutenauer algebra \cite{xu2022cancellation}, only restricted families of permutations were handled.
The permutation pattern algebra remains largely out of reach---until now.
Our main contribution is the first cancellation-free and grouping-free antipode formula for permutation patterns, achieved by carefully adapting the sign-reversing involution method to this previously intractable family.

\

Another method for finding cancellation-free formulas arose in \cite{aguiar2017hopf}.
There, a cancellation-free antipode formula for a Hopf structure on generalized permutahedra was found. 
Because several interesting combinatorial structures can be embedded in the Hopf algebra of the generalized permutahedra, this also yields a cancellation-free formula for these Hopf subalgebras.
Specific examples are graphs, matroids, posets and set partitions.
Yet another method was found in \cite{Foissy}, where the notion of \emph{bialgebra of cointeraction} was used to obtain antipode formulas for the permutation pattern Hopf algebra.

\

Permutation patterns---the study of which smaller patterns appear within larger permutations---is a rich classical subject with roots in computer science.
Knuth's pioneering work on stack-sortable permutations \cite{Knuth} sparked decades of research, establishing permutation pattern avoidance as a central problem in enumerative combinatorics.
However, classical permutation pattern theory focuses on counting and avoiding patterns; algebraic structure remained elusive.

This motivated the second author to introduce the \textbf{permutation pattern Hopf algebra} \cite{Vargas}, elevating permutation patterns to the level of algebraic Hopf algebras.
This construction revealed new structure and opened the door to using tools from Hopf algebra theory---including antipode formulas---to understand permutation patterns.
Our paper extends this program: we develop a cancellation-free antipode formula specifically for the permutation pattern Hopf algebra, and show how it yields new reciprocity results.
Specifically, we consider finite sums of functions $\pat_{\pi}$ of the form 
$$ \binom{\tau}{\pi} \coloneqq \pat_{\pi}(\tau)\coloneqq  \#\{\text{ ways to fit $\pi$ in $\tau$ }\}\, ,$$
the central functions in the study of permutation patterns.
These functions span a vector space that is closed for pointwise product, see \eqref{eq:prodperm}, and a compatible coproduct, see \eqref{eq:coprodperm}.
The corresponding Hopf algebra is the \textbf{permutation pattern Hopf algebra} $\mathcal{A}(\mathtt{Per})$, and is shown to be free in \cite{Vargas}.
For a definition of patterns in permutations, or of \say{ways to fit $\pi$ in $\sigma$}, see \cite{penaguiao2020algebraic}

\

The permutation pattern Hopf algebra construction was generalized to other combinatorial objects by the first author \cite{Penaguiao2020}, for instance graphs, marked permutations or set partitions.
This was done in such a way that any combinatorial object enriched with restriction functions, a structure that we call a \textbf{species with restrictions}, yields a \textbf{pattern Hopf algebra} (a construction presented in \cite{Penaguiao2020}, that we recover here in \cref{thm:conHopfalgebra}).
The algebraic structure of species with restrictions is a generalisation of the classical algebraic structure of combinatorial species from Joyal, see \cite{AM2010}.


\

We present a general antipode formula (\cref{thm:general_antipode}) for any pattern Hopf algebra, derived via the sign-reversing involution method.
While this general formula does not guarantee cancellation-free and grouping-free expressions, the method reveals deeper structure that enables specializations to specific algebras.
We exploit this to derive our main theorems: \textbf{cancellation-free and grouping-free antipode formulas for both packed word patterns} (\cref{thm:antipode_packed}) \textbf{and permutation patterns} (\cref{thm:antipode_perms}).
These concrete formulas are not merely theoretical---they directly yield new reciprocity results.

\

Onwards we introduce a new polynomial invariant in permutations, by considering the evaluation character in the permutation pattern Hopf algebra, see \cref{MOPDef}.
As far as we know, this is a new family of invariants in permutations.
We apply the cancellation-free formula to interpret this polynomial in negative values, leading to a reciprocity result via the antipode formula.
Specifically, we show that the value of this polynomial at $x = -1$ counts so called \textbf{interlacing quasi-shuffle signatures} of specific permutations---a restricted type of decomposition where pattern occurrences maintain a specific ordering, see definition below.

\

A key question in the broader program is: which combinatorial objects can be faithfully represented within the species with restrictions framework?
Some objects are naturally suited---graphs and set partitions have intrinsic labelings that match the framework directly.
However, many unlabeled objects resist naive encoding.
Permutations, for instance, required an ingenious idea: interpreting them as double orders to capture their inherent structure.

Parking functions present a similar challenge.
These objects---sequences encoding how cars park along a road by occupying the first available spot---are rich combinatorial structures with deep connections to Dyck paths and the Catalan numbers.
Yet their lack of intrinsic labeling makes representing them as species with restrictions nontrivial.
We resolve this by exploiting a beautiful bijection between labeled Dyck paths and parking functions \cite{Loehr,BGLPV2021}.
This yields a novel species with restrictions on parking functions, enabling us to study parking function patterns within our framework.
As a bonus, our pattern definition recovers recent results on parking function avoidance \cite{adeniran2022pattern}.

\

\subsubsection*{Structure of This Paper}

We organize our contributions as follows.
The remainder of this introduction develops the permutation pattern Hopf algebra and states our main theorems formally (\cref{thm:antipode_packed, thm:antipode_perms}), emphasizing the combinatorial significance of interlacing quasi-shuffle signatures.

\cref{sec:antipode_computing} reviews Takeuchi's general formula and illustrates the sign-reversing involution method through examples, providing context for our specialized approach.

\cref{sec:species} and \cref{sec:pattern_algebra_contruction} build the foundational machinery: combinatorial species and the general pattern Hopf algebra construction.
Readers familiar with this material may skip ahead.

\cref{sec:species_restrictions} introduces species with restrictions, developing permutation patterns, packed words, and our novel encoding of parking functions.

\cref{sec:formula_general} and \cref{sec:formula_pp} present our main technical contributions: the general antipode formula and its specialization to cancellation-free formulas for packed words and permutations.

Finally, \cref{sec:reciprocity} demonstrates the power of our formulas by introducing the Multiple Occurrences Polynomial and deriving new reciprocity results that interpret this polynomial at negative integers.



\subsubsection*{Roadmap of Results}

Before diving into technical details, let us sketch the key ideas:
\begin{enumerate}
\item We derive a general antipode formula for any pattern Hopf algebra using the sign-reversing involution method, though it lacks the cancellation-free property in general.
\item We specialize this to permutation patterns and carefully construct a combinatorial involution that yields cancellation-free and grouping-free formulas.
\item We introduce a new polynomial invariant on permutations---the \textbf{Multiple Occurrences Polynomial} (MOP)---and apply our formula to derive reciprocity results interpreting this polynomial at negative integers.
\end{enumerate}

The power of this program is that the algebra (Hopf structure) and combinatorics (cancellation-free formulas, interlacing quasi-shuffle signatures) reinforce each other, creating a coherent and elegant narrative.

\subsection{The permutation pattern Hopf algebra and the main result}

In this section we introduce the Hopf algebra structure on permutation patterns, formalize the notion of cancellation-free and grouping-free formulas, and present the main result of this paper: a cancellation-free and grouping-free antipode formula for the permutation pattern Hopf algebra.

\

Let $\pat_{\pi}$ be a function on permutations, so that $\pat_{\pi}(\sigma)$ counts the number of restrictions of the permutation $\sigma$ that fit the pattern $\pi$.
In this way, the collection of permutation pattern functions $\{\pat_{\pi}\}$ is linearly independent, so it is a basis of a vector space $\mathcal A (\mathtt{Per})$.
Further, in \cite{Vargas}, it was shown that the pointwise product of two such functions can be expressed as a sum of other permutation pattern functions:
\begin{equation}\label{eq:prodperm}
\pat_{\pi_1} \pat_{\pi_2} = \sum_{\sigma} \binom{\sigma}{\pi_1, \pi_2} \pat_{\sigma} \, ,
\end{equation}
where the sum runs over all permutations $\sigma$ of any size.
If $|\sigma| > |\pi_1| + |\pi_2|$ we have $\binom{\sigma}{\pi_1, \pi_2} = 0$, in fact, the coefficients $\binom{\sigma}{\pi_1, \pi_2}$ that arise in this product formula count the so called \textbf{quasi-shuffle signatures}, or \textbf{QSS}, of $\sigma$ from $\pi_1, \pi_2$.
Let us make this concept precise:

\begin{defin}[QSS on permutations]
A quasi-shuffle signature, or QSS, of $\sigma$ from $\pi_1, \dots, \pi_n$ is a tuple $\III = (I_1, \dots, I_n)$ of sets on the ground set of the permutation $\sigma$, that cover this ground set in such a way that each $I_i$ is a pattern of $\pi_i$ in $\sigma$ --- that is the restricted permutations $\sigma|_{I_i}$ are the permutation $\pi_i$ for any $i$. Unlike standard shuffles, these sets can overlap and must jointly cover the entire ground set.

The sets $I_1, \ldots, I_n$ are not necessarily disjoint but their union must equal the ground set $I$ of $\sigma$.
We let $\binom{\sigma}{\pi_1, \dots, \pi_n}$ denote the number of such QSS.
It was shown in \cite{Penaguiao2020} that $\binom{\sigma}{\pi_1, \dots, \pi_n}$ is precisely the coefficient that arises in the iterated product of $n$ elements.
This fact justifies the notation choice in \eqref{eq:prodperm}.

\end{defin}

\

This algebra $\mathcal A(\mathtt{Per})$ can be endowed with a Hopf algebra structure with the help of the diagonal sum of permutations, $\oplus$, also called shifted concatenation of permutations.
If one lets $\pi = \pi_1 \oplus \dots \oplus \pi_n$ be the decomposition of $\pi$ into $\oplus$-indecomposable permutations under the $\oplus$ product, we define the shifted deconcatenation coproduct

\begin{equation}\label{eq:coprodperm}
\Delta \pat_{\pi} = \sum_{k=0}^n \pat_{\pi_1\oplus \dots \oplus \pi_k} \otimes \pat_{\pi_{k+1}\oplus \dots \oplus \pi_n} \in \mathcal A (\mathtt{Per}) \otimes \mathcal A (\mathtt{Per})\, .
\end{equation}

To present a cancellation-free antipode formula we need to introduce a restricted subclass of QSS.
For sets $A, B$ of integers, we write $A <B$ if $a<b $ for any $a\in A, b \in B$.

\begin{defin}[Interlacing QSS on permutations]
A QSS $\III = (I_1, \dots, I_n)$ is said to be \textbf{non-interlacing} if there exists an index $i \in \{1, \dots, n-1\}$ where both orderings align: $I_i < I_{i+1}$ and $\sigma(I_i) < \sigma(I_{i+1})$. Intuitively, this means the sets appear in the same relative order in both the positions and values.

Otherwise, we say that the QSS is \textbf{interlacing}. In this case, the sets do not maintain a consistent ordering across both positions and values—they are "mixed" or "crossed" in some way. This property is crucial because it allows cancellation-free antipode formulas to work: the constraint of interlacing reduces the complexity of the sum.

\

Importantly, reordering an interlacing QSS does not in general give an interlacing QSS. This is in contrast with a simple QSS, where a reordering does result in a QSS. This distinction is what makes interlacing QSS a powerful tool for deriving efficient antipode formulas.

We let $\bigl[\!\begin{smallmatrix} \sigma \\ \pi_1, \dots, \pi_n \end{smallmatrix}\!\bigr]$ denote the number of interlacing QSS.
\end{defin}


\begin{thm}[Antipode formula for permutation pattern Hopf algebra]\label{thm:antipode_perms_intro}
Let $\pi$ be a permutation that factors $\pi = \pi_1\oplus \dots \oplus \pi_n$ into $\oplus$-indecomposable permutations.
Then, we have the following formula for the antipode of $\pat_{\pi}$:

$$S(\pat_{\pi}) = (-1)^n \sum_{\sigma} \bigl[\!\begin{smallmatrix} \sigma \\ \pi_1, \dots, \pi_n \end{smallmatrix}\!\bigr] \pat_{\sigma}\, ,$$
where the sum runs over all permutations $\sigma$, and the coefficients count the number of \textbf{interlacing QSS} of $\sigma$ from $\pi_1, \dots, \pi_n$.
\end{thm}

Note that for $|\sigma| > \sum_i |\pi_i|$, the coefficient $\bigl[\!\begin{smallmatrix} \sigma \\ \pi_1, \dots, \pi_n \end{smallmatrix}\!\bigr]$ vanishes, so this sum is finite.

\

In the following we present some examples that help explain QSS and interlacing QSS.
%We also present an antipode formula for the packed words pattern Hopf algebra, with a similar characterisation.

\


\begin{figure}[h]
    \centering
    \includegraphics{../images/interlacing_25314_square.pdf}
    \caption{\textbf{Left:} the permutation 2314, along with a labelling of two of its QSS from $1, 21, 1, 1$. In orange the two sets that do not interlace. \textbf{Right:} the permutation 123, along with its two interlacing QSS from $1, 12$.\label{fig:interlacingQSSsmpl}}
\end{figure}


\begin{smpl}[Interlacing QSS]
The permutation $2314$ has several QSS from $1, 21, 1, 1$, for instance $(4, 13, 2, 1)$ and $(1, 13, 4, 2)$, but from these two, only the first is interlacing.
In \cref{fig:interlacingQSSsmpl}, one can see these two QSS.
Further computations can show that $\binom{2314}{1, 21, 1, 1} = 36$ and $\bigl[\!\begin{smallmatrix} 2314 \\ 1, 21, 1, 1 \end{smallmatrix}\!\bigr] = 8$.

\

One can observe that there are three QSS of $123$ from $1$, $12$, but one of them is non-interlacing (the QSS $(1,23)$), so
$\bigl[\!\begin{smallmatrix} 123 \\ 1, 21 \end{smallmatrix}\!\bigr] = 2$.
In \cref{fig:interlacingQSSsmpl}, one can see these two interlacing QSS.
\end{smpl}

