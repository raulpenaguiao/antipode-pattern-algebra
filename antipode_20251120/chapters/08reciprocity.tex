
\section{Reciprocity results\label{sec:reciprocity}}

In this section, let $\mathtt{R}$ be a connected associative species with restrictions.
On this setting, we define a new polynomial invariant associated to any object in $\mathtt{R}$, the \textbf{multiple occurrences polynomial}, or MOP for short.
The MOP construction in this section is rather general and corresponds to an infinite family of polynomial maps $\mathcal{A}(\mathtt{R}) \to \mathbb{K}[x]$, one for each choice of object $\mathrm{z} \in \mathtt{R}$.

After the definition of the MOP, we present a motivation for this construction.
Our main theorem is that this construction is in fact a Hopf algebra morphism, and we leave the proof of this result to the end of the section.
We also display an application of the cancellation free formula above, \cref{thm:antipode_perms} to produce a reciprocity result, following the steps in \cite{humpert2012incidence}, on the permutation pattern Hopf algebra.


\begin{defin}[Multiple occurrence polynomial on a species with restrictions $\mathtt{R}$]\label{MOPDef}
Fix an object $\mathrm{z}$ of the connected associative species with restrictions $\mathtt{R}$.
We construct a map $\chi^{\mathrm{Z}}: \mathcal{A}(\mathtt{R}) \to \mathbb{K}[x]$.
For that, fix a positive integer $x$ and an object $\mathrm{y}$ in $\mathtt{R}$, $\chi^{\mathrm{Z}}$ has
$$\chi^{\mathrm{z}}(\pat_{\mathrm{y}})(x) \coloneqq \pat_{\mathrm{y}}(\mathrm{z}^{\star x})\, , $$
where $\mathrm{z}^{\star x}$ denotes the $\oplus$-product of $\mathrm{z}$ with itself $x$ times.
\end{defin}

That this indeed defines a polynomial, and that this is a Hopf algebra morphism, is the content of the main theorem below \cref{thm:polynomiality}.
Let us now see what values this invariant takes.
For notation simplicity we write $\chi^{\mathrm{x}}_{\mathtt{y}} \coloneqq \chi^{\mathrm{x}}(\pat_{\mathtt{y}})$.

\begin{smpl}
We compute this invariant on particular objects for $\mathtt{R} = \mathtt{Per}$, defined above in \cref{sec:speciespermutation}.
Take $\mathtt{z} = 1$, the unique permutation of size one.
It is easy to see that $\chi^{\mathrm{z}}_{\mathrm{y}}(x) = 0$ whenever $\mathrm{y}$ is not an increasing permutation, and $\chi^{\mathrm{z}}(\pat_{\mathrm{z}^{\star k}})(x) = \binom{x}{k}$, a polynomial of degree $k$.
In fact, we can observe something much more general: if $\pi, \tau$ have disjoint sets of $\oplus$-indecomposable factors, then $\chi^{\tau}_{\pi} = 0$

Let's now take $\mathtt{z} = 21$, the unique indecomposable permutation of size two.
In this case, $\ker \chi^{\mathrm{z}} = \spn \{ \pat_{\pi} | \text{$\oplus$-factorization of $\pi$ contains a permutation $\tau$ with $|\tau| > 2$} \}$ from the observation above.
If $\pi $ decomposes into $k_1$ copies of $\tau_1 = 1$ and $k_{21}$ copies of $\tau_2 = 21$, then $\chi^{\mathrm{z}}_{\pi}(x) = 2^{k_1} \binom{x}{k_1 + k_{21}}$.
\end{smpl}

The MOP encapsulates the complexity of the species with restrictions quite well.
One can easily observe that, for the species with restrictions $\mathtt{R} = \mathtt{Gr}$ on graphs, a general formula for $\chi^{\mathrm{z}}_{\mathrm{y}}(x)$ can be found for indecomposable $\mathrm{y}$, depending only on $\pat_{\mathrm{y}}(\mathrm{z})$.
This is not so simple for the permutation case.
On $\mathtt{R} = \mathtt{MPer}$, the species with restrictions of marked permutations, the story is even more complicated as one does not have a unique factorization into indecomposables, making the description of all possible factorizations more complex.
In this way, we ask the following question.

\begin{conj}
    For any marked permutations $\mathtt{y}, \mathtt{z}$, the polynomial $\chi^{\mathrm{z}}_{\mathrm{y}}$ is a binomial multiplied by some constant, that is it maps $x \mapsto a \binom{x}{b}$ for $a, b$ independent of $x$.
\end{conj}

\begin{obs}
    Fixing $x = 1$ we get a multiplicative map $\mathcal{A}(\mathtt{R}) \to \mathbb{K}$.
    This map is in fact a \textbf{character}, that is, a unital algebra homomorphism to the field $\mathbb{K}$.
    
    In \cite{aguiar2006combinatorial}, a Hopf algebra morphism was constructed from any connected graded Hopf algebra to $QSym$ using only a character.
    The astute reader may note that there is a classical \textit{specialisation} map $QSym \to \mathbb{K}[x]$, so one should wonder if this $\chi^{\mathrm{z}}$ arises from lifting a character $\mathcal{A}(\mathtt{R}) \to \mathbb{K}$ to a map $QSym \to \mathbb{K}[x]$, followed by composing with said specialisation.
    
    Indeed, the evaluation map $\mathcal{A}(\mathtt{R}) \to \mathbb{K}$ defined by $\pat_{\mathrm{y}} \mapsto \pat_{\mathrm{y}}(\mathrm{z})$ is this character.
    We leave the proof of this fact to the interested reader, because it is mostly tedious unwravelling of definitions that closely relates to the proof of \cref{lm:polynomial} below.
\end{obs}

We now apply \cref{thm:antipode_perms} to extract a reciprocity result for permutations.

\begin{smpl}
    Let us get back to the case $\mathrm{z} = 21$.
    Because the antipode on polynomials simply flips the sign of the variable, we have
$$\chi^{21}_{\pi}(-x) = S(\chi^{21}(\pat_{\pi})(x)) =\chi^{21}(S(\pat_{\pi}))(x) $$

Setting $x = 1$ and using \cref{thm:antipode_perms}, we get
$$\chi^{21}_{\pi}(-1) = S(\pat_{\pi})(21) = (-1)^n \sum_{\sigma} 
    \begin{bmatrix}
    \sigma \\ \pi_1, \dots, \pi_n
    \end{bmatrix}\pat_{\sigma}(21) $$

    This allows us to quickly compute simple cases for $\pi = 12$, for instance $\chi^{21}_{\pi}(-1) = 1 \times \pat_{21}(21) = 1$.
\end{smpl}



\subsection{Proof of polynomiality and Hopf morphism}

We present and prove the main theorem of this section.

\begin{thm}[MOPs in pattern algebras]\label{thm:polynomiality}
Let $\mathtt{R}$ be a connected associative species with restrictions and $\mathrm{x}$ an object.
The map $\chi^{\mathrm{x}}$ takes patterns to polynomials and is a Hopf algebra morphism $\mathcal A(\mathtt{R}) \to \mathbb{K}[x]$.
\end{thm}

We recall that for each object $\yy$ in a connected associative species with restrictions, any factorization into $\ast$-indecomposable elements has the same multiset of factors and, therefore, the same size, which we call $\ell(\yy)$.

\begin{thm}\label{lm:polynomial}
The invariant $\chi^{\xx}_{\yy}(n)$ is a polynomial in $n$.
The degree of this polynomial is at most $\ell(\yy)$, the number of $\ast$-indecomposable factors of $\yy$ (with repetition).
\end{thm}


\begin{proof}
We act by induction on $|\yy|$.
If $|\yy| = 0$, recall that $\mathtt{R}$ is connected, so $\chi^{\xx}_{\yy}(n) = 1$ for all $n$, which is a polynomial of degree zero.

For the induction hypothesis, observe that
$$\Delta \pat_{\yy} = 1\otimes \pat_{\yy} + \pat_{\yy} \otimes 1 + \sum_{\substack{a \ast b = y\\\ell(a), \ell(b) < \ell(\yy)}} \pat_a \otimes \pat_b \, ,$$
thus we have for $n > 1$. The key insight is that differences of consecutive values are polynomials of lower degree:

\begin{equation*}
    \begin{split}
        \pat_{\yy}(\xx^{\ast n}) &= \pat_{\yy}(\xx^{\ast n-1}) + \pat_{\yy}(\xx) + \sum_{\substack{a \ast b = y\\|a|, |b| < |\yy|}} \pat_a (\xx) \otimes \pat_b (\xx^{\ast n-1}) \\
         \pat_{\yy}(\xx^{\ast n}) &- \pat_{\yy}(\xx^{\ast n-1}) = \pat_{\yy}(\xx) + \sum_{\substack{a \ast b = y\\|a|, |b| < |\yy|}} \pat_a (\xx) \chi^{\xx}_b ( n-1)
    \end{split}
\end{equation*}

The right hand side is, by induction hypothesis, a polynomial of degree at most $\ell(\yy) -1$.
The left hand side is $\chi^{\xx}_{\yy}(n) - \chi^{\xx}_{\yy}(n-1)$, which shows that $\chi^{\xx}_{\yy}(n)$ has degree at most $\ell(\yy)$.
\end{proof}


\begin{lm}\label{lm:HA_morphism}
The map $\pat_{\yy} \mapsto \chi^{\xx}_{\yy}$ is a Hopf algebra morphism.
\end{lm}

\begin{proof}
   Observe the product is preserved precisely because the product on $\mathcal{A}(\mathtt{R})$ is the pointwise product: that is, if $\pat_{\yy_1} \pat_{\yy_2} = \sum_{\zz} \binom{\zz}{\yy_1, \yy_2}_{\mathtt{R}} \pat_{\zz}$, then passing in the argument $x^{\ast n}$ of both sides yields a true equality $\chi^{\xx}_{\yy_1}(n) \chi^{\xx}_{\yy_2}(n) = \sum_{\zz} \binom{\zz}{\yy_1, \yy_2}_{\mathtt{R}} \chi^{\xx}_{\zz}(n)  $.

    For the coproduct, we want to show that $\Delta \chi^{\xx}_{\yy} = \chi^{\xx} \otimes \chi^{\xx}(\Delta \pat_{\yy})$.
    It is enough to show that $\Delta \chi^{\xx}_{\yy} (m \otimes n) = \chi^{\xx} \otimes \chi^{\xx}(\Delta \pat_{\yy})(m\otimes n) $ for any non-negative integers $m, n$.

    Write $\chi^{\xx}_{\yy} = \sum_{k\geq 0 } a_k x^k$.
    On the left hand side, we get $\Delta \chi^{\xx}_{\yy} (m \otimes n) = \sum_{k \geq 0} a_k \sum_{j=0}^k \binom{k}{j} x^j\otimes x^{k-j} (m \otimes n) = \sum_{k \geq 0} a_k (m+n)^k = \chi^{\xx}_{\yy}(m+n) $.

    On the right hand side we have 
    
    
    \begin{equation*}
        \begin{split}
        (\chi^{\xx} \otimes \chi^{\xx})\Delta \pat_{\yy}(m\otimes n) =& \sum_{a \ast b = \yy} \chi^{\xx}_a(m) \otimes \chi^{\xx}_a(n)\\
        =& \sum_{a \ast b = \yy} \pat_a(x^{\ast m})\pat_b(x^{\ast n})\\
        =& \sum_{a \ast b = \yy} \pat_a\otimes \pat_b ( x^{\ast m} \otimes x^{\ast n})\\
        =& \Delta \pat_{\yy}(x^{\ast n} \ast x^{\ast b}) = \Delta \pat_{\yy}(x^{\ast m+n}) = \chi^{\xx}_{\yy}(n+m)\, .
        \end{split}
    \end{equation*}
    This is exactly what we obtained earlier, so this map preserves coproducts.
\end{proof}


\begin{proof}[Proof of \cref{thm:polynomiality}]
This is a consequence of \cref{lm:polynomial} and \cref{lm:HA_morphism}.
\end{proof}

