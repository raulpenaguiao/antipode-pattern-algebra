
\section{Examples of species with restrictions \label{sec:species_restrictions}}

\subsection{Species on total orders.\label{sec:specieslinearorders}}

The first relevant species with restrictions to define is the species of total orders $\mathtt{L}$.
This has $\mathtt{L}[I] = \{\text{ total orders on $I$ }\}$, a set with $|I|!$ total orders.
The restriction of an order $\leq$ of $\mathtt{L}[I]$ to a subset $J$ is the induced order, which is a total order.
This defines a species with restrictions.

\

We also define an associative product on $\mathtt{L}$.
If $\mathsf{p}$ is a total order on $I$ and $\mathsf{r}$ is a total order on $J$, and if $I, J$ are disjoint, then we can define a total order $\mathsf{p} \ast\mathsf{r}$ a total order on $I \sqcup J$ as:

\begin{itemize}
\item if $a, b \in I$ such that $a \, \mathsf{p} \, b$, then $a \, (\mathsf{p} \ast \mathsf{r}) \, b$.

\item if $a, b \in J$ such that $a \, \mathsf{r} \, b$, then $a \, (\mathsf{p} \ast \mathsf{r}) \, b$.

\item if $a \in I$ and $b \in J$, then $a \, (\mathsf{p} \ast \mathsf{r}) \, b$.
\end{itemize}
This definition is independent of the choice of labeling and respects restriction to subsets. In particular, the product $\ast$ on orders depends only on the order structures, not on the specific labels chosen, and it respects restrictions to subsets, so this builds a connected associative species with restrictions. 

\

If $\mathsf{p} \in \mathtt{L}[I]$, we write $\mathbb{X}(\mathsf{p}) = I$.
The following fact is an immediate observation.

\begin{prop}\label{prop:linorderideals}
If $\mathbb{X}(\mathsf{p}) = I$, then $I$ is an order ideal in $\mathsf{p} \ast \mathsf{r}$.
That is, if $a\in I$ and $b (\mathsf{p} \ast \mathsf{r}) a$, then $b\in I$.
\end{prop}

\

\subsection{Species on permutations\label{sec:speciespermutation}}

To fit the framework of species with restrictions, we use a rather unusual definition of permutations introduced in \cite{albert2020two}.

This definition is motivated by the fundamental need to represent permutations as a combinatorial species with restrictions—that is, we need a definition where we can systematically understand how smaller permutation patterns fit into larger ones via restriction maps. Rather than viewing a permutation as a bijection (which doesn't naturally support a restriction structure because a bijection's "pattern" is not obviously defined as a sub-bijection), we represent it as a pair of total orders.

Specifically, a permutation on a set $I$ is seen as a pair of total orders $(\leq_P, \leq_V)$ on $I$, called the "position" and "value" orders respectively. Write $\mathbb{X}(\pi) = I$.

This relates to the usual notion of a permutation as a bijection in the following way: if we order the elements of $I = \{a_1 \leq_P \dots \leq_P a_k \} = \{ b_1 \leq_V \dots \leq_V b_k \}$, then the natural pairing $a_i \leftrightarrow b_i$ defines a bijection via the mapping $a_i \mapsto b_i$. Conversely, for any bijection $f$ on $I$, there exist multiple pairs of orders $(\leq_P, \leq_V)$ that correspond to the same bijection $f$; all of these pairs are isomorphic as permutations in the sense that they differ only by relabeling of the set $I$.

Crucially, the pair-of-orders representation allows restriction in a natural way: if $J \subseteq I$ is an injection, we can restrict both the position and value orders to $J$, yielding a permutation on $J$. This restriction structure is what enables permutations to form a species with restrictions.
The resulting species with restrictions structure is denoted by $\mathtt{Per}$.

\

It will be useful to represent permutations in $I$ as square diagrams labeled by $I$.
This is done in the following way: we place each element $a \in I$ at grid position $(i, j)$ where $i$ is the position of $a$ in the $\leq_P$ (position) order and $j$ is the position of $a$ in the $\leq_V$ (value) order. This visual representation directly encodes the bijection: the row index shows where an element comes from (position), and the column index shows where it goes (value).
We always start counting from the bottom left, following the Cartesian coordinates style.

For instance, the permutation $\pi = \{1<_P2<_P3 , 2<_V1<_V3\}$ in $\{1, 2, 3\}$ can be represented as 
\begin{equation}
\begin{array}{|c|c|c|}
	\hline & & 3 \\
    \hline 1 & &  \\
    \hline & 2 & \\
    \hline 
\end{array}\, \, \, .
\end{equation}

In this way, there are $(n!)^2 $ elements in $\mathtt{Per}[n]$.
Up to relabelling, we can represent a permutation as a diagram with one dot in each column and row.
Thus, $\mathtt{Per}[n]/_{\mathfrak S_n}$ (the set of isomorphism classes), has $n!$ isomorphism classes of permutations of size $n$, as expected.
\

%\label{defin:per}
%If we consider a permutation $\pi$ on a set $I$, that is, a pair $(\leq_P, \leq_V) $ of total orders in $I$, we write $\mathbb{X}(\pi) = I$.
%If $f:J \to I $ is an injective map, the preimage of each order $\leq_P, \leq_V$ is well defined and is also a total order in $J$.
%This defines the permutation $\mathtt{Per}[f](\pi )$.
%If $f$ is the inclusion map, this notion recovers the usual concept of a permutation pattern already present in the literature.

\

The $\ast $ operation on orders can be extended to a \textbf{diagonal sum} operation $\oplus $ on permutations, which places all elements of the first permutation before all elements of the second permutation in both the position and value orders.
This is usually referred as the shifted concatenation of permutations.
This endows $\mathtt{Per}$ with an  associative species with restrictions structure.



\subsection{Species on packed words}


For packed words, we will mimic the framework produced above for permutations.
First, recall that a linear partial order $\leq$ on a set $I$ is the pullback order of a surjective map $I \to [m]$, called the \textbf{rank function} of $\leq$, or $rk_{\leq}$.
Equivalently, it is an order $\leq$ where for any two distinct elements $a, b\in I$ such that $a \leq b$, there is a disjoint partition $I = A \uplus B$ such that all elements $x$ in $A$ and all elements $y$ in $B$ satisfy $x \leq y$.


We call the integer $m$ the rank of the order $\leq$.

\begin{figure}[h]
	\centering
	\includegraphics[scale=1]{../images/packedWordOrder.pdf}
	\caption{\textbf{Left:} The description of the function $rk_{\leq}$. \textbf{Right:} The Hasse diagram of the linear order inherited by $f$. \label{fig:packedWordOrder}}
\end{figure}

For instance, if $f = \{ a\mapsto 3, b\mapsto 1, c\mapsto 3, d\mapsto 2\}$ is a surjective map $\{a, b, c, d\}\to[3]$, the pullback order is $\{b < d < \{a, c \}\}$ and has rank three.
Its Hasse diagram is presented in \cref{fig:packedWordOrder}.
In this way, a packed word $\omega$ on $I$ is a pair of orders $(\leq_P, \leq_V)$ where $\leq_P$ is a total order on $I$, and $\leq_V$ is a linear partial order on $I$. This generalizes permutations by allowing ties (incomparabilities) in the value order, represented as ties in the vertical direction of the visual representation.
In particular, note that any permutation on $I$ can be seen as a packed word on $I$, as any total order is a partial linear order.
This relates to the usual notion of packed words (a word in $[m]$ where $m \leq |I|$ represents the rank) in the following way:
if we order the elements of $I = \{a_1 \leq_P \dots \leq_P a_k \}$, then the corresponding packed word is 
$$rk_{\leq_V}(a_1)rk_{\leq_V}(a_2) \dots rk_{\leq_V}(a_k) \, .$$

Conversely, for any packed word $\omega = p_1\dots p_k$, there are several pairs of orders $(\leq_P, \leq_V)$ that correspond to the word $\omega $, all of which are isomorphic.
Specifically, we can fix any total order $\leq_P$, and construct $\leq_V$ from $\omega $ and $\leq_P$: if the $i$-th entry of $\omega$ is $\omega(i)$ and the $i$-th element according to $\leq_P$ is $a_i$, then we set $rk_{\leq_V}(a_i) = \omega(i)$.

\

Consider for instance the packed words $\omega_1 = 13123$ and $\omega_2 = 32413$.
These correspond to packed words on $\{a, b, c, d, e\}$.
For the examples given $\omega_1$ corresponds to $(d \leq_P e \leq_P c \leq_P a \leq_P b, \{d, c\} \leq_V a \leq_V \{b, e\} )$ and $\omega_2$ corresponds to $(a \leq_P b \leq_P c \leq_P d \leq_P e, d \leq_V b \leq_V \{a, e\} \leq_V c )$.


If $I \hookrightarrow J$, by restricting the orders on $I$ to orders on $J$, we obtain a restriction to a packed word on $J$.
The resulting species with restrictions structure is denoted by $\mathtt{PWo}$.

\

It will be useful to represent packed words $\omega $ in $I$ as rectangle diagrams labeled by $I$.
This is done in the following way: let $1\leq m \leq |I|$ be the rank of $\omega$, we place the elements of $I$ in an $m \times |I|$ grid so that the elements are placed horizontally according to the $\leq_P$ order, and vertically according to the $\leq_V$ order.
For instance, the packed word $\omega_1 = 13123 = ( d < e < c < a < b, \{d, c\}) < \{a\} < \{b, e\}$ in $\{a, b, c, d, e\}$ can be represented as 
\begin{equation}
\begin{array}{|c|c|c|c|c|}
	\hline   & e &   &   & b \\
    \hline   &   &   & a &   \\
    \hline d &   & c &   &   \\
    \hline 
\end{array}\, \, \, .
\end{equation}




In this way, there are $(n!) \times \sum_{m = 1}^n \sFunc(I, [m])$ elements in $\mathtt{PWo}[n]$, where $\sFunc(I, [m])$ denotes the number of surjections (surjective functions) from $I$ to $[m]$.
Because each packed word $\omega $ has an isomorphism class of size $n!$, there are $ \sum_{m = 1}^n \sFunc(I, [m])$ many packed words, which is expected.

\

\label{defin:pwo}
If we consider a packed word $\omega = (\leq_P, \leq_V) $ on a set $I$, we write $\mathbb{X}(\omega) = I$.
If $f:J \to I $ is an injective map, the preimage of each order $\leq_P, \leq_V$ is well defined.
Furthermore, the preimage of $\leq_P$ is a total order on $J$, whereas the preimage of $\leq_V$ is a linear order on $J$.
Thus, this defines the packed word $\mathtt{PWo}[f](\omega )$.
The $\ast $ operation on orders can be extended to a \textbf{diagonal sum} operation $\oplus $ on packed words.
This is usually referred to the shifted concatenation of packed words.
This endows $\mathtt{PWo}$ with an  associative species with restrictions structure.

\

\subsection{Relation between parking functions and labelled Dyck paths}

Before we make the goal of this section explicit, let us first recall the definition of a parking function and of a Dyck path.

\begin{defin}[Parking function]
A parking function $\mathfrak{p} = a_1 \dots a_n$ is a sequence of integers in $[n]$ such that, after reordering $a^{(1)} \leq a^{(2)} \leq \dots \leq a^{(n)}$, we have $a^{(i)} \leq i$ for all $i$.
\end{defin}

Intuitively, parking functions model the following scenario: $n$ cars arrive at a street with $n$ parking spots labeled $1, \ldots, n$. Each car $i$ has a preferred spot $a_i$. Cars park greedily: car $i$ parks at its preferred spot if available, otherwise it parks at the next available spot to the right. A sequence is a parking function if and only if all cars can find a parking spot using this greedy algorithm.

Examples of parking functions are $12$, $131$ and $3114$.

\begin{defin}[Dyck path]
Given an $n\times n$ square grid, a \textbf{Dyck path} is an edge path from $(0,0)$ to $(n,n)$ staying weakly above the line $y = x$ (the main diagonal).
It is a classical result that Dyck paths are enumerated by Catalan numbers.
\end{defin}


To describe species on parking functions we need to use the construction of parking functions as labelled Dyck paths.
% (see for instance \cite{Loehr}).
%This notion of species with restrictions will recover a notion of patterns in parking functions studied in \cite{adeniran2022pattern}.
Specifically, if $I$ is a set of size $n$, we label a Dyck path $\mathcal D$ on an $n\times n$ square grid by a function $f$ that assigns unique values on the set $I$ to each of the \textbf{up} segments of the Dyck path. An 'up' segment is an edge from $(i, i)$ to $(i+1, i+1)$; a 'down' segment is an edge from $(i, i+1)$ to $(i+1, i)$.
If we enrich $I$ with a total order $\leq$ in such a way that in each sequence of \textbf{up} segments, the labels arise in \textbf{increasing} order, Bergeron et al.\ construct in \cite{BGLPV2021} a parking function $\mathfrak{p} = \mathfrak{p}(I, \mathcal D, f, \leq)$.
We recover here such construction, adapted to the language in this article, for convenience.

There are two steps in this construction.
The first is to realize the labelled Dyck path as a weak set composition $\opi$ of $I$ in $|I|$ parts. The second is to translate the weak set composition $\opi$ and the order $\leq $ in $I$ as a parking function.
We will use an example on $I = \{1, 2, 3, 4, 5\}$ to help us highlight the most important details of this construction, presented in \cref{fig:construction_parking}.
There, we have a Dyck path $\mathcal D$ and a corresponding assignment $f$ to each \textbf{up} segment.
We assume the usual integer order on $I$.

\

\textbf{In the first step} we group each of the labels that occur on the $i$-th vertical line in the $i$-th block of $\opi$.
This gives us a weak set composition of $I$ with exactly $|I|$ parts.
Notice that some parts may be empty sets, as it happens in the example given in \cref{fig:construction_parking}.

\textbf{In the second step} we read off the position of each of the labels of $I$, writing down in which part these occur. The resulting sequence is a parking function of size $|I|$.
In the example in \cref{fig:construction_parking}, for instance, we notice that there are no entries in the second block, so $2$ does not occur in the corresponding parking function.
Because there are three elements in the first block, namely $245$, in the parking function $\mathfrak p $ the character $1$ arises three times, on the second, fourth and fifth position.

\begin{figure}
    \centering
    \includegraphics{../images/correspondence_parking.pdf}
    \caption{This is an example of a Dyck path labelled by the set $I =\{1, 2, 3, 4, 5\}$. We enrich this set with the usual order on the integers.}
    \label{fig:construction_parking}
\end{figure}


\subsection{Species on parking functions}

The species of \textbf{parking functions}, $\mathtt{PF}$, arises by letting $\mathtt{PF}[I]$ be the collection of all parking functions $\mathfrak{p} = \mathfrak{p}(I, \mathcal D, f, \leq)$, that is Dyck paths $\mathcal D$ in an $|I|\times |I|$ grid, labelled by elements of $I$ along with an order $\leq$ of $I$.
We can give it a notion of species with restrictions: for each inclusion $\iota : I \hookrightarrow J $ and a labelled Dyck path $(J, \mathcal D, f, \leq)$ on $J$, the restriction $\mathtt{PF}[\iota](J, \mathcal D, f, \leq) = (I, \mathcal D|_I, f|_I, \leq|_I)$ has an intuitive meaning, except perhaps for
$\mathcal D|_I$, which we clarify in the following.
This is done via a notion of \textbf{tunnels} --- pairs of an up and down segment at the same level, that have no other segment at that level in between --- introduced by Deutsch and Elizalde in \cite{elizalde2003simple}.
Specifically, the corresponding Dyck path is defined to be the restricted Dyck path by taking the \textbf{tunnels} labelled by elements in $I$, relabelling sequences of \textbf{up} segments if necessary to preserve the increasing property.
One can see in \cref{fig:restriction_parking} how this works on the example given above, as well as the corresponding parking function.


\begin{figure}[h]
\centering
    \subfloat[\centering The parking function 31411 and its restriction]{{\includegraphics[height=4.8cm]{../images/restrictions_parking.pdf} }}%
    \qquad
    \subfloat[\centering The parking function 41511 and its restriction]{{\includegraphics[height=4.8cm]{../images/restrictions_parking_other.pdf}}}%
    \caption{\label{fig:restriction_parking}}%
\end{figure}

\

One defines a shifted concatenation $\oplus$ on parking functions, defined via concatenation of the underlying Dyck paths and orders (similar to the diagonal sum operation for permutations).
The following claim can be established by the same methods presented in \cite{Penaguiao2020}.

\begin{prop}[Species with restrictions on parking functions]
$\mathtt{PF}$ forms a species with restrictions structure. The proof follows the same approach as for permutations: the operations on Dyck paths and orders respect the restriction to subsets.
Furthermore, the shifted concatenation $\oplus $ endows $\mathtt{PF}$ with a monoid structure, and the resulting Hopf algebra $\mathcal A(\mathtt{PF})$ is free.
\end{prop}

For the last part, we observe that because $\mathtt{PF}$ is NCF (see below in \cref{defin:ncf}), we have from \cref{cor:freeNCF} that this algebra is free.

\begin{smpl}
The five smallest parking functions are $\emptyset, 1, 11, 12$ and $21$, where the empty sequence $\emptyset$ is the unique parking function of size 0.
The sixteen parking functions of size three are displayed in \cref{fig:PF3}, along with its corresponding labelled Dyck paths.

For any parking function $\mathfrak p$ of size three, we have $\pat_{\emptyset}(\mathfrak p) = 1$ and $\pat_1(\mathfrak p) = 3$.
The values of $\pat_{11}, \pat_{12}$ and $\pat_{21}$ in parking functions of size three are represented below in \cref{tab:PF3}.
There, one can also check the relation 
\[\pat_1^2 = \pat_1 + 2(\pat_{11} + \pat_{12} + \pat_{21}),\] 
predicted from \eqref{eq:prodrule}.
\end{smpl}
\begin{table}
\begin{tabular}{ c |  c c c c c c c c c c c c c c c c}
 $\pat$ & \small{111} & \small{112} & \small{121} & \small{211} & \small{113} &
\small{131} & \small{311} & \small{122} & \small{212} & \small{221} & \small{123} & \small{132} & \small{213} & \small{312} & \small{231} & \small{321}\\ 
 \hline 
 11 & 3 & 2 & 2 & 2 & 1 & 1 & 1 & 1 & 1 & 1 & 0 & 0 & 0 & 0 & 0 & 0 \\  
 12 & 0 & 1 & 0 & 0 & 2 & 1 & 0 & 2 & 1 & 0 & 3 & 2 & 2 & 1 & 1 & 0 \\  
 21 & 0 & 0 & 1 & 1 & 0 & 1 & 2 & 0 & 1 & 2 & 0 & 1 & 1 & 2 & 2 & 3 \end{tabular}
\caption{\label{tab:PF3}Pattern functions evaluated at parking functions of length three.}
\end{table}

\begin{figure}
\centering
\includegraphics[scale=1]{../images/parking_functions_3}
\caption{\label{fig:PF3}}
\end{figure}





\