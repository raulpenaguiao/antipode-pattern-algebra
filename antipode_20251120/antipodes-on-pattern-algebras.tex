\documentclass[12pt, reqno]{amsart}

\usepackage{graphicx}
\usepackage{amssymb}
\usepackage{amsthm}
\usepackage{listings}
\usepackage{lineno}
\usepackage[margin=3cm]{geometry}
\usepackage[all,cmtip, color,matrix,arrow]{xy}
\usepackage{subfig}

%\usepackage{natbib}
%\usepackage{graphicx}
\usepackage{amsmath}%To use \text 
%\usepackage{amssymb}
%\usepackage{amsthm}
\usepackage[utf8]{inputenc}
%\usepackage[english]{babel}
%\usepackage{biblatex}
\usepackage{hyperref}
\usepackage[capitalize]{cleveref}  
\crefname{thm}{Theorem}{Theorems}
\usepackage{bbold}
\usepackage[export]{adjustbox}
%\usepackage{tikz-cd}
%\usepackage{xr}
%\usetikzlibrary{babel}
\usepackage{todonotes}
\usepackage{bm}
\usepackage{wrapfig}
\usepackage{bbold}
\usepackage{float}
\usepackage{mathtools}
\usepackage{aliascnt}
\newaliascnt{eqfloat}{equation}
\newfloat{eqfloat}{h}{eqflts}
\floatname{eqfloat}{Equation}
\usepackage{dirtytalk}

\newcommand*{\ORGeqfloat}{}
\let\ORGeqfloat\eqfloat
\def\eqfloat{%
  \let\ORIGINALcaption\caption
  \def\caption{%
    \addtocounter{equation}{-1}%
    \ORIGINALcaption
  }%
  \ORGeqfloat
}
\newcommand{\raul}[1]{\todo[color=green!30,inline]{CH, #1}}

\newcommand{\yannic}[1]{\todo[color=violet!30,inline]{YV, #1}}

\theoremstyle{definition}
\newtheorem{thm}{Theorem}[section]
\newtheorem{prop}[thm]{Proposition}
\newtheorem{lm}[thm]{Lemma}	
\newtheorem{cor}[thm]{Corollary}
\newtheorem{obs}[thm]{Observation}
\newtheorem{defin}[thm]{Definition}
\newtheorem{smpl}[thm]{Example}
\newtheorem{quest}[thm]{Question}
\newtheorem{prob}[thm]{Problem}
\newtheorem{conj}[thm]{Conjecture}
\newtheorem{rem}[thm]{Remark}
\crefname{lm}{Lemma}{Lemmas}
\crefname{thm}{Theorem}{Theorems}
\crefname{prop}{Proposition}{Propositions}
\crefname{defin}{Definition}{Definitions}
\crefname{rem}{Remark}{Remarks}

\newcommand{\oPi}{\mathbf{C}}
\newcommand{\opi}{\vec{\boldsymbol{\pi}}}
\newcommand{\otau}{\vec{\boldsymbol{\tau}}}
\newcommand{\olambda}{\vec{\boldsymbol{\gamma}}}
\newcommand{\msum}{\sideset{^M}{}\sum}
\newcommand{\nestohedra}{\text{nestohedra}}
\newcommand{\gpHa}{\mathbf{GP}}
\newcommand{\gHa}{\mathbf{G}}
\newcommand{\citestan}{\cite[Proposition 3.2]{gebhard99}}
\newcommand{\parpi}{\boldsymbol{\pi}}
\newcommand{\partau}{\boldsymbol{\tau}}
\newcommand{\makepar}{\boldsymbol{\lambda}}
\newcommand{\uhsm}{\boldsymbol{\Psi}}
\newcommand{\uHsm}{\boldsymbol{\Psi}}
\newcommand{\sqbinom}{\genfrac{[}{]}{0pt}{}}


\newcommand{\bfpsigrf}{$\mathbf{\Psi}_{\mathbf{Q}}$}

\newcommand{\pper}{marked permutation }
\newcommand{\ppers}{marked permutations }
\newcommand{\pperp}{marked permutation.}
\newcommand{\ppersp}{marked permutations.}

\newcommand{\stirling}[2]{\genfrac{[}{]}{0pt}{}{#1}{#2}}%binomial coefficients
\newcommand{\III}{\vec{\mathbf{I}}}
\newcommand{\JJJ}{\vec{\mathbf{J}}}


\DeclareMathOperator{\w}{\mathcal{W}}
\DeclareMathOperator{\pos}{\mathrm{pos}}
\DeclareMathOperator{\s}{\mathrm{mset}}
\DeclareMathOperator{\ms}{\mathrm{mset}}
\DeclareMathOperator{\Alg}{\mathrm{Alg}}
\DeclareMathOperator{\im}{im}
\DeclareMathOperator{\id}{id}
\DeclareMathOperator{\comu}{comu}
\DeclareMathOperator{\rest}{\mathbf{res}}
\DeclareMathOperator{\Orth}{Orth}
\DeclareMathOperator{\cano}{cano}
\DeclareMathOperator{\Ppatp}{\mathbf{Ppat}^*}
\DeclareMathOperator{\dpt}{\mathbf{depth}}
\DeclareMathOperator{\pat}{\mathbf{pat}}
\DeclareMathOperator{\Pat}{\mathrm{Pat}}
\DeclareMathOperator{\Func}{\mathrm{Func}}
\DeclareMathOperator{\spn}{\mathrm{span}}
\DeclareMathOperator{\inc}{\mathrm{inc}}
\DeclareMathOperator{\Ch}{\mathrm{Ch}}
\DeclareMathOperator{\tvs}{\textvisiblespace}
\DeclareMathOperator{\sFunc}{\mathrm{SurFunc}}
\DeclareMathOperator{\End}{\mathrm{End}}
%\usepackage{lipsum}

\newcommand{\bfa}{\mathbf{a}}

%combinatorial concepts
\newcommand{\Sr}{\mathrm{S}} %symmetric group
\newcommand{\opp}[1]{\overline{#1}} %for opposite
\newcommand{\len}{l} % for length (degree) of a composition or partition
%\newcommand{\maxflat}{\hat{1}}
\newcommand{\maxflat}{\widehat{I}}
\newcommand{\minflat}{\{I\}}
\newcommand{\ifac}{
\begin{picture}(3,5)(0,0)
\put(0,0){\textup{!}}\put(1.5,4.8){\circle{3}}
\end{picture}
} %cyclic factorial
\newcommand{\acyc}[1]{#1 !} %acyclic orientations

%linear operators


%categories
\newcommand{\Fset}{\mathsf{Set^{\times}}}
\newcommand{\Veck}{\mathsf{Vec}_\Kb} 
\newcommand{\Vect}{\mathsf{Vect}}
\newcommand{\gVec}{\mathsf{gVec}}
\newcommand{\Set}{\mathsf{Set}}
\newcommand{\Ss}{\mathsf{Sp}} % set species
\newcommand{\Ssk}{\mathsf{Sp}_\Kb} %vector species
\newcommand{\Spr}{\mathsf{Spr}} % set species with restrictions

\newcommand{\Mo}[1]{\mathsf{Mon}(#1)} %monoids
\newcommand{\Co}[1]{\mathsf{Comon}(#1)} %comonoids
\newcommand{\Bo}[1]{\mathsf{Bimon}(#1)} %bimonoids
\newcommand{\Kb}{\mathbb{K}} 



%generic set species
\newcommand{\rP}{\mathrm{P}}
\newcommand{\rQ}{\mathrm{Q}}
\newcommand{\rH}{\mathrm{H}}
\newcommand{\rR}{\mathrm{R}}


%generic species with restrictions
\newcommand{\prP}{\mathtt{P}}
\newcommand{\prQ}{\mathtt{Q}}
\newcommand{\prH}{\mathtt{H}}
\newcommand{\prR}{\mathtt{R}}


%set species
\newcommand{\rE}{\mathrm{E}}
\newcommand{\rL}{\mathrm{L}}
%\newcommand{\rX}{\mathrm{X}}
\let\rPi=\Pi %flats
\let\rSig=\Sigma
\newcommand{\rSigh}{\widehat{\rSig}} %decompositions = weak compositions
\newcommand{\rG}{\mathrm{G}} %graphs
\newcommand{\rGP}{\mathrm{GP}} %generalized permutahedra
\newcommand{\SP}{\mathfrak{p}} %standard permutahedron

%generic species
\newcommand{\thh}{\mathbf{h}} 
\newcommand{\tg}{\mathbf{g}}
\newcommand{\tk}{\mathbf{k}}
\newcommand{\tp}{\mathbf{p}} 
\newcommand{\tq}{\mathbf{q}}
\newcommand{\tr}{\mathbf{r}}
\newcommand{\ta}{\mathbf{a}}
\newcommand{\tb}{\mathbf{b}} 
\newcommand{\tc}{\mathbf{c}}
\newcommand{\td}{\mathbf{d}}
\newcommand{\trr}{\mathbf{r}} 
\newcommand{\tm}{\mathbf{m}} %module
\newcommand{\brac}{\nu} %Lie bracket

%examples of species
\newcommand{\tone}{\mathbf{1}}
\newcommand{\wU}{\mathbf{U}}
\newcommand{\wX}{\mathbf{X}}
\newcommand{\wE}{\mathbf{E}} %exponential species
\newcommand{\wL}{\mathbf{L}} %linear orders
\newcommand{\wI}{\mathbf{I}}%used in macro \tLL
\newcommand{\tLL}{\wI\hspace*{-2pt}\wL} %pairs of chambers
\newcommand{\tLie}{\mathbf{Lie}} %Lie species
\newcommand{\tPi}{\mathbf{\Pi}} %flats
\newcommand{\tSig}{\mathbf{\Sigma}} %faces
\newcommand{\tSigh}{\mathbf{\widehat{\Sigma}}} % decompositions
\newcommand{\te}{\mathbf{e}} %elements
\newcommand{\tG}{\mathbf{G}} %graphs
\newcommand{\tcG}{\mathbf{cG}} %connected graphs
\newcommand{\tGP}{\mathbf{GP}} 

%some isomorphisms
\let\isoflat=\psi
\let\isolinear=\psi
\let\isograph=\varphi


%Fock functors
\newcommand{\contra}[1]{#1^{\vee}} 
\newcommand{\Kc}{\mathcal{K}}
\newcommand{\Kcb}{\overline{\Kc}}
\newcommand{\cKc}{\contra{\Kc}}
\newcommand{\cKcb}{\contra{\Kcb}}

%functors
\newcommand{\Tc}{\mathcal{T}}
\newcommand{\Tcq}{\Tc_q}
\newcommand{\Sc}{\mathcal{S}}
\newcommand{\Pc}{\mathcal{P}} %primitive element functor
\newcommand{\Qc}{\mathcal{Q}} %indecomposables
\newcommand{\Uc}{\mathcal{U}} %universal enveloping algebra

\newcommand{\sq}{{{\scriptstyle{\square}}}}

%objects
\newcommand{\xx}{\mathrm{x}}
\newcommand{\yy}{\mathrm{y}}
\newcommand{\zz}{\mathrm{z}}


%% natbib.sty is loaded by default. However, natbib options can be
%% provided with \biboptions{...} command. Following options are
%% valid:

%%   round  -  round parentheses are used (default)
%%   square -  square brackets are used   [option]
%%   curly  -  curly braces are used      {option}
%%   angle  -  angle brackets are used    <option>
%%   semicolon  -  multiple citations separated by semi-colon
%%   colon  - same as semicolon, an earlier confusion
%%   comma  -  separated by comma
%%   numbers-  selects numerical citations
%%   super  -  numerical citations as superscripts
%%   sort   -  sorts multiple citations according to order in ref. list
%%   sort&compress   -  like sort, but also compresses numerical citations
%%   compress - compresses without sorting
%%
%% \biboptions{comma,round}

% \biboptions{}


%\usepackage[backend=bibtex]{biblatex}
%\addbibresource{biblio.bib}

\usepackage{amsaddr}

\begin{document}

%% Title, authors and addresses
\title{Antipode formulas for pattern Hopf algebras} % Subtitle


%\author{Raul Penaguiao\footnote{\href{mailto:raulpenaguiao@sfsu.edu}{raulpenaguiao@sfsu.edu}}\footnote{Institute of Mathematics, University of Zurich, Winterthurerstrasse 190, Zurich, CH - 8057.}\footnote{{\bf Keywords:} marked permutations, presheaves, species, Hopf algebras, free algebras}\footnote{2010 AMS Mathematics Subject Classification 2010: 05E05, 16T05, 18D10}}

\author{Raul Penaguiao, Yannic Vargas}
\email{raul.penaguiao@mis.mpg.de}
\email{yannic.vargas@cunef.edu}
\address{HAEUSLER AG}
\address{CUNEF University}
\keywords{permutations, presheaves, species with restrictions, species, Hopf algebras, free algebras, antipode, cancellation-free, chromatic, reciprocity}
\subjclass[2010]{05E05, 16T05, 18D10}
\date{\today} % Date

\begin{abstract}
Computing antipodes in Hopf algebras is notoriously difficult. Takeuchi's classical formula applies broadly but is computationally unwieldy, riddled with unnecessary cancellations and groupings. This work develops systematically \textbf{cancellation-free and grouping-free} antipode formulas for a rich family of Hopf algebras constructed from combinatorial structures---the pattern Hopf algebras.

The key object is the \textbf{permutation pattern Hopf algebra}, where the product counts permutation patterns (via quasi-shuffle signatures) and the coproduct respects decomposition. This algebra fits a general framework via species with restrictions, unifying permutation patterns, graphs, parking functions, and beyond.

Our main contribution is a cancellation-free and grouping-free antipode formula for the permutation pattern Hopf algebra, derived via the sign-reversing involution method. Rather than merely theoretical, this formula has concrete power: it directly yields a new polynomial invariant on permutations---the Multiple Occurrences Polynomial---and reveals reciprocity interpretations when evaluating this polynomial at negative integers.

We also develop formulas for the packed word pattern Hopf algebra and introduce an original species with restrictions structure on parking functions, recovering recent results on parking function avoidance. Throughout, the interplay between algebraic structure and combinatorial form creates an elegant and powerful framework.
\end{abstract}


\maketitle

\tableofcontents

\section{Introduction\label{sec:intro}}

In his now celebrated  ``Lemma 14'', Takeuchi (see \cite{Takeuchi1971}, Lemma 14) obtained a quite general formula for the antipode of a Hopf algebra. 
This is an antipode formula for any filtered Hopf algebra, which can be applied in much generality and fits the framework of \textbf{pattern Hopf algebras}, introduced in \cite{penaguiao2020algebraic} and recovered below.
However, it has been observed that it is not the most economical formula, in the sense that it leaves some cancellations and groupings to be made.

\

Economical formulas for antipodes in Hopf algebras in combinatorics have played an important role in extracting old and new combinatorial equations, see \cite{Schmitt1993, humpert2012incidence, BS2017, aguiar2017hopf, xu2022cancellation}.
In particular, we argue Humpert and Martin stand out, as they were able to explain, in \cite{humpert2012incidence}, an elusive \textit{reciprocity relation} on graphs first presented by Stanley in \cite{stanley1975combinatorial}, where the number of acyclic orientations plays a role.

\

The method for applying new antipode formulas to obtain reciprocity results is worth describing here.
Generally speaking, a reciprocity result is a combinatorial interpretation of specialisations of a polynomial at negative integer values.
The classical examples are the characteristic polynomial of a graph $\chi_G$ or matroid $\chi_M$.
These in fact define Hopf algebra morphisms $\chi_{\eta, \cdot } : H \to \mathbb{K}[x]$, so they commute with the antipode, that is:
$$\chi_{\eta, h } (-x) = \chi_{\eta, S(h) } \, . $$
In this way, a combinatorial interpretation of the antipode $S(h)$ elicits a combinatorial interpretation of the polynomial at negative values.

\

%On a graph $G$, we define its chromatic function by counting, for each $n\geq 1$, the number of \textit{stable colourings} of its vertices.
%That is, functions $f:V(G) \to [n]$ such that each edge is not monochromatic.
%This function, $\chi_G$, turns out to be a polynomial, and its degree is the number of vertices of $G$.
%It makes sense to explore the specialisation $\chi_G(-1)$, and Stanley proved that this counts the \textit{acyclic orientations} of $G$.
%The observation of Humpert and Martin is that $\chi$ is in fact a Hopf algebra morphism from the \textbf{incidence Hopf algebra on graphs} to the polynomial Hopf algebra, so it commutes with the antipodes of each Hopf algebra.
%The antipode of the polynomial Hopf algebra is $S(x) = -x$, so combinatorial interpretations of the chromatic polynomial on negative numbers can be elicited, via
%$$\chi_G(-x) = \chi_{S(G)}(x)\, .$$

%A cancellation-free formula for the antipode $S(G)$ on the incidence Hopf algebra on graphs was also presented by Humpert and Martin, where the number of acyclic orientations plays a role and explains the previous result from Stanley.
%This antipode formula was also used to obtain new relations involving the Tutte polynomial.


\

A method for obtaining cancellation-free formulas that has had success with a large family of Hopf algebras was brought forth by Sagan and Benedetti in \cite{BS2017}, called \textit{sign-reversing involution method}.
This is a classical method in enumerative combinatorics and has found applications in fields as far as number theory e.g. in \cite{zagier2009one}.
%There, it was shown that $x^2+y^2 = p$ has integer solutions for $p$ prime whenever $p\equiv_4 1$.
%This is a classical fact, but this new proof uses an involution method.


\

Sign-reversing involution methods require careful treatment of the antipode formula of Takeuchi, and they vary widely depending on the combinatorial Hopf algebra at hand.
%Therefore, for each Hopf algebra, a new and original application of the method is needed.
Notwithstanding, this has been shown to work in the shuffle Hopf algebra, the incidence Hopf algebra on graphs, the Hopf algebras of quasisymmetric functions, and the Hopf algebra of multi-quasisymmetric functions (see \cite{BS2017}). It is still a challenging problem to find sign-reversing involutions that yield cancellation-free formulas.
For instance, in \cite{MalvenutoReutenauer,xu2022cancellation} the antipode of the Malvenuto–Reutenauer Hopf algebra was computed, but only for a restricted family of permutations.

\

Another method for finding cancellation-free formulas arose in \cite{aguiar2017hopf}.
There, a cancellation-free antipode formula for a Hopf structure on generalized permutahedra was found. 
Because several interesting combinatorial structures can be embedded in the Hopf algebra of the generalized permutahedra, this also yields a cancellation-free formula for these Hopf subalgebras.
Specific examples are graphs, matroids, posets and set partitions.
Yet another method was found in \cite{Foissy}, where the notion of \emph{bialgebra of cointeraction} was used to obtain antipode formulas for the permutation pattern Hopf algebra.

\

Parallel to this development is the study of permutation patterns.
This is a study with roots in computer science, pioneered by Knuth in \cite{Knuth}, where a description of \textit{stack sortable} permutations was presented via permutation patterns.
In the meantime, permutation patterns has become a well established area of expertise in combinatorics, see \cite{linton2010permutation}.

\

The second author, in \cite{Vargas}, introduced an algebraic tool to study permutation patterns, by constructing the permutation pattern Hopf algebra.
Specifically, we consider finite sums of functions $\pat_{\pi}$ of the form 
$$ \binom{\tau}{\pi} \coloneqq \pat_{\pi}(\tau)\coloneqq  \#\{\text{ ways to fit $\pi$ in $\tau$ }\}\, ,$$
the central functions in the study of permutation patterns.
These functions span a vector space that is closed for pointwise product, see \eqref{eq:prodperm}, and a compatible coproduct, see \eqref{eq:coprodperm}.
The corresponding Hopf algebra is the \textbf{permutation pattern Hopf algebra} $\mathcal{A}(\mathtt{Per})$, and is shown to be free in \cite{Vargas}.
For a definition of patterns in permutations, or of \say{ways to fit $\pi$ in $\sigma$}, see \cite{penaguiao2020algebraic}

\

The permutation pattern Hopf algebra construction was generalized to other combinatorial objects by the first author \cite{Penaguiao2020}, for instance graphs, marked permutations or set partitions.
This was done in such a way that any combinatorial object enriched with restriction functions, a structure that we call a \textbf{species with restrictions}, yields a \textbf{pattern Hopf algebra} (a construction presented in \cite{Penaguiao2020}, that we recover here in \cref{thm:conHopfalgebra}).
The algebraic structure of species with restrictions is a generalisation of the classical algebraic structure of combinatorial species from Joyal, see \cite{AM2010}.


\

Our most general result is summed up in \cref{thm:general_antipode}, and presents an antipode formula for any pattern Hopf algebra.
This is an application of the sign-reversing involution method, although it is not a cancellation-free and grouping-free formula for a general pattern Hopf algebra.
However, this will be applied to packed words in \cref{thm:antipode_packed} and to permutations in \cref{thm:antipode_perms}, where we indeed obtain a cancellation-free and grouping-free antipode formula for the respective pattern Hopf algebras.

\

Onwards we introduce a new polynomial invariant in permutations, by considering the evaluation character in the permutation pattern Hopf algebra, see \cref{MOPDef}.
As far as we know, this is a new family of invariants in permutations.
We apply the cancellation-free formula to interpret this polynomial in negative values, leading to a reciprocity result via the antipode formula.
Specifically, we show that the value of this polynomial at $x = -1$ counts so called \textbf{interlacing quasi-shuffle signatures} of specific permutations.

\

A parallel question posed in \cite{Penaguiao2020} is, which combinatorial objects can be faithfully represented by species with restrictions.
In that paper, naturally labelled objects like graphs and set partitions are trivially interpreted as species with restrictions.
Crucially, objects that do not have an inherent labelling are not \textit{directly} amenable to an interpretation as species.
Therefore, some unlabelled objects need original ideas to be codified into species with restrictions. 
For instance the permutation species with restrictions needs to use an interpretation of permutations as double orders.

\

In contribution to this question, we present in this paper an original species with restrictions structure on parking functions.
The idea used here is a bijection between labelled Dyck paths and parking functions, presented in \cite{Loehr,BGLPV2021}.
The underlying notion of patterns in parking functions here presented recovers the one recently presented in \cite{adeniran2022pattern}.
There, the authors study the number of parking functions that avoid a set of five parking functions of size three, recovering sequences like the Catalan numbers.
%The original introduction of patterns in parking functions hearkens back to \cite{qiu2018patterns}.
%Parking functions themselves were introduced for the first time in \cite{konheim1966occupancy}, where it was shown that there are $(n+1)^{n-1}$ parking functions of length $n$.

\

We now discuss the structure of the paper.
In what remains of \cref{sec:intro}, we introduce the product and coproduct structure on the permutation pattern Hopf algebra and give more details on the main result of this paper, the new formula for the antipode in permutation patterns.
In \cref{sec:antipode_computing}, we present Takeuchi's formula along with some examples of applications of this formula.
An introduction to the basic notion of Joyal species and algebraic structures related to these is presented in \cref{sec:species}. In \cref{sec:pattern_algebra_contruction}, we present the algebra and category theory background for pattern Hopf algebras. We recall the pattern Hopf algebra construction from \cite{Penaguiao2020} from a species with restrictions, so that this article is self contained, summarised in \cref{thm:conHopfalgebra}.
In \cref{sec:species_restrictions}, we present several examples of species with restrictions, centrally the one on permutations, but also an original one in parking functions.
%We also present here species with restrictions on Dyck paths and on parking functions, and present simple examples of patterns in parking functions.
In \cref{sec:formula_general,sec:formula_pp}, we present the main result, the cancellation-free and grouping-free formula for the antipode in packed word patterns and in permutation patterns.
%In this section we start by describing the sign-reversing involution method to obtain a cancellation-free formula for the antipode of a Hopf algebra.
%Then we present the main result of this section in \cref{thm:antipode_perms_intro}, a cancellation-free formula for the antipode of $\mathcal{A}(\mathtt{Per})$.
Finally, in \cref{sec:reciprocity}, we present an application of the cancellation-free and grouping-free formula on permutations by obtaining a reciprocity result in a new ``pattern-sensitive'' polynomial invariant, called the \emph{multiple occurrence polynomial} on a species with restrictions (MOP for short, see definition \eqref{MOPDef}).



\subsection{The permutation pattern Hopf algebra and the main result}

In this section we introduce the Hopf algebra structure on permutation patterns, and present the main result on this paper: a cancellation-free and grouping-free antipode formula on permutation patterns Hopf algebra.

\

Let $\pat_{\pi}$ be a function on permutations, so that $\pat_{\pi}(\sigma)$ counts the number of restrictions of the permutation $\sigma$ that fit the pattern $\pi$.
In this way, the collection of permutation pattern functions $\{\pat_{\pi}\}$ is linearly independent, so it is a basis of a vector space $\mathcal A (\mathtt{Per})$.
Further, in \cite{Vargas}, it was shown that the pointwise product of two such functions can be expressed as a sum of other permutation pattern functions:
\begin{equation}\label{eq:prodperm}
\pat_{\pi_1} \pat_{\pi_2} = \sum_{\sigma} \binom{\sigma}{\pi_1, \pi_2} \pat_{\sigma} \, ,
\end{equation}
where the sum runs over all permutations $\sigma$ of any size.
If $|\sigma| > |\pi_1| + |\pi_2|$ we have $\binom{\sigma}{\pi_1, \pi_2} = 0$, in fact, the coefficients $\binom{\sigma}{\pi_1, \pi_2}$ that arise in this product formula count the so called \textbf{quasi-shuffle signatures}, or \textbf{QSS}, of $\sigma$ from $\pi_1, \pi_2$.
Let us make this concept precise:

\begin{defin}[QSS on permutations]
A quasi-shuffle signature, or QSS, of $\sigma$ from $\pi_1, \dots, \pi_n$ is a tuple $\III = (I_1, \dots, I_n)$ of sets on the ground set of the permutation $\sigma$, that cover this ground set in such a way that each $I_i$ is a pattern of $\pi_i$ in $\sigma$ --- that is the restricted permutations $\sigma|_{I_i}$ are the permutation $\pi_i$ for any $i$.

The sets $I_1, \ldots, I_l$ are not necessarily disjoint but they must satisfy $\bigcup_i I_i = I$, where $I$ is the ground set of $\sigma $.
We let $\binom{\sigma}{\pi_1, \dots, \pi_n}$ denote the number of such QSS.
It was shown in \cite{Penaguiao2020} that $\binom{\sigma}{\pi_1, \dots, \pi_n}$ is precisely the coefficient that arises in the iterated product of $n$ elements.
This fact justifies the notation choice in \eqref{eq:prodperm}.

\end{defin}

\

This algebra $\mathcal A(\mathtt{Per})$ can be endowed with a Hopf algebra structure with the help of the diagonal sum of permutations, $\oplus$, also called shifted concatenation of permutations.
If one lets $\pi = \pi_1 \oplus \dots \oplus \pi_n$ be the decomposition of $\pi$ into $\oplus$-indecomposable permutations under the $\oplus$ product, we define the shifted deconcatenation coproduct

\begin{equation}\label{eq:coprodperm}
\Delta \pat_{\pi} = \sum_{k=0}^n \pat_{\pi_1\oplus \dots \oplus \pi_k} \otimes \pat_{\pi_{k+1}\oplus \dots \oplus \pi_n} \in \mathcal A (\mathtt{Per}) \otimes \mathcal A (\mathtt{Per})\, .
\end{equation}

To present a cancellation-free antipode formula we need to refine the notion of QSS.
For sets $A, B$ of integers, we write $A <B$ if $a<b $ for any $a\in A, b \in B$.

\begin{defin}[Interlacing QSS on permutations]
A QSS $\III = (I_1, \dots, I_n)$ is said to be \textbf{non-interlacing} if there is some $i=1, \dots, n-1$ such that $I_i < I_{i+1}$ and $\sigma(I_i) < \sigma(I_{i+1})$.
Otherwise, we say that the QSS is \textbf{interlacing}. 

\

Note that, unlike in the case on QSS, reordering an interlacing QSS does not in general give an interlacing QSS.
We let $\bigl[\!\begin{smallmatrix} \sigma \\ \pi_1, \dots, \pi_n \end{smallmatrix}\!\bigr]$ denote the number of interlacing QSS.
\end{defin}


\begin{thm}[Antipode formula for permutation pattern Hopf algebra]\label{thm:antipode_perms_intro}
Let $\pi$ be a permutation that factors $\pi = \pi_1\oplus \dots \oplus \pi_n$ into $\oplus$-indecomposable permutations.
Then, we have the following formula for the antipode of $\pat_{\pi}$:

$$S(\pat_{\pi}) = (-1)^n \sum_{\sigma} \bigl[\!\begin{smallmatrix} \sigma \\ \pi_1, \dots, \pi_n \end{smallmatrix}\!\bigr] \pat_{\sigma}\, ,$$
where the sum runs over all permutations $\sigma$, and the coefficients count the number of \textbf{interlacing QSS} of $\sigma$ from $\pi_1, \dots, \pi_n$.
\end{thm}

Note that for $|\sigma| > \sum_i |\pi_i|$, the coefficient $\bigl[\!\begin{smallmatrix} \sigma \\ \pi_1, \dots, \pi_n \end{smallmatrix}\!\bigr]$ vanishes, so this sum is finite.

\

In the following we present some examples that help explain QSS and interlacing QSS.
%We also present an antipode formula for the packed words pattern Hopf algebra, with a similar characterisation.

\


\begin{figure}[h]
    \centering
    \includegraphics{../images/interlacing_25314_square.pdf}
    \caption{\textbf{Left:} the permutation 2314, along with a labelling of two of its QSS from $1, 21, 1, 1$. In orange the two sets that do not interlace. \textbf{Right:} the permutation 123, along with its two interlacing QSS from $1, 12$.\label{fig:interlacingQSSsmpl}}
\end{figure}


\begin{smpl}[Interlacing QSS]
The permutation $2314$ has several QSS from $1, 21, 1, 1$, for instance $(4, 13, 2, 1)$ and $(1, 13, 4, 2)$, but from these two, only the first is interlacing.
In \cref{fig:interlacingQSSsmpl}, one can see these two QSS.
Further computations can show that $\binom{2314}{1, 21, 1, 1} = 36$ and $\bigl[\!\begin{smallmatrix} 2314 \\ 1, 21, 1, 1 \end{smallmatrix}\!\bigr] = 8$.

\

One can observe that there are three QSS of $123$ from $1$, $12$, but one of them is non-interlacing (the QSS $(1,23)$), so
$\bigl[\!\begin{smallmatrix} 123 \\ 1, 21 \end{smallmatrix}\!\bigr] = 2$.
In \cref{fig:interlacingQSSsmpl}, one can see these two interlacing QSS.
\end{smpl}



\section{Computing the antipode\label{sec:antipode_computing}}

In this section we explore the antipode of a Hopf algebra using Takeuchi's formula.
We explore examples on the polynomial algebra and on the permutation pattern Hopf algebra.
We start by recalling Takeuchi's formula, in the form that is presented in \cite{GrinbergReiner}, as well as some convenient notation.
%This is the beginning of any cancellation-free formula.
Let us define the $\star$ notation on linear maps $a, b: C \to A$.
Whenever $A$ is an algebra, and $C$ is a coalgebra, we define:
$$a \star b \coloneqq \mu_A \circ (a \otimes b) \circ \Delta_C\, ,$$
which is an associative and unitary product on linear maps from $C$ to $A$. We will be focused on the case when $C=A=H$ is a Hopf algebra, so this defines a convolution operation on $\End(H)$.

The following result is from \cite[Lemma 14]{Takeuchi1971}.

\begin{prop}[Takeuchi's formula]\label{lm:takeuchi}
If $H = (H, \mu, \iota, \Delta, \epsilon, S)$ is a Hopf algebra such that $(\iota\circ \epsilon - \id_H)$ is $\star$-nilpotent (such as in any filtered Hopf algebra), then 
\begin{equation}\label{eq:eq1}
S = \sum_{k\geq 0 }  ( \iota  \circ\epsilon- \id_H)^{\star k} = \sum_{k\geq 0} (-1)^k \mu^{\circ (k-1)} \circ (\id_{H} - \iota \circ \epsilon)^{\otimes k} \circ \Delta^{\circ (k-1)}\, ,
\end{equation}
where we use the convention that $\Delta^{\circ (-1)} = \epsilon $ and $\mu^{\circ (-1)} = \iota$.
Notice that the $\star$-nilpotent property ensures that this sum is finite.
\end{prop}

Note that, any pattern algebra is a filtered Hopf algebra, so for our objects of study we can always apply Takeuchi's formula.
%This is summarized in \cref{thm:conHopfalgebra}.

\

In the Hopf algebra of polynomials, this gives us the following:
$$S(x^3) = \underbrace{0}_{k = 0} - \underbrace{x^3}_{k = 1} + \underbrace{3 x^2 \cdot x + 3 x \cdot x^2}_{k=2} - \underbrace{6 x \cdot x \cdot x}_{k = 3} = - x^3 \, .$$


We now present another example, this time on the permutation pattern Hopf algebra $\mathcal A(\mathtt{Per})$.
Consider $\pi = 132 = 1 \oplus 21$. Then Takeuchi's formula gives us:
\begin{align*}
S(\pat_{132}) =& \sum_{k=0}^2 (-1)^k \mu^{\circ (k-1)} \circ (\id_{\mathcal A(\mathtt{Per})} - \iota \circ \epsilon)^{\otimes k} \circ \Delta^{\circ (k-1)}(\pat_{132})\\
=& -(\id_{\mathbb{K}[x]} - \iota \circ \epsilon)(\pat_{132}) + \mu \circ (\id_{\mathbb{K}[x]} - \iota\circ\epsilon)^{\otimes 2}(\pat_1 \otimes \pat_{12}) \\
=& - \underbrace{\pat_{132}}_{k=1} + \underbrace{\pat_1 \pat_{21}}_{k=2} \\
=& 3 \pat_{321} + 2 \pat_{231} + 2 \pat_{312} + \pat_{213} + 2 \pat_{21} \, .
\end{align*}

\

These coefficients can be seen as enumerating quasi-shuffle signatures of $132$ from $1$ and $12$ that are \textbf{interlacing}, according to \cref{thm:antipode_perms_intro}.



\

\subsection{The sign-reversing involution method}

The application of the sign-reversing involution method to compute antipodes of Hopf algebras was first presented in \cite{BS2017}.
This is a method to find cancellation-free formulas for the antipode of a Hopf algebra.
It starts in the formula given by Takeuchi, and results in a $\pm1$ sum that runs over a collection of objects, say $\mathcal O$.
An involution $\zeta $ is an endomorphism such that $\zeta \circ \zeta$ is the identity.
A sign-reversing involution $\zeta $, is an involution on $\mathcal O$ such that if $\zeta(x) \neq x$, then $x$ and $\zeta(x)$ contribute with opposite signs to the sum over $\mathcal O$ that results in the antipode.
As a consequence, when applying Takeuchi's formula, we can cancel terms that are not fixed points of $\zeta$.

%keeps track of all the terms to be summed, usually by means of compositions, that arise in this formula.
%Thus, the sum obtained runs over a collection of objects, say $\mathcal O$, that is partitioned into families indexed by compositions.
%These compositions play an important role in the sum, as their length determines the sign of the corresponding objects.

%\
%
%Recall that an involution $\zeta $ is an endomorphism such that $\zeta \circ \zeta$ is the identity.
%We describe an involution in $\mathcal O$, call it $\zeta $, in such a way that if $\zeta(x) \neq x$, then $x$ and $\zeta(x)$ contribute with opposite signs to the antipode.

\

%We give an example of how this method is applied in the Hopf algebra $\mathbb{K}[x]$.
%The following is a computation from \cite{BS2017}.
%There are easier ways to obtain an antipode formula for the polynomial Hopf algebra: the interested reader can find some for instance in \cite{GrinbergReiner}.
%However, this example shows the power of sign-reversing involutions, as the whole process can be done with little recourse to intuition after picking the right involution.

An example of how this method is applied in the Hopf algebra $\mathbb{K}[x]$ is given in \cite{BS2017}.


\begin{thm}[The antipode formula for the polynomial Hopf algebra]\label{thm:polyHA}
The antipode $S$ for $\mathbb{K}[x] $ is 
$$ S(x^n) =(-x)^n\, . $$
\end{thm}
%
%To prove this formula, we first introduce \textit{weak compositions}.
%A \textbf{weak composition} $\alpha$ of an integer $n$ is a list of non-negative integers $\alpha = (\alpha_1, \dots , \alpha_l)$ such that $\sum_i \alpha_i = n$.
%We denote the length of the weak composition by $\ell = \ell(\alpha)$, and use the shorthand notation $\alpha \models^0 n$.
%If $\alpha $ has no zero entries, we say that $\alpha$ is a \textbf{composition}, and write $\alpha \in \mathcal C_n$.
%
%\
%
%A \textbf{weak set composition} $\opi$ of a set $A$ is a list of pairwise disjoint sets $\opi = (A_1, \dots , A_{\ell})$ such that $\bigcup_i A_i = A$.
%Note that some sets may be empty.
%We denote the length of the weak composition by $\ell = \ell(\opi)$, and use the shorthand notation $\opi \models^0 A$ to indicate that $\opi$ is a weak set composition of the set $A$.
%If the weak set composition does not contain any empty set, we say that this is a \textbf{set composition}, and we denote $\opi \models A$ to indicate that $\opi$ is a set composition of the set $A$.
%There are finitely many set compositions of a given set $A$, whereas there are infinitely many weak set compositions of $A$.
%Let $\mathbf{C}_A$ be the collection of set composition of $A$.
%We abbreviate to $\mathbf{C}_n$ when $A = [n]$.
%
%\begin{proof}[Proof of \cref{thm:polyHA}]
%We use Takeuchi formula, given in \eqref{eq:eq1}, which holds for graded Hopf algebras,
%$$S(x^n) = \sum_{k = 0} (-1)^k \mu^{\circ (k-1)} \circ (\id_{\mathbb K[x]} - \iota \circ \epsilon )^{\otimes k} \circ \Delta^{\circ (k-1)} (x^n) \, .$$
%
%The following results from induction on $k$:
%$$ \Delta^{\circ (k-1) } (x^n) = \sum_{(A_1, \dots , A_k) \models^0 [n] } x^{|A_1|} \otimes \dots \otimes x^{|A_k|} \, ,$$
%where the sum runs over weak set compositions of the set $[n]$ with lenght $k$.
%Thus, the antipode formula can be rewritten as
%
%\begin{align*}
%S(x^n)&= \sum_{k=0}^n(-1)^k \sum_{(A_1, \dots , A_k) \models^0 [n]} \mu^{\circ (k-1)} \circ (\id_{\mathbb K[x]} - \iota \circ \epsilon )^{\otimes k} (x^{|A_1|} \otimes \dots \otimes x^{|A_k|})   \\
%	  &= \sum_{k=0}^n(-1)^k \sum_{(A_1, \dots , A_k) \models [n]} \mu^{\circ (k-1)} (x^{|A_1|} \otimes \dots \otimes x^{|A_k|})   \\
%	  &= \sum_{k=0}^n(-1)^k \sum_{(A_1, \dots , A_k) \models [n]} x^n   \\
%	  &= x^n \sum_{\opi \models [n]} (-1)^{\ell(\opi)}\, .
%\end{align*}
%
%Consider the following involution $\zeta: \oPi_{n} \to \oPi_{n} $.
%For $\opi = (A_1, \dots , A_k) $, let $j_{\opi}$ be the smallest index such that $|A_{j_{\opi}}| \neq 1$ or $\max A_{j_{\opi}} > \max A_{j_{\opi}+1} $.
%Then, there are three cases:
%
%\begin{enumerate}
%
%\item The set $A_{j_{\opi}} $ is a singleton with $j_{\opi}\leq k-1$, then let $\zeta(\opi ) $ be the set composition resulting from merging $A_{j_{\opi}} $ and $A_{j_{\opi}+1}$.
%Note how, in this case, $\max ( A_{j_{\opi}} \cup A_{j_{\opi} + 1} ) $ is the only element in $A_{j_{\opi}}$.
%
%\item The set $A_{j_{\opi}} $ is not a singleton, then we define $\zeta(\opi ) $ to be the set composition resulting from splitting $A_{j_{\opi}} $ into $\{ \max A_{j_{\opi}} \} $ and $A_{j_{\opi}} \setminus \{\max A_{j_{\opi}} \}$, in this order.
%
%\item There is no such $j_{\opi}$. Then $\opi = (\{1\}, \dots , \{n\})$ and we define $\zeta(\opi )= \opi $.
%
%\end{enumerate}
%
%It is a direct observation that $\zeta $ is an involution.
%In fact, the only fixed point is $\opi = (\{1\}, \dots , \{n\})$, and for any other set composition $\opi$, whenever the index $j_{\opi}$ in $\opi $ behaves as described in case 1, then the index $j_{\zeta(\opi)}$ in $\zeta ( \opi ) $ behaves as described in case 2, in which case we have $l(\opi) = 1+\ell(\zeta (\opi))$ and we can easily see that $\zeta(\zeta(\opi )) = \opi$.
%%We also have the converse statement. 
%Thus, we have that 
%\[x^n \sum_{\opi \in \oPi_n} (-1)^{\ell(\opi)} = x^n (-1)^{\ell(\{1\}, \dots, \{n\} )}= (-x)^n,\] 
%as desired.
%\end{proof}
%
%
%In this way we see that a formula for the antipode depends simply on an understanding of the structure of the length of compositions of $[n]$.
%This is a general feature whenever we apply Takeuchi's formula.
%
%



\


\section{Combinatorial species\label{sec:species}}


In this section we give the preliminaries on monoids in species.
This will follow closely \cite{AM2010} and \cite{Schmitt1993}.
Specifically, we introduce species with vector spaces and sets.
We also introduce species with restrictions, and we will clarify the meaning of a monoid, comonoid, bimonoid and Hopf monoid in each of these monoidal categories.
We finally present some examples of species with restrictions that will be important in the remaining paper.


\begin{figure}
\centering
\includegraphics[scale=.6]{../images/comuting_diagram_algebras.png}
\caption{Diagram reflecting the context of the pattern Hopf algebra. 
Given a restriction species $\tr$, its linearisation is denoted $\mathbf{r} = \mathbb{K}\tr$. 
The product $\ast$ in the top-right corner corresponds to the pointwise product of functions in the dual space $\overline{\mathcal{K}}(\mathbf{r})$. The coproduct $\Delta_{\square} $ of the pattern Hopf algebra corresponds to the dual of the concatenation product $\square$, after applying the Fock functor $\overline{\mathcal{K}}$.}
\end{figure}

\

\subsection{Species}
In this section we recall the basic definitions of the general theory of \emph{combinatorial species}. Following \cite{AM2010}, we will focus first in \emph{vectorial species} and \emph{set species}.

\

Let $\mathbb{K}$ be a field of arbitrary characteristic. Let $\Fset$ be the category of finite sets and bijections between finite sets, and $\Vect_{\mathbb{K}}$ be the category of $\mathbb{K}$-vector spaces and linear maps between vector spaces. A {\bf vector species}, or simply a \textbf{species}, is a functor $\tp: \Fset \to \Vect_{\mathbb{K}}$. A morphism between species $\tp$ and $\tq$ is a natural transformation between the functors $\tp$ and $\tq$.
For clarity, we denote the vector species with a bold lowercase Latin letter, with few exceptions.

\

A species $\tp$ is said to be {\bf positive} if $\tp[\emptyset]=0$. The {\bf positive part} of a species $\tq$ is the positive species $\tq_+$ given by
\[\tq_+[I]=\begin{cases}
\tq[I],& \text{ if } I\neq \emptyset\\
\emptyset, & \text{ otherwise}
\end{cases}.\]

Given a vector space $V$, let ${\bf 1}_V$ be the vector species defined by
\[{\bf 1}_V[I]=\begin{cases}
V,& \text{ if } I= \emptyset\\
\emptyset, & \text{ otherwise}
\end{cases}.\]

\

We write $\Ss_{\mathbb{K}}$ for the category of vector species over the field $\mathbb{K}$.
We consider two monoidal structures on this category: the {\bf Cauchy} and {\bf Hadamard} products $\cdot$ and $\times$, respectively: for any finite set $I$,
\[(\tp \cdot \tq)[I]:=\bigoplus_{I = S \sqcup T}\tp[S]\otimes \tq[T];\]
\[(\tp \times \tq)[I]:=\tp[I] \otimes \tq[I].\]

We denote by $(\Ssk, \cdot)$ the resulting monoidal category obtained from the Cauchy operation.

\

We can also consider {\bf set species}, these are functors $\rP: \Fset \to \Set$, where $\Set$ is the category of arbitrary sets and arbitrary maps between sets. Given a set species $\rP$, the notions of {\bf positive part} $\rP_+$ of $\rP$, {\bf positive set species} are defined analogously as for vector species. If $C$ is a set, let ${\bf 1}_C$ be the set species defined by
\[{\bf 1}_C[I]=\begin{cases}
C,& \text{ if } I= \emptyset\\
\emptyset, & \text{ otherwise}
\end{cases},\]
for any finite set $I$.
For clarity, we denote a set species with a capital Latin letter, with few exceptions.

\

The Cauchy and Hadamard products of vector species have their analogues in this context. For instance, if $\rP$ and $\rQ$ are two set species, let
\[(\rP \cdot \rQ)[I]:=\bigsqcup_{I = S \sqcup T}\rP[S]\times \rQ[T];\]
\[(\rP \times \rQ)[I]:=\rP[I]\times  \rQ[I],\]
on any finite set $I$, where the $\times$ symbol in the right-hand sides refers to the Cartesian product.

\

It is possible to relate set species to vector species via the \emph{linearisation functor} $\mathbb{K}(-): \Set \to \Vect_{\mathbb{K}}$, which sends a set to the vector space generated by the given set. Composing a set species $\rP$ with the linearisation functor gives a vector species, denoted by $\mathbb{K}\rP$. A {\bf linearized species} is a vector species $\tp$ of the form $\tp=\mathbb{K}\rP$, for some set species $\rP$. We have natural isomorphisms
 \[\mathbb{K}(\rP \cdot \rQ)\simeq \mathbb{K}\rP \cdot \mathbb{K}\rQ \qquad , \qquad \mathbb{K}(\rP \times \rQ)\simeq \mathbb{K}\rP \times \mathbb{K}\rQ.\]

\

\subsection{Algebraic structures on vector species}


\subsubsection{Monoids}
A {\bf monoid} in $(\Ssk, \cdot)$ consists of a species $\tp$ equipped with morphisms of species
\begin{equation*}
    \mu: \tp \cdot \tp \to \tp \qquad \text{ and } \qquad \iota: \mathrm{1}_{\mathbb{K}} \to \tp.
\end{equation*}

That is, for each finite set $I$ and for each decomposition $I=S \sqcup T$, we have a linear map 
\begin{equation*}
    \mu_{S,T}: \tp[S] \otimes \tp[T]\to \tp[I] \text{ and } \iota_\emptyset: \mathbb{K} \to \tp[\emptyset].
\end{equation*}

If $x \in \tp[S]$, $y \in \tp[T]$,  let 
\[x \cdot y \in \tp[I]\]
denote the image of $x\otimes y$ under $\mu_{S,T}$. 

\

The collection of linear maps $\mu=(\mu_{S,T})$, called the {\bf product} of the monoid, must satisfy the following axioms.

\

\begin{itemize}
    \item[(i)] Naturality axiom: for finite sets $I,J$, a bijection $\sigma: I \to J$, a decomposition $I=S \sqcup T$ and elements $x \in \tp[S]$ and $y \in \tp[T]$, we have
\begin{equation*}
\tp[\sigma](x \cdot y)=\tp[\sigma(S)](x) \cdot \tp[\sigma(T)](y) \, .
\end{equation*}

\item[(ii)] Associativity axiom: for finite set $I$, a decomposition $I=R \sqcup S \sqcup T$ and for elements $x \in \tp[R]$, $y \in \tp[S]$ and $z \in \tp[T]$, we have
\begin{equation}\label{eq:axiomii}
    (x \cdot y)\cdot z=x \cdot (y \cdot z).
\end{equation}

\item[(iii)] Unit axiom: for each finite set $I$ and $x \in \tp[I]$, we have
\begin{equation*}
 x \cdot \iota_\emptyset(1) = x =  \iota_\emptyset(1) \cdot x,
\end{equation*}
where $1\in \mathbb{K}$ is the unit of the field $\mathbb{K}$.

\end{itemize}


\

A monoid $(\tp, \mu, \iota)$ in $(\Ssk, \cdot)$ is {\bf commutative} if
\begin{equation*}
    x\cdot y=y\cdot x,
\end{equation*}
for all $I=S \sqcup T$, $x \in \tp[S]$ and $y \in \tp[T]$.

\

Let $(\tp, \mu, \iota)$ be a monoid. From the associativity axiom, for any decomposition $I=S_1 \sqcup \cdots \sqcup S_k$ with $k \geq 2$ there is a unique map
\begin{equation}
    \mu_{S_1, \hdots, S_k}: \tp[S_1]\otimes \cdots \otimes \tp[S_k] \to \tp[I],
\end{equation}
called the {\bf higher product map} of $\tp$, obtained by iterating the product maps in any meaningful way. 
This is well defined from \eqref{eq:axiomii}.
We can extend the definition of higher product map for all $k\geq 0$: for $k=1$, $\mu_I$ is defined as the identity map of $\tp[I]$ ($I$ is the only decomposition of itself in one block); if $k=0$, then $\mu_{\emptyset }:=\iota_{\emptyset }$.
%Note how in this case $I = \emptyset $.
\

Monoids are closed under the Cauchy product. Also, if $(\tp, \mu, \iota)$ is a monoid, then $\tp[\emptyset]$ is an algebra with product $\mu_{\emptyset, \emptyset}$ and unit $\iota_\emptyset(1)$ (see \cite{AM2013}, section 2.3).


\

\subsubsection{Comonoids}
A {\bf comonoid} in $(\Ssk, \cdot)$ corresponds to the dual of a monoid in $(\Ssk, \cdot)$.
Specifically, a comonoid consists of a species $\tp$ equipped with morphisms of species
\begin{equation*}
    \Delta: \tp \to \tp \cdot \tp \qquad \text{ and } \qquad \varepsilon: \tp \to 1_{\mathbb{K}}.
\end{equation*}

That is, for each finite set $I$ and for each decomposition $I=S \sqcup T$, we have a linear map 
\begin{equation*}
    \Delta_{S,T}: \tp[I] \to \tp[S] \otimes \tp[T] \text{ and } \varepsilon_\emptyset: \tp[\emptyset] \to \mathbb{K}.
\end{equation*}

\

The collection of linear maps $\Delta=(\Delta_{S,T})$, called the {\bf coproduct} of the monoid, must satisfy the following axioms.

\

\begin{itemize}
    \item[(i)] Naturality axiom: for finite sets $I, J$, bijection $\sigma: I \to J$, a decomposition $I = S \sqcup T$ and an element $x\in \tp [I]$.
\begin{equation*}
(\tp[\sigma|_S] \otimes \tp[\sigma|_T])\circ\Delta_{S,T}(x)=\Delta_{\sigma(S), \sigma(T)} \circ\tp[\sigma](x).
\end{equation*}

\item[(ii)] Coassociativity axiom: for a finite set $I$, a decomposition $I=R \sqcup S \sqcup T$ and for each $x \in \tp[I]$, we have
\begin{equation}\label{eq:coaxiomii}
    (\Delta_{R,S}\otimes \text{id}_{\tp[T]})\circ \Delta_{R \sqcup S, T}(x)=(\text{id}_{\tp[R]} \otimes \Delta_{S,T})\circ \Delta_{R, S \sqcup T}(x).
\end{equation}
\vspace{.1in}
\item[(iii)] Counit axiom: for each finite set $I$ and $x \in \tp[I]$, we have
\begin{equation*}
(\varepsilon_\emptyset \otimes \text{id}_{\tp[I]})\circ\Delta_{\emptyset, I}(x)= x = (\text{id}_{\tp[I]} \otimes \varepsilon_\emptyset)\circ\Delta_{I,\emptyset}(x).
\end{equation*}

\end{itemize}


\

A comonoid $(\tp, \mu, \iota)$ in $(\Ssk, \cdot)$ is {\bf cocommutative} if for any finite disjoint sets $S, T$, and element $x\in \tp[S\sqcup T]$, we have
\begin{equation*}
    \Delta_{S,T}(x)=\Delta_{T,S}(x).
\end{equation*}

\

Let $(\tp, \mu, \iota)$ be a comonoid. Dually to the case of monoids, every decomposition $I=S_1 \sqcup \cdots \sqcup S_k$ with $k \geq 0$ gives rises to a unique linear map
\begin{equation}
    \Delta_{S_1, \hdots, S_k}: \tp[I] \to \tp[S_1]\otimes \cdots \otimes \tp[S_k],
\end{equation}
called the {\bf higher coproduct map} of $\tp$, obtained by iterating the coproducts map $\Delta_{S,T}$. 
This is well defined because of \cref{eq:coaxiomii}.
We extend this definition to $k=1$ as the identity of $\tp[I]$ and for $k=0$, this map is the counit map $\varepsilon_\emptyset$.

\

Comonoids are closed under the Cauchy product. 
If $(\tp, \Delta, \varepsilon)$ is a comonoid, then $\tp[\emptyset]$ is a coalgebra with coproduct $\Delta_{\emptyset, \emptyset}$ and counit $\varepsilon_\emptyset$ (see \cite{AM2013}, section 2.4).

\

\subsubsection{Bimonoids  and Hopf monoids}
A {\bf bimonoid} $(\thh, \mu, \Delta, \iota, \varepsilon)$ in $(\Ssk, \cdot)$ is a monoid and comonoid such that the diagram
\[\xymatrix{
\thh[A] \otimes \thh[B] \otimes \thh[C] \otimes \thh[D] \ar[rr]^-{\cong} && \thh[A] \otimes \thh[C] \otimes \thh[B] \otimes \thh[D] \ar[dd]^-{\mu_{A,C}\otimes\mu_{B,D}}\\
&&\\
\thh[S_1]\otimes \thh[S_2]\ar[uu]^-{\Delta_{A,B}\otimes \Delta_{C,D}} \ar[r]_-{\mu_{S_1, S_2}} & \thh[I] \ar[r]_-{\Delta_{T_1, T_2}} & \thh[T_1]\otimes \thh[T_2]
}\]
commutes, where $I=S_1\sqcup S_2=T_1 \sqcup T_2$ are two decompositions of a finite set $I$ with the following resulting pairwise intersections:
\[A:=S_1\cap T_1 \quad , \quad B:=S_1 \cap T_2 \quad , \quad C:=S_2 \cap T_1 \quad , \quad D:=S_2 \cap T_2.\]
This is also schematically presented in \cref{fig:bimonoid}.


\begin{figure}
\includegraphics[scale=0.5]{../images/diagram_bialgebra}
\caption{The bimonoid compatibility axiom.\label{fig:bimonoid}}
\end{figure}

\


A morphism of species $s: \thh \to \thh$, is called and {\bf antipode} of $\thh$, if $\thh[\emptyset]$ is a Hopf algebra with antipode $s_\emptyset: \thh[\emptyset] \to \thh[\emptyset]$, and for each nonempty set $I$, we have
\begin{equation}\label{eq:antipode_species}
    \sum_{S \sqcup T = I} \mu_{S,T}(\text{id}_S \otimes s_T)\Delta_{S,T} = 0 = \sum_{S \sqcup T = I} \mu_{S,T}(s_S \otimes \text{id}_T)\Delta_{S,T}.
\end{equation}


We define the convolution algebra $\text{End}_{\Ssk}(\thh)$ as the monoid of natural transformations $l: \thh \to \thh$ with the product $\star$.
Note how $\iota \circ \epsilon $ is the identity.
In this case, \eqref{eq:antipode_species} can be rephrased as $s = \id_{\thh}^{\star -1}$.

A {\bf Hopf monoid} in $(\Ssk, \cdot)$ is a bimonoid along with an antipode $s:\thh \to \thh $.
Recall that this is also the case in the classical Hopf algebras.

\

\subsection{Algebraic structures on set species}

The notions of monoid, comonoid, bimonoid and Hopf monoid for set species can be described in terms similar to those in the previous section. 

\

\subsubsection{Monoids}
A {\bf monoid in set species} is a set species $\rP$ equipped with morphisms of species
\begin{equation}
    \mu: \rP \cdot \rP \to \rP \qquad \text{ and } \qquad \iota: \mathrm{1}_{\{\emptyset \} } \to \rP.
\end{equation}

That is, for each decomposition $I=S \sqcup T$, we have maps 
\begin{equation}
    \mu_{S,T}: \rP[S] \times \rP[T]\to \rP[I] \text{ and } \iota_\emptyset: \{\emptyset\} \to \rP[\emptyset].
\end{equation}

If $x \in \rP[S]$, $y \in \rP[T]$,  let 
\[x \cdot y \in \rP[I]\]
denote the image of $x$ under $\mu_{S,T}$. Also, let $e\in \rP[\emptyset]$ denote the image of $\emptyset$ under $\iota_\emptyset$.

\

The collection of maps $\mu=(\mu_{S,T})$, called the {\bf product} of the comonoid, must satisfy naturality, associativity and unit axioms analogue to the ones defined for monoids in vector species.

\

Note that, for any monoid on a set species $(\rP, \mu, \iota)$, then $(\rP[\emptyset], \mu_{\emptyset, \emptyset}, \iota_\emptyset)$ is a set theoretical monoid.

\

A monoid in set species $(\rP, \mu, \iota)$ is {\bf commutative} if
\[x\cdot y=y\cdot x,\]
for all $I=S \sqcup T$, $x \in \rP[S]$ and $y \in \rP[T]$.

\

\subsubsection{Comonoids}
A {\bf comonoid in set species} consist of a species $\rP$ equipped with morphisms of species
\[\Delta:\rP \to \rP \cdot \rP \qquad \text{ and } \qquad \varepsilon: \rP \to \mathrm{1}_{\{\emptyset \} }. \]
That is, for each decomposition $I=S \sqcup T$ we have maps $\Delta_{S,T}: \rP[I]\to \rP[S] \times \rP[T]$, and $\varepsilon_\emptyset: \rP[\emptyset] \to \{\emptyset\}$. If $x \in \rP[I]$,  let 
\[(x|_S, x /_S)\in \rP[S]\times \rP[T]\]
denote the image of $(x,y)$ under $\Delta_{S,T}$. The map $x \mapsto x|_S$ can be thought as a ``restriction'' of the structure $x$ from $I$ to $S$, while $x \mapsto x/_S$ can be associated to a ``contraction'' of $S$ from $x$, resulting in a structure on $T$.

\

The collection of maps $\Delta=(\Delta_{S,T})$, called the {\bf coproduct} of the monoid, must satisfy naturality, coassociativity and counit axioms analogues to the ones defined for monoids in vector species.
These axioms can be described explicitly:
\begin{itemize}
    \item Naturality axiom: for each bijection $\sigma: I \to J$,  we have
\[{\Big (}\rP[\sigma](x){\Big )}\Big|_{\sigma(S)}=\rP[\sigma|_S](x|_S) \qquad , \qquad {\Big (}\rP[\sigma](x){\Big )}\Big/_{\sigma(S)}=\rP[\sigma|_T](x/_S),\]
for all $x \in \rP[I]$.
\vspace{.1in}
\item Coassociativity axiom: for all decomposition $I=R \sqcup S \sqcup T$, \[(x|_{R\sqcup S})|_R=x|_R \qquad , \qquad (x|_{R\sqcup S})/_R=(x/_R)|_S \qquad , \qquad x/_{R \sqcup S}=(x/_R)/_S, \]
for all $x \in \rP[I]$
\vspace{.1in}
\item Counit axiom: we have
\[x|_I= x = x/_\emptyset,\]
for each finite set $I$ and for each $x \in \rP[I]$. In particular, $\Delta_{\emptyset, \emptyset}(x)=(x,x)$, for each $x \in \rP[\emptyset]$.
\end{itemize}

\


A comonoid in set species $(\rP, \Delta, \varepsilon)$ is {\bf cocommutative} if
\[x|_S=x/_T,\]

for any disjoint finite sets $S, T$ and $x \in \rP[S\sqcup T]$.

\

\subsubsection{Bimonoids}
A bimonoid in set species $(\rH, \mu, \Delta, \iota, \varepsilon)$ is a monoid and comonoid in set species such that the diagram
\[\xymatrix{
\rH[A] \times \rH[B] \times\rH[C] \times \rH[D] \ar[rr]^-{\cong} && \rH[A] \times \rH[C] \times \rH[B] \times \rH[D] \ar[dd]^-{\mu_{A,C}\times\mu_{B,D}}\\
&&\\
\rH[S_1]\times \rH[S_2]\ar[uu]^-{\Delta_{A,B}\times \Delta_{C,D}} \ar[r]_-{\mu_{S_1, S_2}} & \rH[I] \ar[r]_-{\Delta_{T_1, T_2}} & \rH[T_1]\times \rH[T_2]
}\]
commutes, where $I=S_1\sqcup S_2=T_1 \sqcup T_2$ are two decompositions of a finite set $I$ with the following resulting pairwise intersections:
\[A:=S_1\cap T_1 \quad , \quad B:=S_1 \cap T_2 \quad , \quad C:=S_2 \cap T_1 \quad , \quad D:=S_2 \cap T_2.\]

The compatibility axiom in the definition of bimonoid can be reformulated as
\[x|_A \cdot y|_C= (x\cdot y)|_{T_1} \qquad , \qquad x/_A \cdot y/_C=(x \cdot y)/_{T_1},\]
for any disjoint sets $S_1, S_2$, for elements $x \in \rH[S_1]$, $y\in \rH[S_2]$ and for any set $T_1\subseteq S_1 \sqcup S_2$, by letting $A= S_1 \cap T_1$ and $C= S_2\cap T_1$.

\

A {\bf Hopf monoid} in $(\Ss, \cdot)$ $\rH$ is a bimonoid in set species such that the monoid $\rH[\emptyset]$ is a group.
Its antipode on $\emptyset $ is the map $s_\emptyset: \rH[\emptyset]\to \rH[\emptyset]$ given by $s_\emptyset(x):=x^{-1}$.
We define an antipode map $s: H\to H$ via the Takeuchi formula, adapted to set species.


\

\subsection{Fock functor}
In \cite{AM2010} (Part III), a construction is presented allowing to produce a (graded) Hopf algebra from a Hopf monoid. This is a categorical approach of a construction due to Stover (\cite{Stover}, Section 14), studied later by Patras, Schocker and Reutenauer in \cite{PR2004}, \cite{PS2006} and \cite{PS2008}.

\

We recall briefly this construction. Let $\mathbb{K}$ be a field of characteristic zero. If $\tp \in \Ssk$, then there is an action of the symmetric group $\mathfrak{S}_n$ on $\tp[n]$ by relabeling, for each $n \geq 0$. We denote by $\tp[n]_{\mathfrak{S}_n}$ the \emph{space of $\mathfrak{S}_n$-coinvariants of $\tp[n]$}:
\begin{equation}
  \tp[n]_{\mathfrak{S}_n}:=\tp[n]/\left\langle \, x- \tp[\alpha](x) \, | \, \alpha \in \mathfrak{S}_n, x\in \tp[n] \,\right\rangle.
\end{equation}

\


Consider $\gVec$ be the category of graded vector spaces over $\mathbb{K}$. The functors $\Kc, \Kcb: \Ssk \to \gVec$ given by 
\begin{equation}
    \Kc(\tp):=\bigoplus_{n \geq 0}\tp[n] \qquad , \qquad \Kcb(\tp):= \bigoplus_{n \geq 0}\tp[n]_{\mathfrak{S}_n}
\end{equation}
are referred in \cite{AM2010} as \emph{full Fock funtor} and \emph{bosonic Fock functor}, respectively. From any monoid (resp. comonoid, Hopf monoid) $\tp$, it is possible to obtain algebras (resp. coalgebras, Hopf algebras) $\Kc(\tp)$ and $\Kcb(\tp)$ from those of $\tp$, together with certain canonical transformations (see \cite{AM2010}, section 15.2).

\


\section{The pattern Hopf algebra \label{sec:pattern_algebra_contruction}}

\subsection{Species with restrictions}
The general setting for our approach to patterns is given by the notion of \emph{species with restrictions}, a terminology due to Schmitt (see \cite{Schmitt1993}) and used by the first author in \cite{Penaguiao2020}, where these were called combinatorial presheaves.

\

Let $\Fset_{\!\!\!\!\!\hookrightarrow}$ be the category of finite sets with injections as morphisms. A (set) {\bf species with restrictions} is a contravariant functor $\prR:\Fset_{\!\!\!\!\!\hookrightarrow} \to \Set$. Given a species with restrictions $\prR$ and a couple of finite sets $I,J$ such that $J \subseteq I$, the {\bf restriction map} $\text{res}_{I,J}:=\text{res}[\hookrightarrow]$ is the image under the functor $\prR$ of the inclusion $J \hookrightarrow I$:
\[\text{res}_{I,J}: \prR[I]\to \prR[J].\]

By functoriality, these maps satisfy the contravariant axioms
\begin{equation}\label{Axrestr}
    \text{res}_{J,K}\circ\text{res}_{I,J}=\text{res}_{I,K} \qquad , \qquad \text{res}_{I,I}=\text{id}_{\prR[I]},
\end{equation}
for any finite sets $I \supseteq J \supseteq K$. 
Since any arbitrary injection equals a bijection followed by an inclusion, any species with restrictions is equivalent to a set species together with restriction maps satisfying the axioms \eqref{Axrestr}.
Going forward, we use the notation $a|_J = \rest_{I, J}(a)$ for simplicity.

%REVIEWNov19
%added definition of restriction here

\

Species with restrictions form a category $\Spr$, where the arrows are natural transformations between functors.
For clarity, we denote species with restrictions with a typewriter typescript.
Note how a species with restrictions is also a set species.

Notice that, for any finite set $C$, the set species $\mathrm{1}_C$ is also a species with restrictions, where $\text{res}_{\emptyset, \emptyset} = \id_C$.

\

\subsection{Schmitt's comonoid}
In \cite{Schmitt1993} (Section 3), Schmitt gave a construction of coalgebras and bialgebras from certain species. We describe the coalgebra construction, following the notation of \cite{AM2010} (Section 8.7).

\

Given a species with restrictions $\prR$, we construct a linearized comonoid in $(\Ss, \cdot)$ as follows. Let $\trr=\mathbb{K}\prR$ be the linearisation of $\prR$. Given a decomposition $I=S \sqcup T$, consider the linear map
\[
\Delta_{S,T}: \trr[I]\to \trr[S] \otimes \trr[T]
\]
given by
\begin{equation}\label{CoprodRestr}
\Delta_{S,T}(x):=\text{res}_{I,S}(x)\otimes \text{res}_{I,T}(x),
\end{equation}

for any $x \in \prR[I]$. Let $\epsilon_\emptyset: \trr[\emptyset]\to \mathbb{K}$ be the linear extension of the map sending every element of $\prR[\emptyset]$ to $1$. Hence, we have the following result.

\begin{lm}[Schmitt]
The vector species $\trr$ is a linearized comonoid in $(\Ss, \cdot)$. In particular, the comonoid $\trr$ is cocommutative.
\end{lm}


Consider now the converse, a linearized comonoid $\tp=\mathbb{K}\rP$ in $(\Ss, \cdot)$. 
In this case, the coproduct gives a pure tensor
\[\Delta_{S.T}(x)=x|_S \otimes x/_S,\]
for each $x \in \rP[I]$ and for each decomposition $I = S \sqcup T$. We may then define restriction maps on $\tp$ either by

\begin{align*}
\text{res}^{(1)}_{I,J}: \tp[I] &\to \tp[J] \qquad \qquad  \text{or}  &\text{res}^{(2)}_{I,J}: \tp[I] &\to \tp[J],\\
x&\mapsto x|_J \qquad &x&\mapsto x/_{I\setminus J}
\end{align*} 
for $x \in \tp[I]$. Each restriction map $\text{res}^{(1)}$ or $\text{res}^{(2)}$ turns $\tp$ into a species with restrictions. When $\tp$ is cocommutative, then both restriction maps coincide. We have then the following characterisation of species with restrictions (see \cite{AM2010}, Proposition 8.29, for other characterisations).

\begin{thm}[Schmitt, Aguiar-Mahajan]\label{thm:swr_lcc}
There is an equivalence between the category of species with restrictions and the category of linearized cocommutative comonoids.
\end{thm}

\

\subsection{Monoids with restrictions}
We see now that the restriction structures are stable for the Cauchy product.
Let $\prP, \prQ$ be two species with restrictions.
Given two finite sets $I$ and $J$ with an inclusion $J \hookrightarrow I$, consider the map $\text{res}_{I,J}$ defined as the sum of the maps running over all decompositions $I=S\sqcup T$:

\[\xymatrix{
\prP[S]\times \prQ[T] \ar[rrr]^-{\text{res}_{S, S\cap J} \times \text{res}_{T, T\cap J}}&&& \prP[S\cap J]\times \prQ[T\cap J] \subseteq (\prP \cdot \prQ)[J],
}\]
where the first and second restrictions on the arrow above are the restrictions maps corresponding to $\prP$ and $\prQ$, respectively, from the inclusions $S \cap U \hookrightarrow S$ and $T \cap U \hookrightarrow T$, respectively. 
This defines restriction maps on $\prP \cdot \prQ$.
\

Therefore, the category of species with restrictions is a monoidal category $(\Spr, \cdot, \mathrm{1}_{\{\emptyset\}})$.

\

We describe monoids in the monoidal category $(\Spr, \cdot, \mathrm{1}_{\{\emptyset \} })$. 
A monoid $(\prP, \mu, 1)$ in species with restrictions is a monoid in set species such that for each $J \subseteq I=S\sqcup T$, the diagram
\[\xymatrix{
\rP[S]\times \rP[T]\ar[d]_-{\mu_{S,T}} \ar[rrr]^-{\text{res}_{S, S\cap J} \times \text{res}_{T, T\cap J}}&&&\rP[S \cap J]\times \rP[T\cap J]\ar[d]^-{\mu_{S \cap J, S\cap J}}\\
\rP[I]\ar[rrr]_-{\text{res}_{I,J}}&&& \rP[J]
}\]
commutes.

\

Given a monoid $\prP$ in the monoidal category of species with restrictions $(\Spr, \cdot)$, let $\tp:=\mathbb{K}\prP$ be the linearisation of the underlying set species of $\prP$. 
By \cref{thm:swr_lcc}, $\tp$ is a cocommutative comonoid. Since $\prP$ is a monoid, then $\tp$ is a monoid in the category of vector species. Moreover, the above diagram implies that the product and coproduct of $\prP$ are compatible, meaning that $\prP$ is a cocommutative bimonoid. 
This proves the following (originally from \cite{AM2010}:

\begin{thm}[Aguiar-Mahajan]\label{MonRestr1}
There is an equivalence between the category of monoids in $(\Spr, \cdot)$ and the category of linearized cocommutative bimonoids.
\end{thm}


\begin{thm}[Aguiar-Mahajan]\label{MonRestr2}
If $\prP$ is a connected monoid in species with restrictions, then $\mathbb{K}\prP$ is a Hopf monoid in vector species.
\end{thm}

\

\subsection{Pattern functions and the pattern Hopf algebra}


Given a species with restrictions $\prR$ and two finite sets $I$ and $J$, two objects $a\in \prR[I], b\in \prR[J]$ are said to be {\bf isomorphic objects}, or $a\sim b$, if there is a bijection $\sigma:I\to J$ such that $\prR[\sigma](b)=a$. 

\

The collection of equivalence classes of a species with restrictions $\prR$ is denoted by \begin{equation}
\mathcal{G}(\prR) := \bigcup_{n\geq 0 } \prR[n]_{\mathfrak{S}_n}.
\end{equation}
In this way, the set $\mathcal G(\prR) $ is the collection of all the $\prR$-objects up to isomorphism. It is straightforward to show that $\mathcal{G}(\prR)$ is a basis for the vector space $\Kcb(\prR)$.

\

%Recall that for every couple of finite set $I,J$ such that $J \subseteq I$, there is a restriction map 
%\[\text{res}_{I,J}: \prR[I] \to \prR[J]\]
%defined as the image by $\prR$ of the injection map $J \hookrightarrow I$. If $b \in \prR[I]$, we denote $b|_J$ for $\text{res}_{I,J}(b)$.
%REVIEWNov19 already defined

\begin{defin}[Pattern coefficients]\label{defin:patterncoeff}
Let $\prR$ be a species with restrictions. Given two finite sets $I,J$ such that $J \subseteq I$ and two objects $a\in \prR[I], b\in \prR[J]$,
we say that the subset $J \subseteq I$ is a {\bf pattern} of $b$ in $a$ if $a|_{J} \sim b$. More precisely, $J \subseteq I$ is a pattern of $b$ in $a$ if there exists a bijection $\sigma: J \to J$ such that
\[\prR[\sigma](b)=\text{res}_{I,J}(a).\]

We define the {\bf pattern coefficient} of $b$ in $a$ as
\begin{equation}
    \binom{a}{b}_{\!\prR} : = \left| \{J \subseteq I \, : \, a|_J \sim b \} \right| \, .
\end{equation}
\end{defin}

\

This definition only depends on the isomorphism classes of $b \in \prR[J]$ and $a \in \prR[I]$; see \cite{Penaguiao2020}. This motivates the following notion.

\begin{defin}[Patterns functions]\label{defin:pattern}
Let $\prR$ be a species with restrictions. Given a finite set $I$, we define the {\bf pattern function associated to} $b \in \prR[I]$ as the function
\[\pat_b: \mathcal{G}(\prR) \to \mathbb{K} \]
given by
\begin{equation}
    a \mapsto \binom{a}{b}_{\!\prR},
\end{equation}
for all $a \in \mathcal{G}(\prR)$.
\end{defin}

By definition $\pat_b \in \mathcal{F}(\mathcal{G}(\prR), \mathbb{K})$, where $\mathcal{F}(\mathcal{G}(\prR), \mathbb{K})$ denotes the set of functions from $\mathcal{G}(\prR)$ to $\mathbb{K}$. Without loss of generality, we denote by $\pat_b$ the linear extension to $\mathcal{A}(\mathtt{\prR})$ of the pattern function associated to $b$. Hence, we can consider $\{ \pat_b \}_{b\in \mathcal{G}(\prR)}$ as a family of linear functions from $\Kcb(\prR)$ to $\mathbb{K}$, indexed by $\mathcal{G}(\prR)$.
%In \cref{sec:speciespermutation}, we see an example of a species with restrictions structure on permutations.

\

\begin{defin}[Pattern spaces]
If $\prR$ is a species with restrictions, then the linear span of the pattern functions is denoted by
\begin{equation}
    \mathcal{A}(\prR):=\mathbb{K}\{\pat_a \, : \, a\in \mathcal{G}(\prR)\}.
\end{equation}
\end{defin}

We write $\mathcal{A}(\prR)$ for the {\bf pattern space} associated to the species $\prR$. By definition, $\mathcal{A}(\prR)$ is a linear subspace of the space of linear functions $\Kcb(\prR)^*$ from $\Kcb(\prR)$ to $\mathbb{K}$. The following was proven in \cite{Penaguiao2020}.

\begin{thm}
The subspace $\mathcal{A}(\prR)$ of $\Kcb(\prR)^*$ is closed under pointwise multiplication and has a unit.
It forms an algebra, called the {\bf pattern algebra associated to} $\prR$.
More precisely, if $a, b \in \mathcal G(\prR)$,
\begin{equation}\label{eq:prodrule}
\pat_ a   \pat_b = \sum_c \binom{c}{a, b}_{\! \prR} \pat_c \, ,
\end{equation}
where the coefficients $\binom{c}{a, b}_{\!\prR}$ are the number of ``quasi-shuffles'' of $a, b$ that result in $c$, specifically, if we take $c\in \prR[C]$ to be a representative of the equivalence class $c$, then:
$$ \binom{c}{a, b}_{\!\prR} = \left| \{(I, J) \, \text{ such that } \, \,  I \cup J = C \, ,\, \, c|_{I} \sim a, \, c|_{J} \sim b \} \right| \, .  $$
\end{thm} 

%For details on quasi-shuffles of combinatorial objects, the interested reader can see \cite{hoffman00,aguiar10,foissy16}.

Consider now a positive species with restrictions $\prR$ endowed with an associative product $(\prR, \sq)$. Examples of associative operations on species with restrictions are the direct sum of permutations $\oplus$, introduced above. Recall by Theorem \eqref{MonRestr2} that 
$(\prR, \sq)$ is equivalent to the linearized cocommutative Hopf monoid $(\trr, \sq, \Delta)$, where $\trr=\mathbb{K}\prR$ and $\Delta$ is defined as in \eqref{CoprodRestr}. If $*$ denotes the pointwise product in $\Kcb(\trr^*)$, then the space of pattern function $\mathcal{A}(\prR)$ is a subalgebra of $\Kcb(\trr^*, *, \Delta_{\sq})$, where $\Delta_{\sq}$ denotes the dual map of $\sq$.

%\raul{I don't see why we want a positive species here}

Under the natural identification of the function algebra $\mathcal{F}(\mathcal G (\prR), k)^{\otimes 2}$ as a subspace of $\mathcal{F}(\mathcal G (\prR) \times \mathcal G (\prR), k)$, we have
\begin{equation}\label{eq:coproddefin}
 \Delta_{\sq} \pat_a (b \otimes  c) =  \pat_a (b \,\sq\, c) \, .
\end{equation}
This is shown in \cref{thm:conHopfalgebra}. Therefore, we have the following coproduct in the pattern algebra $\mathcal A (\prR)$:
\begin{equation}\label{eq:coprodformula}
\Delta_{\sq} \pat_ a = \sum_{\substack{ b, c \, \in \, \mathcal G (\prR) \\ a = b \, \sq \, c}} \pat_b \otimes \pat_c \, .
\end{equation}
where the sum runs over coinvariants $b, c$ such that $a = b \,\sq \,c$. 
The relation \eqref{eq:coproddefin} is central in establishing that the coproduct $\Delta_{\sq} $ is compatible with the product in $\mathcal A (\prR)$.


\begin{thm}\label{thm:conHopfalgebra}
Let $(\prR, \sq, 1) $ be an associative species with restrictions.
Then the pattern algebra of $\prR$ together with the coproduct $\Delta_{\sq}$, and a natural choice of counit, forms a bialgebra.
If additionally $| \prR[\emptyset ] | = 1 $, the pattern algebra forms a filtered Hopf algebra.
\end{thm}

%Presheaves that satisfy $|h[\emptyset ]| = 1$ are called \textit{connected}.
%Connected algebraic structures are a classical resource in graded Hopf algebras, as in this way we can find an antipode through the so called \textit{Takeuchi formula}, introduced in \cite{takeuchi71}.

Some known Hopf algebras can be constructed as the pattern algebra of a species with restrictions.
An example is $Sym$, the Hopf algebra of \textit{symmetric functions}.
This Hopf algebra has a basis indexed by partitions, and corresponds to the pattern Hopf algebra of the species on set partitions.
%The pattern Hopf algebra corresponding to the presheaf on permutations described above was introduced by Vargas in \cite{vargas14}.
%Some other Hopf algebras constructed here, like the ones on graphs and on marked permutations below, are new, and some we conjecture are isomorphic to known Hopf algebras like the pattern Hopf algebra on set compositions, which may be simply the Hopf algebra of quasi-symmetric functions; see \cref{conj:QSym} below.

\

We end this section with a relevant theorem on functors of species with restrictions.

\begin{thm}[Theorem 3.8. in \cite{Penaguiao2020}]\label{thm:functoriality}
    If $\mathtt{f} : \prR \Rightarrow \mathtt{S}$ is a morphism of associative species with restrictions, between connected species, then $\mathcal A(\mathtt{f})$ is a Hopf algebra morphism that maps $\mathcal A(\mathtt{S}) \to \mathcal A(\mathtt{R})$.
\end{thm}


\section{Examples of species with restrictions \label{sec:species_restrictions}}

\subsection{Species on total orders.\label{sec:specieslinearorders}}

The first relevant species with restrictions to define is the species of total orders $\mathtt{L}$.
This has $\mathtt{L}[I] = \{\text{ total orders on $I$ }\}$, a set with $|I|!$ total orders.
The restriction of an order $\leq$ of $\mathtt{L}[I]$ to a subset $J$ is the induced order, which is a total order.
This defines a species with restrictions.

\

We also define an associative product on $\mathtt{L}$.
If $\mathsf{p}$ is a total order on $I$ and $\mathsf{r}$ is a total order on $J$, and if $I, J$ are disjoint, then we can define a total order $\mathsf{p} \ast\mathsf{r}$ a total order on $I \sqcup J$ as:

\begin{itemize}
\item if $a, b \in I$ such that $a \, \mathsf{p} \, b$, then $a \, (\mathsf{p} \ast \mathsf{r}) \, b$.

\item if $a, b \in J$ such that $a \, \mathsf{r} \, b$, then $a \, (\mathsf{p} \ast \mathsf{r}) \, b$.

\item if $a \in I$ and $b \in J$, then $a \, (\mathsf{p} \ast \mathsf{r}) \, b$.
\end{itemize}
This definition is independent of the choice of labeling and respects restriction to subsets. In particular, the product $\ast$ on orders depends only on the order structures, not on the specific labels chosen, and it respects restrictions to subsets, so this builds a connected associative species with restrictions. 

\

If $\mathsf{p} \in \mathtt{L}[I]$, we write $\mathbb{X}(\mathsf{p}) = I$.
The following fact is an immediate observation.

\begin{prop}\label{prop:linorderideals}
If $\mathbb{X}(\mathsf{p}) = I$, then $I$ is an order ideal in $\mathsf{p} \ast \mathsf{r}$.
That is, if $a\in I$ and $b (\mathsf{p} \ast \mathsf{r}) a$, then $b\in I$.
\end{prop}

\

\subsection{Species on permutations\label{sec:speciespermutation}}

To fit the framework of species with restrictions, we use a rather unusual definition of permutations introduced in \cite{albert2020two}.

This definition is motivated by the fundamental need to represent permutations as a combinatorial species with restrictions—that is, we need a definition where we can systematically understand how smaller permutation patterns fit into larger ones via restriction maps. Rather than viewing a permutation as a bijection (which doesn't naturally support a restriction structure because a bijection's "pattern" is not obviously defined as a sub-bijection), we represent it as a pair of total orders.

Specifically, a permutation on a set $I$ is seen as a pair of total orders $(\leq_P, \leq_V)$ on $I$, called the "position" and "value" orders respectively. Write $\mathbb{X}(\pi) = I$.

This relates to the usual notion of a permutation as a bijection in the following way: if we order the elements of $I = \{a_1 \leq_P \dots \leq_P a_k \} = \{ b_1 \leq_V \dots \leq_V b_k \}$, then the natural pairing $a_i \leftrightarrow b_i$ defines a bijection via the mapping $a_i \mapsto b_i$. Conversely, for any bijection $f$ on $I$, there exist multiple pairs of orders $(\leq_P, \leq_V)$ that correspond to the same bijection $f$; all of these pairs are isomorphic as permutations in the sense that they differ only by relabeling of the set $I$.

Crucially, the pair-of-orders representation allows restriction in a natural way: if $J \subseteq I$ is an injection, we can restrict both the position and value orders to $J$, yielding a permutation on $J$. This restriction structure is what enables permutations to form a species with restrictions.
The resulting species with restrictions structure is denoted by $\mathtt{Per}$.

\

It will be useful to represent permutations in $I$ as square diagrams labeled by $I$.
This is done in the following way: we place each element $a \in I$ at grid position $(i, j)$ where $i$ is the position of $a$ in the $\leq_P$ (position) order and $j$ is the position of $a$ in the $\leq_V$ (value) order. This visual representation directly encodes the bijection: the row index shows where an element comes from (position), and the column index shows where it goes (value).
We always start counting from the bottom left, following the Cartesian coordinates style.

For instance, the permutation $\pi = \{1<_P2<_P3 , 2<_V1<_V3\}$ in $\{1, 2, 3\}$ can be represented as 
\begin{equation}
\begin{array}{|c|c|c|}
	\hline & & 3 \\
    \hline 1 & &  \\
    \hline & 2 & \\
    \hline 
\end{array}\, \, \, .
\end{equation}

In this way, there are $(n!)^2 $ elements in $\mathtt{Per}[n]$.
Up to relabelling, we can represent a permutation as a diagram with one dot in each column and row.
Thus, $\mathtt{Per}[n]/_{\mathfrak S_n}$ (the set of isomorphism classes), has $n!$ isomorphism classes of permutations of size $n$, as expected.
\

%\label{defin:per}
%If we consider a permutation $\pi$ on a set $I$, that is, a pair $(\leq_P, \leq_V) $ of total orders in $I$, we write $\mathbb{X}(\pi) = I$.
%If $f:J \to I $ is an injective map, the preimage of each order $\leq_P, \leq_V$ is well defined and is also a total order in $J$.
%This defines the permutation $\mathtt{Per}[f](\pi )$.
%If $f$ is the inclusion map, this notion recovers the usual concept of a permutation pattern already present in the literature.

\

The $\ast $ operation on orders can be extended to a \textbf{diagonal sum} operation $\oplus $ on permutations, which places all elements of the first permutation before all elements of the second permutation in both the position and value orders.
This is usually referred as the shifted concatenation of permutations.
This endows $\mathtt{Per}$ with an  associative species with restrictions structure.



\subsection{Species on packed words}


For packed words, we will mimic the framework produced above for permutations.
First, recall that a linear partial order $\leq$ on a set $I$ is the pullback order of a surjective map $I \to [m]$, called the \textbf{rank function} of $\leq$, or $rk_{\leq}$.
Equivalently, it is an order $\leq$ where for any two distinct elements $a, b\in I$ such that $a \leq b$, there is a disjoint partition $I = A \uplus B$ such that all elements $x$ in $A$ and all elements $y$ in $B$ satisfy $x \leq y$.


We call the integer $m$ the rank of the order $\leq$.

\begin{figure}[h]
	\centering
	\includegraphics[scale=1]{../images/packedWordOrder.pdf}
	\caption{\textbf{Left:} The description of the function $rk_{\leq}$. \textbf{Right:} The Hasse diagram of the linear order inherited by $f$. \label{fig:packedWordOrder}}
\end{figure}

For instance, if $f = \{ a\mapsto 3, b\mapsto 1, c\mapsto 3, d\mapsto 2\}$ is a surjective map $\{a, b, c, d\}\to[3]$, the pullback order is $\{b < d < \{a, c \}\}$ and has rank three.
Its Hasse diagram is presented in \cref{fig:packedWordOrder}.
In this way, a packed word $\omega$ on $I$ is a pair of orders $(\leq_P, \leq_V)$ where $\leq_P$ is a total order on $I$, and $\leq_V$ is a linear partial order on $I$. This generalizes permutations by allowing ties (incomparabilities) in the value order, represented as ties in the vertical direction of the visual representation.
In particular, note that any permutation on $I$ can be seen as a packed word on $I$, as any total order is a partial linear order.
This relates to the usual notion of packed words (a word in $[m]$ where $m \leq |I|$ represents the rank) in the following way:
if we order the elements of $I = \{a_1 \leq_P \dots \leq_P a_k \}$, then the corresponding packed word is 
$$rk_{\leq_V}(a_1)rk_{\leq_V}(a_2) \dots rk_{\leq_V}(a_k) \, .$$

Conversely, for any packed word $\omega = p_1\dots p_k$, there are several pairs of orders $(\leq_P, \leq_V)$ that correspond to the word $\omega $, all of which are isomorphic.
Specifically, we can fix any total order $\leq_P$, and construct $\leq_V$ from $\omega $ and $\leq_P$: if the $i$-th entry of $\omega$ is $\omega(i)$ and the $i$-th element according to $\leq_P$ is $a_i$, then we set $rk_{\leq_V}(a_i) = \omega(i)$.

\

Consider for instance the packed words $\omega_1 = 13123$ and $\omega_2 = 32413$.
These correspond to packed words on $\{a, b, c, d, e\}$.
For the examples given $\omega_1$ corresponds to $(d \leq_P e \leq_P c \leq_P a \leq_P b, \{d, c\} \leq_V a \leq_V \{b, e\} )$ and $\omega_2$ corresponds to $(a \leq_P b \leq_P c \leq_P d \leq_P e, d \leq_V b \leq_V \{a, e\} \leq_V c )$.


If $I \hookrightarrow J$, by restricting the orders on $I$ to orders on $J$, we obtain a restriction to a packed word on $J$.
The resulting species with restrictions structure is denoted by $\mathtt{PWo}$.

\

It will be useful to represent packed words $\omega $ in $I$ as rectangle diagrams labeled by $I$.
This is done in the following way: let $1\leq m \leq |I|$ be the rank of $\omega$, we place the elements of $I$ in an $m \times |I|$ grid so that the elements are placed horizontally according to the $\leq_P$ order, and vertically according to the $\leq_V$ order.
For instance, the packed word $\omega_1 = 13123 = ( d < e < c < a < b, \{d, c\}) < \{a\} < \{b, e\}$ in $\{a, b, c, d, e\}$ can be represented as 
\begin{equation}
\begin{array}{|c|c|c|c|c|}
	\hline   & e &   &   & b \\
    \hline   &   &   & a &   \\
    \hline d &   & c &   &   \\
    \hline 
\end{array}\, \, \, .
\end{equation}




In this way, there are $(n!) \times \sum_{m = 1}^n \sFunc(I, [m])$ elements in $\mathtt{PWo}[n]$, where $\sFunc(I, [m])$ denotes the number of surjections (surjective functions) from $I$ to $[m]$.
Because each packed word $\omega $ has an isomorphism class of size $n!$, there are $ \sum_{m = 1}^n \sFunc(I, [m])$ many packed words, which is expected.

\

\label{defin:pwo}
If we consider a packed word $\omega = (\leq_P, \leq_V) $ on a set $I$, we write $\mathbb{X}(\omega) = I$.
If $f:J \to I $ is an injective map, the preimage of each order $\leq_P, \leq_V$ is well defined.
Furthermore, the preimage of $\leq_P$ is a total order on $J$, whereas the preimage of $\leq_V$ is a linear order on $J$.
Thus, this defines the packed word $\mathtt{PWo}[f](\omega )$.
The $\ast $ operation on orders can be extended to a \textbf{diagonal sum} operation $\oplus $ on packed words.
This is usually referred to the shifted concatenation of packed words.
This endows $\mathtt{PWo}$ with an  associative species with restrictions structure.

\

\subsection{Relation between parking functions and labelled Dyck paths}

Before we make the goal of this section explicit, let us first recall the definition of a parking function and of a Dyck path.

\begin{defin}[Parking function]
A parking function $\mathfrak{p} = a_1 \dots a_n$ is a sequence of integers in $[n]$ such that, after reordering $a^{(1)} \leq a^{(2)} \leq \dots \leq a^{(n)}$, we have $a^{(i)} \leq i$ for all $i$.
\end{defin}

Intuitively, parking functions model the following scenario: $n$ cars arrive at a street with $n$ parking spots labeled $1, \ldots, n$. Each car $i$ has a preferred spot $a_i$. Cars park greedily: car $i$ parks at its preferred spot if available, otherwise it parks at the next available spot to the right. A sequence is a parking function if and only if all cars can find a parking spot using this greedy algorithm.

Examples of parking functions are $12$, $131$ and $3114$.

\begin{defin}[Dyck path]
Given an $n\times n$ square grid, a \textbf{Dyck path} is an edge path from $(0,0)$ to $(n,n)$ staying weakly above the line $y = x$ (the main diagonal).
It is a classical result that Dyck paths are enumerated by Catalan numbers.
\end{defin}


To describe species on parking functions we need to use the construction of parking functions as labelled Dyck paths.
% (see for instance \cite{Loehr}).
%This notion of species with restrictions will recover a notion of patterns in parking functions studied in \cite{adeniran2022pattern}.
Specifically, if $I$ is a set of size $n$, we label a Dyck path $\mathcal D$ on an $n\times n$ square grid by a function $f$ that assigns unique values on the set $I$ to each of the \textbf{up} segments of the Dyck path. An 'up' segment is an edge from $(i, i)$ to $(i+1, i+1)$; a 'down' segment is an edge from $(i, i+1)$ to $(i+1, i)$.
If we enrich $I$ with a total order $\leq$ in such a way that in each sequence of \textbf{up} segments, the labels arise in \textbf{increasing} order, Bergeron et al.\ construct in \cite{BGLPV2021} a parking function $\mathfrak{p} = \mathfrak{p}(I, \mathcal D, f, \leq)$.
We recover here such construction, adapted to the language in this article, for convenience.

There are two steps in this construction.
The first is to realize the labelled Dyck path as a weak set composition $\opi$ of $I$ in $|I|$ parts. The second is to translate the weak set composition $\opi$ and the order $\leq $ in $I$ as a parking function.
We will use an example on $I = \{1, 2, 3, 4, 5\}$ to help us highlight the most important details of this construction, presented in \cref{fig:construction_parking}.
There, we have a Dyck path $\mathcal D$ and a corresponding assignment $f$ to each \textbf{up} segment.
We assume the usual integer order on $I$.

\

\textbf{In the first step} we group each of the labels that occur on the $i$-th vertical line in the $i$-th block of $\opi$.
This gives us a weak set composition of $I$ with exactly $|I|$ parts.
Notice that some parts may be empty sets, as it happens in the example given in \cref{fig:construction_parking}.

\textbf{In the second step} we read off the position of each of the labels of $I$, writing down in which part these occur. The resulting sequence is a parking function of size $|I|$.
In the example in \cref{fig:construction_parking}, for instance, we notice that there are no entries in the second block, so $2$ does not occur in the corresponding parking function.
Because there are three elements in the first block, namely $245$, in the parking function $\mathfrak p $ the character $1$ arises three times, on the second, fourth and fifth position.

\begin{figure}
    \centering
    \includegraphics{../images/correspondence_parking.pdf}
    \caption{This is an example of a Dyck path labelled by the set $I =\{1, 2, 3, 4, 5\}$. We enrich this set with the usual order on the integers.}
    \label{fig:construction_parking}
\end{figure}


\subsection{Species on parking functions}

The species of \textbf{parking functions}, $\mathtt{PF}$, arises by letting $\mathtt{PF}[I]$ be the collection of all parking functions $\mathfrak{p} = \mathfrak{p}(I, \mathcal D, f, \leq)$, that is Dyck paths $\mathcal D$ in an $|I|\times |I|$ grid, labelled by elements of $I$ along with an order $\leq$ of $I$.
We can give it a notion of species with restrictions: for each inclusion $\iota : I \hookrightarrow J $ and a labelled Dyck path $(J, \mathcal D, f, \leq)$ on $J$, the restriction $\mathtt{PF}[\iota](J, \mathcal D, f, \leq) = (I, \mathcal D|_I, f|_I, \leq|_I)$ has an intuitive meaning, except perhaps for
$\mathcal D|_I$, which we clarify in the following.
This is done via a notion of \textbf{tunnels} --- pairs of an up and down segment at the same level, that have no other segment at that level in between --- introduced by Deutsch and Elizalde in \cite{elizalde2003simple}.
Specifically, the corresponding Dyck path is defined to be the restricted Dyck path by taking the \textbf{tunnels} labelled by elements in $I$, relabelling sequences of \textbf{up} segments if necessary to preserve the increasing property.
One can see in \cref{fig:restriction_parking} how this works on the example given above, as well as the corresponding parking function.


\begin{figure}[h]
\centering
    \subfloat[\centering The parking function 31411 and its restriction]{{\includegraphics[height=4.8cm]{../images/restrictions_parking.pdf} }}%
    \qquad
    \subfloat[\centering The parking function 41511 and its restriction]{{\includegraphics[height=4.8cm]{../images/restrictions_parking_other.pdf}}}%
    \caption{\label{fig:restriction_parking}}%
\end{figure}

\

One defines a shifted concatenation $\oplus$ on parking functions, defined via concatenation of the underlying Dyck paths and orders (similar to the diagonal sum operation for permutations).
The following claim can be established by the same methods presented in \cite{Penaguiao2020}.

\begin{prop}[Species with restrictions on parking functions]
$\mathtt{PF}$ forms a species with restrictions structure. The proof follows the same approach as for permutations: the operations on Dyck paths and orders respect the restriction to subsets.
Furthermore, the shifted concatenation $\oplus $ endows $\mathtt{PF}$ with a monoid structure, and the resulting Hopf algebra $\mathcal A(\mathtt{PF})$ is free.
\end{prop}

For the last part, we observe that because $\mathtt{PF}$ is NCF (see below in \cref{defin:ncf}), we have from \cref{cor:freeNCF} that this algebra is free.

\begin{smpl}
The five smallest parking functions are $\emptyset, 1, 11, 12$ and $21$, where the empty sequence $\emptyset$ is the unique parking function of size 0.
The sixteen parking functions of size three are displayed in \cref{fig:PF3}, along with its corresponding labelled Dyck paths.

For any parking function $\mathfrak p$ of size three, we have $\pat_{\emptyset}(\mathfrak p) = 1$ and $\pat_1(\mathfrak p) = 3$.
The values of $\pat_{11}, \pat_{12}$ and $\pat_{21}$ in parking functions of size three are represented below in \cref{tab:PF3}.
There, one can also check the relation 
\[\pat_1^2 = \pat_1 + 2(\pat_{11} + \pat_{12} + \pat_{21}),\] 
predicted from \eqref{eq:prodrule}.
\end{smpl}
\begin{table}
\begin{tabular}{ c |  c c c c c c c c c c c c c c c c}
 $\pat$ & \small{111} & \small{112} & \small{121} & \small{211} & \small{113} &
\small{131} & \small{311} & \small{122} & \small{212} & \small{221} & \small{123} & \small{132} & \small{213} & \small{312} & \small{231} & \small{321}\\ 
 \hline 
 11 & 3 & 2 & 2 & 2 & 1 & 1 & 1 & 1 & 1 & 1 & 0 & 0 & 0 & 0 & 0 & 0 \\  
 12 & 0 & 1 & 0 & 0 & 2 & 1 & 0 & 2 & 1 & 0 & 3 & 2 & 2 & 1 & 1 & 0 \\  
 21 & 0 & 0 & 1 & 1 & 0 & 1 & 2 & 0 & 1 & 2 & 0 & 1 & 1 & 2 & 2 & 3 \end{tabular}
\caption{\label{tab:PF3}Pattern functions evaluated at parking functions of length three.}
\end{table}

\begin{figure}
\centering
\includegraphics[scale=1]{../images/parking_functions_3}
\caption{\label{fig:PF3}}
\end{figure}





\


\section{The antipode formula for pattern algebras \label{sec:formula_general}}
\

In this section we give a general formula for the antipode of a pattern algebra, whenever our connected species with restrictions is of the form $\mathtt{L} \times -$. 
This antipode formula is a dual analogue of the antipode formula in \cite{BergeronBenedetti}, where a similar formula was obtained for linearized Hopf species.
We note explicitly that species on parking functions and packed words, as well as $\mathtt{Per}$, are of the form $\mathtt{L} \times -$.

\

The formula obtained is not cancellation free, but it serves as a starting platform to explore the cancellation free formulas for the cases presented above: permutations, packed words and parking functions. The requirement that our species with restrictions is of the form $\mathtt{L} \times -$ explains why no cancellation free formulas for other species with restrictions, for instance in marked permutations, introduced in \cite{Penaguiao2020}, were found.

\

We start by recalling Takeuchi's formula, from above in \cref{lm:takeuchi}.
If $H$ is a Hopf algebra such that $(  \iota  \circ\epsilon - \id_H)$ is $\star$-nilpotent, then 
$$S = \sum_{k\geq 0 }  ( \iota  \circ\epsilon- \id_H)^{\star k}\, . $$

Recall that for any pattern Hopf algebra $\mathcal A (\mathtt{h})$, $(\iota\circ \epsilon - \id_H)$ is $\star$-nilpotent. 

\

\subsection{$\mathtt{L}\times -$ species with restrictions}


In \cite[Corollary 4.4.]{Penaguiao2020} it was shown that in species with restrictions $\mathtt{h}$, any factorisation into $\ast$-indecomposibles is unique possibly up to order of the factors.
In some species with restrictions, we can also drop the ``up to order'' adjective, as there is exactly one factorisation into $\ast$-indecomposibles. We make this precise in the \textit{non-commuting factorisation} definition.
For clarity, we use $\star$ for the operation on $\End (H)$ and use $\ast $ for the associative structure on a species.



\begin{defin}[Non-Commuting factorisation on pattern Hopf algebras]\label{defin:ncf}
A monoidal species with restrictions is called a \textbf{non-commuting factorisation} species, or simply an NCF species, if any element $x$ has a unique factorisation into $\ast$-indecomposibles elements $x = x_1 \ast \dots \ast x_n$.
\end{defin}


\begin{lm}[Linear species with restrictions have NCF]
Let $\mathtt{S}$ be a connected species with restrictions.
Then $\mathtt{L} \times \mathtt{S}$ has NCF.
\end{lm}

In the following, we will refer to associative and connected species of the form $\mathtt{L} \times \mathtt{S}$ as \textbf{ordered species with restrictions}.

\begin{proof}
Let $x = x_1 \ast \dots \ast x_k $ and $ y =  y_1 \ast \dots \ast y_n$ such that $x \sim y$.
From \cite[Corollary 3.4.]{Penaguiao2020}, we know that $k = n$ and the multisets $\{x_i\}_{i=1}^k,  \{y_i\}_{i=1}^n$ are the same.
It remains to show that the factorisations order coincide and that we have $x_i \sim y_i$ for $i = 1, \dots , n$.
To that effect we act by induction, where $n = 1$ is trivial.

\

%For the induction step, let us consider the bijection $\sigma:[n] \to [n]$ such that $y_{\sigma(i)} \sim x_i$ given by \cite[Corollary 3.4.]{Penaguiao2020}.
We write $x_i = (l_i, p_i)\in (\mathtt{L}\times \mathtt{S})[I_i]$ and $y_i = (m_i, q_i)\in (\mathtt{L}\times \mathtt{S})[J_i]$.
Write $x = (l, p) \in (\mathtt{L}\times \mathtt{S})[I] $ and $y = (m, q) \in (\mathtt{L}\times \mathtt{S})[J]$.
Assume wlog that $|I_1 | \geq |J_1|$.

\

By hypothesis, we have that $x \sim y$, so there exists some bijection $\phi: J \to I$ such that $ (\mathtt{L}\times \mathtt{S})[\phi](x) = y$.
This bijection yields a correspondence between two linear orders $\mathtt{L} [\phi] (l_1 \ast \dots \ast l_n) = m_1\ast \dots \ast m_n $.
Observe that $I_1$ and $J_1$ are ideals in $l_1 \ast \dots \ast l_n$ and $m_1 \ast \dots \ast m_n$, respectively, as seen in \cref{prop:linorderideals}.
So either $J_1 = \phi (I_1)$ or $J_1 \subsetneq  \phi(I_1) $.
Assume for sake of contradiction that $J_1 \subsetneq  \phi(I_1)$, and consider the following factorisation of $x_1 $:
\begin{equation}\label{eq:linearpats:non_trivial_fact}
x_1 = x|_{I_1} \sim y|_{\phi(I_1)} = y_1|_{\phi(I_1) \cap J_1} \ast (y_2 \ast \dots \ast y_n)|_{(J\setminus J_1) \cap \phi(I_1)}\, . 
\end{equation}

From $|I_1| > |J_1|$ and $\phi $ is a bijection, we have that $| (J\setminus J_1) | + | \phi(I_1) | = |J| - |J_1| + |I_1| > |J|$, so the intersection $(J\setminus J_1) \cap \phi(I_1)$ is non-empty.
On the other hand, $\phi(I_1) \cap J_1 = J_1$ is also non-empty.

\

Going back to \eqref{eq:linearpats:non_trivial_fact}, we get a non-trivial factorisation of $x_1$, which contradicts the fact that we started with a factorisation into indecomposables. Thus $|I_1 | = |J_1|$, which shows that $\phi(I_1) = J_1$.
Therefore,  $\phi(I\setminus I_1) = J \setminus  J_1$, $ \mathtt{R}[\phi] (p|_{I_1}) = q|_{J_1} $ and $ \mathtt{R}[\phi] (p|_{I \setminus I_1}) = q|_{J \setminus J_1} $.
We get that $x_1 \sim y_1$ and $x_2 \ast \cdots \ast x_n \sim y_2 \ast \cdots \ast y_n$, via $\phi$.
By induction hypothesis, this tells us that $x_2 \sim y_2, \cdots , x_n \sim y_n$, as desired.
\end{proof}


\begin{cor}\label{cor:freeNCF}
If $\mathtt{R}$ is an ordered species with restrictions, then $\mathcal A(\mathtt{R})$ is free.
\end{cor}
This follows from \cite[Theorem 5.2]{Vargas}, as this proof only uses the fact that $\mathtt{Per}$ is NCF.

\

The pattern Hopf algebra on permutations, on packed words and on parking functions all satisfy the NCF property.
In fact, permutations, packed words and parking functions are of the form $\mathtt{L} \times \mathtt{R}$.
This property allows for an easy manipulation of the coproduct, and results in a tractable approach to the antipode formula.

\begin{defin}[Composition poset and cumulative sum]
Recall that we write $\mathcal C_n$ for the set of compositions of size $n$.
We define $\mathbf{CS}$, a bijection between $\mathcal C_n $ and $2^{[n-1]}$, as follows.
If $\alpha =(\alpha_1, \dots, \alpha_{\ell} ) \in \mathcal C_n$, define $f_i = \sum_{j=1}^i \alpha_j$ and 
\begin{equation}
    \mathbf{CS}(\alpha) \coloneqq \{f_i\, | \, i = 1, \dots, \ell - 1\} \, .
\end{equation}
This bijection allows us to define an order $\leq $ in $\mathcal C_n$, via the pullback from the boolean poset in $2^{[n-1]}$.
This order can also be defined as follows: we say that $\alpha \leq \beta$ if $\alpha$ arises from $\beta $ after merging and adding consecutive entries.
\end{defin}

\
Recall that a QSS $\III = (I_1, \dots , I_n)$ of $y\in \mathtt{R}[I]$ from $x_1, \dots, x_n$ satisfies $y|_{I_i} \sim x_i$ for all $i = 1, \dots , n$, and $I = \bigcup_i I_i$.

\begin{defin}[Compositions and QSS]
Consider again $\mathtt{R}$ an ordered species with restrictions.
Let $x \in \mathtt{R}[J], y \in  \mathtt{R}[I]$, and $x = x_1\ast \dots \ast x_n$ be the unique factorisation of $x$ into indecomposables.
Further say that $y = (\leq_y, \iota)$, where $\leq_y$ is a linear order in $I$.
Let $\III = (I_1, \dots , I_n)$ be a QSS of $y$ from $x_1, \dots, x_n$ and consider a composition $\alpha \models n$.

\

Suppose that $ \mathbf{CS} (\alpha) = \{f_1, \dots, f_{\ell(\alpha) - 1} \}$ and use the convention that $f_0 = 0$ and $f_{\ell(\alpha)} = n$. 
Then we define for $i = 1, \dots, \ell(\alpha)$:
\[I^{\alpha}_i := I_{f_{i-1} + 1} \cup \dots \cup I_{f_i} \quad , \quad x^{\alpha}_i := x_{f_{i-1} + 1} \ast \dots \ast x_{f_i}.\]

\

For a partial order $\leq$ on a set $I$, and two sets $A, B \subseteq I$, we say that $A \lneq B$ if $A, B$ are disjoint and $a \leq b$ for any $a \in A$ and $b \in B$.

\

Two indices $i< j$ are said to be \textbf{merged} by $\alpha$ if there is some $k$ in $\{1, \dots, n\}$ such that $ f_{k-1} < i < j \leq f_k$.
We say that a QSS $\III$ of $y$ from $x_1, \dots , x_n$ is $\alpha $-stable if $(I^{\alpha}_i)_i \text{ is a QSS of $y$ from } (x^{\alpha}_i)_i$ and, whenever $x_i \sim x_j$ and $i < j$ are merged with $\alpha $, then $I_i \lneq_y I_j$.

\

Finally, we define 
\begin{equation}
   \mathcal I^{x, y}_{\III} \coloneqq \left\{ \alpha \models n \,\Big| \,\III \text{ is an $\alpha$-stable QSS of $y$ from } (x_i)_i \right\} \, . 
\end{equation}
\end{defin}

\

\begin{smpl}[$\alpha$-stable QSS on $\mathtt{PWo}$]
Consider the packed word $\rho = 21 \oplus 111 = 21333 = (1 < 2 < 3 < 4 < 5, 2 < 1 < \{3, 4, 5\})$ on the set $[5]$.
This packed word has three QSS from $21, 1, 11$, precisely $\III_1 =(12, 3, 45)$, $\III_2 =(12, 4, 35)$ and $\III_3 =(12, 5, 34)$.
All three are $(1, 1, 1)$-stable.

\

We now observe that $\III_1, \III_2$ and $\III_3$ are $(2, 1)$-stable, but not $(1, 2)$-stable.
Indeed, $\III_1^{(1, 2)} = \III_2^{(1, 2)} = \III_3^{(1, 2)} = (12, 345)$, and $\rho|_{345} = 111$ is not $\rho|_{I_2} \oplus \rho|_{I_3}$ for any of the QSS.
On the other hand, $\III_1^{(2, 1)} = (123, 45)$, $\III_2^{(2, 1)} = (124, 35)$ and $\III_3^{(2, 1)} = (125, 34)$, in each case is easy to note that $\rho|_{I_1^{(2, 1)}} = 21\oplus 1$ and $\rho|_{I_2^{(2, 1)}} = 11$ for each of the QSS.

Therefore, in each case we can see that
$$\mathcal I_{\III}^{21\oplus 1\oplus 11, 21333} = \{(1, 1, 1), (2, 1)\}\, . $$
\end{smpl}


The following example portrays the consequences of the additional requirement that $I_i \lneq_y I_j$ whenever $x_i \sim x_j$ and $i, j$ are merged by $\alpha$, in the context of packed words.

\begin{smpl}[$\alpha$-stable on $\mathtt{PWo}$, with $x_i \sim x_j$]
Let us consider now the packed word $\rho = 2133 = (1 \leq_P 2 \leq_P 3 \leq_P 4, 2 \leq_V 1 \leq_V \{3, 4\} )$, we will be considering QSS of $\rho$ from $\omega_1 = 12, \omega_2 = 1, \omega_3 = 1$.
Namely, $\III_1 = (13, 2, 4)$ and $\III_2 = (13, 4, 2)$.
We have that $\rho|_{I_2 \cup I_3} = \rho|_{I_2} \oplus \rho|_{I_3} = 1 \oplus 1$ in either of the cases.
However, note that $\omega_2 \sim \omega_3$, and the composition $(1, 2)$ merges $2$ and $3$, so $(1, 2)$-stability requires $I_2 \lneq_P I_3$.
Indeed, because $2 \leq_P 4$, we have that $\III_1$ is $(1, 2)$-stable, whereas $\III_2$ is not $(1, 2)$-stable.
\end{smpl}


Define the composition $\mu_i \coloneqq (\underbrace{1, \dots , 1}_\text{$i-1$ times}, 2, 1, \dots, 1)$.
If $\III = (I_1, \dots, I_n)$ is a $\mu_i$-stable QSS, then $y|_{I_1 \cup I_{i+1}} = y|_{I_i} \ast y|_{I_{i+1}}$.
This motivates the following lemma:

\begin{lm}\label{obs:pwords-characterisation}
Let $\mathtt{R}$ be an ordered species with restrictions.
Consider $x, y$ objects in $\mathtt{R}$, such that $y = (\leq_y, m)$ and $x = x_1\ast \dots \ast x_n$ a factorisation into indecomposibles.
If $\III = (I_1, \dots, I_n)$ is a QSS of $y$ from $x_1, \dots x_n$ that is $\mu_i$-stable, then $I_i \lneq_y I_{i+1}$.
\end{lm}


We observe latter that in the case of packed words, a stronger claim can be used to compute $\alpha$-stability.
See \cref{lm:QSSpackedWords}.

\begin{proof}
If $x_i \sim x_{i+1}$, the $\mu_i$ stability implies that $I_i \lneq_y I_{i+1}$, because $\mu_i$ merges $i$ and $i+1$.
We can now assume that $x_i \not\sim x_{i+1}$.
Let $X_j = \mathbb{X}(x_j)$ for $j = i, i+1$.
Note that because $y|_{I_j} \sim x_j$, we have that $|X_j| = |I_j|$.
The stability condition further gives us that $y|_{I_i \cup I_{i+1}} \sim x_i \ast x_{i+1}$.
Let $\phi : X_i \sqcup X_{i+1} \to  I_i \cup I_{i+1}$ be bijection such that $\mathtt{R}[\phi](y|_{ I_i \cup I_{i+1}}) = x_i \ast x_{i+1}$.

\

Because $\phi$ is a bijection, and $|X_j| = |I_j|$, this means that $I_i, I_{i+1}$ are disjoint.
We will consider the case that $|I_i| \geq |I_{i+1}|$ here.
The proof on the $|I_i| \leq |I_{i+1}|$ case can be done in a similar way.

\

Let $\inc $ be the injection $I_i \to I_i \cup I_{i+1}$.
Observe that
\begin{equation}\label{eq:lmonPWo}
    \begin{split}
        \mathtt{h}[\phi \circ \inc](y|_{I_i \cup I_{i+1}}) =& (x_i \ast x_{i+1})|_{\phi^{-1}(I_i)}\\
        \mathtt{h}[\phi](y|_{I_i}) =& (x_i)|_{X_i \cap \phi^{-1}(I_i)} \ast (x_{i+1})|_{X_{i+1} \cap \phi^{-1}(I_i)}\, .
    \end{split}
\end{equation}

However, $\mathtt{R}[\phi](y|_{I_i}) \sim y|_{I_i} \sim x_i$, which is $\ast$-indecomposible, we conclude that either $|X_i \cap \phi^{-1}(I_i)| = 0 $ or $|X_{i+1} \cap \phi^{-1}(I_i)| = 0$.
Assume the first for sake of contradiction, and because $\phi^{-1}(I_i) \subseteq X_i \cap X_{i+1}$, we have that $\phi^{-1}(I_i) \subseteq X_{i+1}$.
But $|\phi^{-1}(I_i)| = |I_i| \geq |I_{i+1}| = |X_{i+1}|$, therefore we have equality, that is $\phi^{-1}(I_i) = X_{i+1}$

\

This together with \eqref{eq:lmonPWo} and connectedness of $\mathtt{R}$ yields $\mathtt{h}[\phi](y|_{I_i}) = x_{i+1}$, which implies $x_i \sim x_{i+1}$, a contradiction.
We conclude that $|X_{i+1} \cap \phi^{-1}(I_i)| = 0$, and so $\phi^{-1}(I_i) = X_i$.
Because $I_i$ and $I_{i+1}$ are disjoint, it follows that $\phi^{-1}(I_{i+1}) = X_{i+1}$.

\

If we write $y = (\leq_y, m)$ and $x_i \ast x_{i+1} = (\leq_x, n)$, then $X_i \lneq_x X_{i+1}$ from \cref{prop:linorderideals}.
Because $\mathtt{R}[\phi^{-1}](x_i \ast x_{i+1}) = y|_{I_i \cup I_{i+1}}$, this gives us $\phi(X_i) \lneq_y \phi(X_{i+1})$, as desired.
\end{proof}

\


\begin{obs}
Let $\mathtt{R}$ be a species with restrictions, $y, x_1, \dots, x_n \in \mathcal G(\mathtt{R})$ and $\III$ a QSS of $y$ from $x_1\dots, x_n$.
Call $x \coloneqq x_1 \ast \dots \ast x_n$.
Then, $\mathcal I^{x, y}_{\III}$ has a unique maximal element, $\mathbb{1} = (1, \dots, 1)$.
\end{obs}

The following is the main theorem in this section

\begin{thm}\label{thm:general_antipode}
For a species with restrictions $\mathtt{R}$ that is a multiple of $\mathtt{L}$, and an element $x$, along with its factorisation $x=x_1 \ast \dots \ast x_n$, we have the following antipode formula in $\mathcal A (\mathtt{R})$:

$$S(\pat_x) = \sum_y \pat_y \sum_{\substack{\III \text{ QSS of $y$}\\ \text{from }x_1, \dots , x_n}}  \sum_{\alpha \in \mathcal I^{x, y}_{\III}} (-1)^{\ell ( \alpha)} \, .$$
\end{thm}

\begin{proof}
\begin{align*}
\Delta^{\circ (k-1)} (\pat_x)
    =& \sum_{x = \chi_1 \ast \dots \ast \chi_k} \pat_{\chi_1}\otimes \dots \otimes \pat_{\chi_k}
\end{align*}
Because the species with restrictions $\mathtt{R}$ has NCF, the only ways to factorize $x$ into $k$ factors is to start from the original factorisation $x = x_1 \ast \dots \ast x_n$ and bracket these factors into $k$ blocks, possibly empty, of $n$.
Therefore, we can enumerate these factorisations using weak compositions, as follows:

\begin{align*}
(\iota \circ \epsilon -  \id_{\mathcal A (\mathtt{R})})^{\star k} (\pat_x)
    =& (-1)^k  \mu^{\circ(k-1)}( \id_{\mathcal A (\mathtt{R})} - \iota \circ \epsilon)^{\otimes k} \Delta^{\circ(k-1)} (\pat_x)\\
    =& (-1)^k  \mu^{\circ(k-1)}( \id_{\mathcal A (\mathtt{R})} - \iota \circ \epsilon)^{\otimes k}\left(\sum_{\substack{\alpha\models^0 n \\ \ell(\alpha) = k}} \pat_{x^{\alpha}_1} \otimes \dots \otimes \pat_{x^{\alpha}_k} \right) \\
    =&  (-1)^k  \mu^{\circ(k-1)} \sum_{\substack{\alpha\models n \\ \ell(\alpha) = k}}  \pat_{x^{\alpha}_1} \otimes \dots \otimes \pat_{x^{\alpha}_k}\\
    =&  (-1)^k \sum_{\substack{\alpha\models n \\ \ell(\alpha) = k}}  \pat_{x^{\alpha}_1}  \cdots \pat_{x^{\alpha}_k}
\end{align*}

Note how we used that $(\id_{\mathcal A (\mathtt{R})} - \iota \circ \epsilon ) (\pat_x) = \mathbb{1}[x \neq 1]\pat_x$.
Takeuchi's formula gives:
%Thus, the theorem follows from Takeuchi's formula (\cref{lm:takeuchi}) after some clever rearrangements:

\begin{align*}
S(\pat_x)&= \sum_{k\geq 0 } (\iota \circ \epsilon -  \id_{\mathcal A (\mathtt{R})})^{\star k} (\pat_x) \\
         &= \sum_{\alpha \models n} (-1)^{\ell(\alpha)} \pat_{x^{\alpha}_1} \cdots \pat_{x^{\alpha}_{\ell(\alpha)}}\\
         &= \sum_{\alpha \models n } (-1)^{\ell(\alpha)} \sum_y \pat_y \binom{y}{x^{\alpha}_1,  \dots , x^{\alpha}_{\ell(\alpha)}}\\
         &= \sum_y \pat_y \sum_{\alpha \models n} \sum_{  \substack{\III \text{ QSS of $y$}\\ \text{from }x^{\alpha}_1,  \dots , x^{\alpha}_{\ell(\alpha)} }} (-1)^{\ell(\alpha)}\\
         &= \sum_y \pat_y \sum_{  \substack{\III \text{ QSS of $y$}\\ \text{from }x_1 , \dots ,  x_n }} \sum_{\alpha \in \mathcal I^{x, y}_{\III}} (-1)^{\ell(\alpha)}.
\end{align*}

All equalities but the last one are simple rearrangements.
To motivate the last equality, we will construct a series of bijections, for each composition $\alpha \models n$:
\begin{align*}
    \Phi_{\alpha}^{x, y} = \Phi_{\alpha} : \left\{ \JJJ \text{ QSS of $y$ from $x^{\alpha}_1, \dots, x^{\alpha}_{\ell(\alpha)}$}\right\} &\to \left\{ \III \text{ $\alpha$-stable QSS of $y$ from $x_1, \dots, x_n$}\right\}\, , \\
    (J_1, \dots, J_{\ell(\alpha)} ) &\mapsto (I_1, \dots , I_n).
\end{align*}

\

Recall that $y = (\leq_y, \iota)$ induces a linear order $m$ in $I$.
The sets $(I_i)_i$ are defined so that $J_s = I_{f_{s-1}+1} \uplus \dots \uplus I_{f_s}$, that $|I_i | = |x_i|$ and that $I_i \lneq_y I_j $ whenever $f_{s-1} < i < j \leq f_{s}$.
There is clearly a unique way to choose such $(I_i)_i$.
This is easily seen to be an $\alpha$-stable QSS of $y$ from $x_1, \dots, x_n$.

\

Inversely, we define:
\begin{align*}
    \Psi_{\alpha}^{x, y} = \Psi_{\alpha}  : \left\{ \III \text{ $\alpha$-stable QSS of $y$ from $x_1, \dots, x_n$}\right\} &\to \left\{ \JJJ \text{ QSS of $y$ from $x^{\alpha}_1, \dots, x^{\alpha}_{\ell(\alpha)}$}\right\} \, , \\
    (I_1, \dots , I_n) &\mapsto (I_1^{\alpha}, \dots, I^{\alpha}_{\ell(\alpha)} ),
\end{align*}
where we recall that $I^{\alpha}_i = I_{f_{s-1}+1} \cup \dots  \cup I_{f_s}$.

\

The maps $\Psi_{\alpha}$ and $\Phi_{\alpha}$ are easily seen to be inverses of each other, establishing the bijection and the desired result.
\end{proof}

\

%\begin{smpl}[Example in permutation patterns]
%Consider $\sigma = 312$ and $\pi = 123 = 1\oplus 1 \oplus 1$.
%We will compute the antipode of $\pat_{\sigma}$ using this formula, by computing $\mathcal I^{\sigma, \pi}_{\III}$ for each QSS $\III$.
%
%\end{smpl}

\begin{prop}[Filtered structure of $\mathcal I$]\label{prop:filter_structure_I}
For an ordered species with restrictions $\mathtt{R}$, and elements $x\in  \mathtt{R}[J], y\in \mathtt{R}[I]$, along with a factorisation into $\ast$-indecomposibles $x = x_1 \ast \dots \ast x_n$ and a QSS $\III$ of $y$ from $x_1, \dots, x_n$, we have that $\mathcal I^{ x, y}_{\III}$ is a filter.
That is, if $\alpha \in \mathcal I^{ x, y}_{\III}$ and $\beta \geq \alpha$ then $\beta \in \mathcal I^{ x, y}_{\III}$.

\
Furthermore, if $\mathcal I^{ x, y}_{\III}$ has a unique minimal element distinct from $\mathbb{1}$, then
$$\sum_{\alpha \in \mathcal I^{x, y}_{\III}} (-1)^{\ell(\alpha)} = 0 \, .$$
\end{prop}

\begin{proof}
Say that $y = (\leq_y, \iota)$.
To establish the first fact, assume that $\alpha \in \mathcal I^{x, y}_{\III}$ and let $\beta \geq \alpha $.
Because $\beta \geq \alpha$, if $\beta $ merges $i < j$ then $\alpha $ merges $i < j$, thus we have that $I_i \lneq_y I_j$.
It remains to show that $(I^{\beta}_i)_i$ is a QSS of $y$ from $(x^{\beta}_i )_i$, or equivalent that $y|_{I_i^{\beta}} \sim x_i^{\beta}$.

\

However, because $\beta \geq \alpha$, for each $i$ we have $I^{\beta}_i \subseteq I^{\alpha}_j$ so 
\[y|_{I^{\beta}_i} = {\Big(}y|_{I^{\alpha}_j}{\Big )}|_{I^{\beta}_i} =x^{\alpha}_j |_{I^{\beta}_i} = x^{\beta}_i \, .\]

\

For the concluding part, we simply observe that if $\mathcal I^{x, y}_{\III}$ has a unique minimal element $\mathbb{0}$, then it is the interval of a boolean poset $\mathcal I^{x, y}_{\III} = [\mathbb{0}, \mathbb{1}]$.
%Furthermore, we have that $\mathbb{0} \neq \mathbb{1}$.
Let $K $ be the collection of sets $A \subseteq [n-1]$ that contain $X \coloneqq \mathbf{CS}(\mathbb{0})$.

\

Consider $\zeta: K \to K$.
Choose $\chi \in [n-1] \setminus X$, if $X \neq [n-1]$, and consider the following symmetric difference:
\[\zeta (A) = A \,\Delta\, \{ \chi \}\]

If no such $\chi $ exists, define $\zeta(A) = A$.
This is an involution on $K$, % subsets of $[n-1]$ that contain $X$, which 
which corresponds to a sign-reversing involution in compositions in $[\mathbb{0} , \mathbb{1}]$:
%It can be seen that this is indeed sign-reversing, so we conclude that 
$$\sum_{\alpha \in \mathcal I^{x, y}_{\III}} (-1)^{\ell(\alpha)} =  \mathbb{1}[[n-1] \setminus X \text{ is empty } ] = \mathbb{1}[\mathbb{1} = \mathbb{0}] \, .$$
This concludes the proof.
\end{proof}

%We use the bijection $\mathbf{CS}$ between compositions and subsets of $\{1, \dots , n-1\}$ to compute the following sum.
%Let $X \coloneqq \mathbf{CS}(\mathbb{0} )$, then we can consider $\zeta : \mathcal K \to \mathcal K$

%\begin{align*}
%\sum_{\alpha \in \mathcal I^{y, x}_{\III}} (-1)^{\ell(\alpha)} &= \sum_{X \subseteq Y \subseteq [n-1]} (-1)^{|Y| +1} = \sum_{k = |X|}^{n-1} \sum_{\substack{X \subseteq Y \subseteq [n-1]\\ |Y| = k}} (-1)^{k +1} \\
%&= \sum_{k = |X|}^{n-1}\binom{n-|X| - 1}{k - |X|} (-1)^{k +1} = \sum_{k = 0}^{n- |X|-1}\binom{n-|X| - 1}{k} (-1)^{k + |X| +1}  \\
%&= (1 -1)^{n - |X| - 1 + |X| +1} = 0 \, .\\
%\end{align*}

%Notice that because $X \neq [n-1]$, we can use the Newton identity with exponent $n-|X| - 1$.

\

The consequence is, if $\mathcal I_{\III}^{x, y}$ has a unique minimal elemet for any two objects $x, y$ and QSS $\III$, the antipode formula reduces to counting how many of these minimal elements are the composition $\mathbb{1}$.
Indeed, these will contribute with $(-1)^n$ to the total coefficient of $\pat_y$, giving us a cancellation-free formula.

\


\section{The antipode formula on special cases\label{sec:formula_pp}}
\subsection{The antipode formula for the pattern algebra on packed words\label{sec:formula_packed}}
%Let $\rho = (\leq_P, \leq_V)$ be a packed word.
For a partial order $\leq$ on a set $I$, and two sets $A, B \subseteq I$, recall that we say that $A \lneq B$ if $A\cap B = \emptyset$ and $a \leq b$ for any $a \in A$ and $b \in B$.
We recall the definition of non-interlacing QSS on packed words.


\begin{defin}[Interlacing QSS on packed words]
Let $\rho, \omega_1, \dots, \omega_n$ be packed words, where $\rho = (\leq_P, \leq_V)$ is a packed word on $I$.
Let $\III = (I_1, \dots, I_n)$ be a QSS of $\rho$ from $\omega_1, \dots, \omega_n$.
We say that this QSS is \textbf{non-interlacing} if there exists some $i = 1, \dots, n-1$ such that $I_i \lneq_P I_{i+1}$ and $I_i \lneq_V I_{i+1}$.
If no such $i$ exists, we say that the QSS is \textbf{interlacing}.

\

Additionally, let $ \bigl[\!\begin{smallmatrix} \rho  \\ \omega_1, \dots, \omega_n \end{smallmatrix}\!\bigr]$ be the number of interlacing QSS of $\rho$ from $\omega_1, \dots, \omega_n$.
\end{defin}

\

Our goal in this section is to analyse $\mathcal I^{\omega, \rho}_{\III}$, show that it always has a unique minimal element, and that it is precisely $\mathbb{1}$ whenever $\III$ is interlacing.
Recall that $\mu_i = (\underbrace{1, \dots , 1}_\text{$i-1$ times}, 2, 1, \dots, 1)$.

\

\begin{lm}\label{lm:QSSpackedWords}
Let $\omega, \rho = (\leq_P, \leq_V)$ be packed words, such that $\omega = \omega_1 \oplus \dots \oplus \omega_n$ is its factorisation into $\oplus$-indecomposibles.
Let $\III$ be a QSS of $\rho$ from $\omega_1, \dots , \omega_n$.
Then $\III$ is $\mu_i$-stable if and only if $I_i \lneq_P I_{i+1}$ and $I_i \lneq_V I_{i+1}$.
\end{lm}


\begin{proof}
Let us take care of the \textbf{forward direction} first.
From \cref{obs:pwords-characterisation}, we have that $I_i \lneq_P I_{i+1}$, therefore $I_i$ and $I_{i+1}$ are disjoint and $I_i$ is the unique ideal of $(\leq_P)|_{I_i \cup I_{i+1}}$ of size $|I_i|$.
We have that $\rho|_{I_i \cup I_{i+1}} \sim \omega_i \oplus \omega_{i+1}$, so the unique ideal of $(\leq_V)|_{I_i \cup I_{i+1}}$ of size $|I_i|$ is also an ideal of $(\leq_P)|_{I_i \cup I_{i+1}}$, so it has to be $I_i$.
We conclude that $I_i$ is an ideal of $(\leq_V)|_{I_i \cup I_{i+1}}$, so $I_i \lneq I_{i+1}$, concluding this part of the proof.

\

For the \textbf{backwards direction}, if $I_i \lneq_P I_{i+1}$ then $I_i$ and $I_{i+1}$ are disjoint.
Therefore, by the definition of $\ast $ in partial orders, 
$$((\leq_P)|_{I_i \cup I_{i+1}}) = ((\leq_P)|_{I_i} \ast (\leq_P)|_{I_{i+1}})\, . $$
Similarly for $\leq_V$.
So we conclude that 
$$\rho|_{I_i \cup I_{i+1}} = ((\leq_P)|_{I_i} \ast (\leq_P)|_{I_{i+1}}, (\leq_V)|_{I_i} \ast (\leq_V)|_{I_{i+1}}) = \rho|_{I_i} \oplus \rho|_{I_{i+1}}\, .  $$
If $\omega_i \sim \omega_{i+1}$, we also have that $I_i \lneq_P I_{i+1}$, so $\III$ is $\mu_i$-stable.
\end{proof}

\

\begin{lm}\label{lm:minpacked}
Let $\omega$ be  packed word, along with $\omega = \omega_1 \oplus \dots \oplus \omega_n$, its unique factorisation into $\oplus$-indecomposible packed words.
Let $\rho$ be another packed word, and $\III$ a QSS of $\rho$ from $\omega_1, \dots , \omega_n$. Then $\mathcal I^{\omega, \rho}_{\III}$ has a unique minimal element and, therefore, it is an interval in $\mathcal C_n$.
\end{lm}

\begin{proof}
%Recall that $\rho$ is a pair of orders in $I$.
%Notice that from the factorisation $\omega = \omega_1 \oplus \dots \oplus \omega_n$
We give a concrete description of $\mathcal I^{\omega, \rho}_{\III}$.
Consider the following set:
$$J = \{i \in [n-1] | I_i \lneq_P I_{i+1} \text{ and } I_i \lneq_V I_{i+1} \} = \{i \in [n-1] | \, \III \text{ is $\mu_i$-stable } \} \, ,$$
where the second equality is due to \cref{lm:QSSpackedWords}.
Let $\beta = \mathbf{CS}^{-1}( J)$. 
We claim that $\beta \in \mathcal I^{\omega, \rho}_{\III}$ and that this is its smallest element.

\
\begin{itemize}
\item {\bf That $\beta$ is in $\mathcal I^{\omega, \rho}_{\III}$} we prove now.\\
Indeed, we just need to establish that for each $I^{\beta}_i$ we have that $\rho|_{I^{\beta}_i} = x^{\beta}_i$.
Because $I^{\beta}_i = I_{f_{i-1} + 1} \cup \dots \cup I_{f_i}$, and $I_{f_{i-1} + 1} \lneq_P \dots \lneq_P I_{f_i}$, $I_{f_{i-1} + 1} \lneq_V \dots \lneq_V I_{f_i}$, we get that 
$$\rho|_{I^{\beta}_i} = \rho|_{I_{f_{i-1} + 1}} \oplus \dots \oplus \rho|_{I_{f_i}} = x_{f_{i-1} + 1} \oplus \dots\oplus x_{f_i} = x^{\beta}_i\,  ,$$

\

    \item {\bf That $\beta$ is the smallest element in $\mathcal I^{\omega, \rho}_{\III}$} we prove now. \\
Let $\alpha $ be a composition of $n$ such that $\alpha \not \geq \beta$.
Assume that $\alpha \neq \beta$. 
Then $J^c \cap \mathbf{CS}(\alpha)\neq \emptyset $, so pick some $i \in  J^c \cap \mathbf{CS}(\alpha) $.
Because $i \not\in J$, is such that $ I_i \not\lneq_P I_{i+1} \text{ or } I_i \not\lneq_V I_{i+1} $.

\

If $\omega_i \not\sim \omega_{i+1}$, one can see that $\rho|_{I_i \cup I_{i+1}} \neq \omega_i \oplus \omega_{i+1}$.
If $\omega_i \sim \omega_{i+1}$, stability would require $I_i \lneq_P I_{i+1}$, so $I_i \not\lneq_V I_{i+1}$, and we conclude again that $\rho|_{I_i \cup I_{i+1}} \neq \omega_i \oplus \omega_{i+1}$.

\

So we can never have $\rho|_{I^{\alpha}_j} = \omega^{\alpha}_j$ for $j$ such that $\omega^{\alpha}_j$ includes both $\omega_i \oplus \omega_{i+1}$ in its factorisation.
If $\omega_i \sim \omega_{i+1}$, stability would require $I_i \lneq_P I_{i+1}$, so $I_i \not\lneq_V I_{i+1}$, and we conclude again that $\rho|_{I_i \cup I_{i+1}} \neq \omega_i \oplus \omega_{i+1}$.
This is the contradiction that we are aiming for.
\end{itemize}
With the construction of the minimal element, we have that $\mathcal I^{\omega, \rho}_{\III}$ is an interval in $\mathcal{C}_n$.
\end{proof}

\

\begin{thm}\label{thm:antipode_packed}
Let $\omega $ be a packed word, and $\omega = \omega_1 \oplus \dots \oplus \omega_n$ be its decomposition into $\oplus$-indecomposible packed words.
Then, on the pattern Hopf algebra of packed words, we have the following cancellation free and grouping free formula:
$$S(\pat_{\omega}) = (-1)^n  \sum_{\rho} \bigl[\!\begin{smallmatrix} \rho  \\ \omega_1, \dots, \omega_n \end{smallmatrix}\!\bigr] \pat_{\rho}  \, .$$
\end{thm}


\begin{proof}
From \cref{thm:general_antipode}, we only need to establish that

\begin{equation}\label{eq:packed_alternating_sum}
\sum_{\alpha\in \mathcal I^{\rho, \omega}_{\III}} (-1)^{\ell (\alpha)} = (-1)^n \mathbb{1}[\III \text{ is interlacing QSS of $\rho$ from } \omega_1, \dots, \omega_n]\, .      
\end{equation}

\

Further, from \cref{prop:filter_structure_I} we know that the sum $\sum_{\alpha\in \mathcal I^{\rho, \omega}_{\III}}  (-1)^{\ell (\alpha)} $ vanishes whenever $\mathcal I^{\rho, \omega}_{\III}$ is an interval with more than one element.
From \cref{lm:minpacked}, we know that $\mathcal I^{\rho, \omega}_{\III}$ is indeed an interval.
The minimal interval is $\mathbb{1}$ if and only if $\III$ is an interlacing QSS from \cref{lm:minpacked}.
This concludes the proof.
\end{proof}

\

%From \cref{lm:minpacked}, we know that $\mathcal I^{\rho, \omega}_{\III}$ is indeed an interval, so the sum on the LHS of \eqref{eq:packed_alternating_sum} is zero except when $\mathbb{1} = \mathbf{CS}^{-1}([n-1]\setminus J)$, that is, when 
%$$\{i \in [n-1] | I_i <_P I_{i+1} \text{ and } I_i <_V I_{i+1} \} = \emptyset \, .$$

%This is precisely when $\III$ is an interlacing QSS.

\

Notice that this proof hides a sign-reversing involution in it.
Specifically, it was used in establishing \cref{prop:filter_structure_I}.

\

\subsection{The antipode formula for the pattern algebra on permutations\label{sec:formula_permutation}}
We start by recalling the definition of interlaced QSS on permutations.

\begin{defin}[Interlacing QSS on permutations]
Let $\sigma, \pi_1, \dots, \pi_n$ be permutations, where $\sigma = (\leq_P, \leq_V)$ is a permutation on $I$.
Let $\III = (I_1, \dots, I_n)$ be a QSS of $\sigma$ from $\pi_1, \dots, \pi_n$.
We say that $\III$ is \textbf{non-interlacing} if there exists $i = 1, \dots, n-1$ such that $I_i \lneq_P I_{i+1}$ and $I_i \lneq_V I_{i+1}$.
If no such $i$ exists, we say that the QSS is \textbf{interlacing}.

\

Additionally, let 
$ \bigl[\!\begin{smallmatrix} \sigma \\ \pi_1, \dots, \pi_n \end{smallmatrix}\!\bigr]$ be the number of interlacing QSS of $\sigma$ from $\pi_1, \dots, \pi_n$
\end{defin}

\

\begin{thm}\label{thm:antipode_perms}
Let $\pi $ be a permutation, and $\pi = \pi_1 \oplus \dots \oplus \pi_n$ be its decomposition into irreducible permutations.
Then, on the pattern Hopf algebra of permutations, we have the following cancellation free and grouping free formula:
$$S(\pat_{\pi}) = (-1)^n \sum_{\sigma} \bigl[\!\begin{smallmatrix} \sigma \\ \pi_1, \dots, \pi_n \end{smallmatrix}\!\bigr] \pat_{\sigma} \, .$$
\end{thm}

\

Although we can obtain the antipode formula by showing that a relevant poset of compositions is an interval (see \cref{lm:minpacked}), we present here a different proof.
Specifically, we will be leveraging a map $\mathcal A[\mathrm{inc}]: \mathcal A(\mathtt{PW}) \to \mathcal A(\mathtt{Per})$ and the previous result on packed words.

\

First, observe that any permutation is a packed word, because any pair of total orders $(\leq_P, \leq_V)$ is a packed word, that is, a pair of partial orders such that $\leq_P$ is a total order and $\leq_V$ is partial linear order.
This gives us an inclusion map $\mathrm{inc} : \mathtt{Per} \to \mathtt{PW}$ that preserves restrictions and the monoidal structure.
Therefore, this gives us a surjective Hopf algebra morphism $\mathcal A[\mathrm{inc}]: \mathcal A(\mathtt{PW}) \to \mathcal A(\mathtt{Per})$.


\begin{proof}
From \cref{thm:antipode_packed}, we know that for any permutation $\pi$ seen as a packed word we have that
$$S(\pat_{\pi}) = (-1)^n \sum_{\omega} \bigl[\!\begin{smallmatrix} \omega \\ \pi_1, \dots, \pi_n \end{smallmatrix}\!\bigr] \pat_{\omega}\, . $$

If $\omega$ is a packed word, we compute $\mathcal A[\,\mathrm{inc}] (\pat_{\omega}) = \pat_{\omega}\mathbb{1}[\omega \text{ is a permutation }]$.
Thus, applying $\mathcal A[\mathrm{inc}]$ to both sides of the equation above, we get the desired result.
Note that $\mathcal A[\mathrm{inc}]$ is a Hopf algebra morphism from \cref{thm:functoriality}, so it commutes with the antipode.
\end{proof}
%
%Observe that we also have a map $\mathrm{st}: \mathtt{PW} \to \mathtt{Per}$.
%This gives us an injective map $\mathcal A[\mathrm{st}]: \mathcal A[\mathtt{Per}] \to  \mathcal A[\mathtt{PW}] $ defined as follows
%$$\mathcal A[\mathrm{st}] (\pat_{\pi} ) = \sum_{\mathrm{st}(\omega ) = \pi} \pat_{\omega}.$$
%We can also use this map to find an antipode formula for the permutation pattern Hopf algebra.
%\todo[inline]{Extend this remark.}
%
%\todo[inline]{what follows is the old proof, kept here for legacy purposes, and because some definitions needed above are still buried in this section, a keen eye will need to be used to parse them}
%
%
%Fix a permutation $\pi $, such that $\pi = \pi_1 \oplus \dots \oplus \pi_k $ is its $\oplus$-factorisation into $\oplus$-indecomposable permutations.
%We will assume this notation for the remaining of this section.
%
%The intermediate step to conclude \cref{thm:cancellationformAPer} is \cref{lm:interlacingcefsformula}, which is a description of the interlacing coefficients via an inclusion-exclusion sum.
%In order to establish this, we define a classical poset structure in compositions, which we will see is isomorphic to a Boolean poset.
%This allows us to simplify sums quite drastically, which leads us to the desired formula.
%
%\begin{defin}[Order on compositions]\label{defin:ordercomps}
%We can endow the set of compositions of $n$ with an order $\leq $ as follows:
%We say that $\alpha \leq \beta $ if $\beta $ results from $\alpha $ by merging two consecutive entries.
%By taking the transitive closure, we have a poset in $\mathcal C_n$.
%Note that $(n)$ is the unique minimal element, whereas $(1, \dots , 1)$ is the unique maximal element.
%\end{defin}
%
%%\begin{rem}
%%This results in the \textbf{opposite} order of the coarsening order, defined in \cref{defin:coarseningorder} for set partitions and set compositions, and extended in \cref{lm:ordercomps} to an order $\leq'$ in compositions.
%%\end{rem}
%
%We now describe the bijection between $\mathcal C_n$, the set of compositions of $n$, and the Boolean poset $[k-1]$.
%
%\begin{defin}[Cumulative sum map]
%Assume $n \geq 1$.
%Then we define an explicit bijection between subsets of $[n-1]$ and the set $\mathcal C_n$: if $\alpha=(\alpha_1, \dots ,\alpha _j)$ is a set composition, then we define
%$$ \mathbf{CS}(\alpha ) = \{\alpha_1, \alpha_1 + \alpha_2, \dots , \alpha_1 +\dots + \alpha_{j-1}\}\, . $$
%This is a subset of $[n-1]$.
%That this is a bijection is immediate, as an inverse can be readily constructed.
%
%We further define the integers $ \mathbf{CS}(\alpha )_1 < \mathbf{CS}(\alpha )_2 < \dots < \mathbf{CS}(\alpha )_{j-1}$ such that 
%$$ \mathbf{CS}(\alpha ) =\{ \mathbf{CS}(\alpha )_1 ,  \mathbf{CS}(\alpha )_2 ,  \dots ,  \mathbf{CS}(\alpha )_{j-1} \}\, . $$
%and set $\mathbf{CS}(\alpha )_0 = 0 $ and $ \mathbf{CS}(\alpha )_j = n $.
%
%Furthermore, this map is a poset isomorphism, where it maps $\leq $, the partial order in $\mathcal C_n$ introduced in \cref{defin:ordercomps}, to the inclusion of sets.
%Details on this map are given in \cite{stanley00}.
%
%Recall that we fix a permutation $\pi $ with $\pi = \pi_1 \oplus \dots \oplus \pi_k $.
%Given a composition $\alpha \models n$, and an integer $i \in \{ 1 , \dots , l(\alpha )\}$, we write 
%$$\pi^{(i)}_{\alpha } = \pi_{\mathbf{CS}(\alpha )_{i-1}+1} \oplus \dots \oplus \pi_{\mathbf{CS}(\alpha )_{i}}\, . $$
%This notation can be further extended to weak compositions $\alpha $ by setting $\pi^{(i)}_{\alpha } = \pi^{(i)}_{\beta }$, where $\beta$ is the composition resulting from erasing the zeros from $\alpha $.
%\end{defin}
%
%
%
%
%\begin{lm}\label{lm:interlacingcefsformula}
%$$\bigl[\!\begin{smallmatrix} \sigma  \\ \pi_1, \dots ,\pi_k \end{smallmatrix}\!\bigr] =  \sum_{\alpha\models n} (-1)^{l(\alpha )} \binom{\sigma }{\pi^{(1)}_{\alpha }, \dots ,\pi^{(l(\alpha ))}_{\alpha }}\, . $$
%\end{lm}
%
%We now see that this result suffices to establish the main result of this section.
%
%\begin{proof}[Proof of \cref{thm:cancellationformAPer}]
%We simply apply Takeuchi's formula and use \cref{lm:interlacingcefsformula} when needed:
%\begin{equation}
%\begin{split}
%S(\pat_{\pi}) &= \sum_{j=0}^k (-1)^j \mu^{ \circ j-1} \circ ( \id_{\mathcal A ( \mathtt{Per} )} - \iota \circ \varepsilon)^{ \otimes j}\circ \Delta^{\circ j-1}( \pat_{\pi}) \\
%&= \sum_{\alpha\models^0 k} (-1)^{l(\alpha )} \mu^{\circ l(\alpha )-1 } \circ ( \id_{\mathcal A ( \mathtt{Per} )} - \iota \circ \varepsilon)^{ \otimes l(\alpha )}  ( \pat_{\pi^{(1)}_{\alpha }} \otimes \dots \otimes \pat_{\pi^{(l(\alpha ))}_{\alpha } }) \\
%&= \sum_{\alpha\models k} (-1)^{l(\alpha )} \mu^{\circ l(\alpha )-1 }  ( \pat_{\pi^{(1)}_{\alpha }} \otimes \dots \otimes \pat_{\pi^{(l(\alpha ))}_{\alpha } })= \sum_{\alpha\models k} (-1)^{l(\alpha )}  \pat_{\pi^{(1)}_{\alpha }}  \dots  \pat_{\pi^{(l(\alpha ))}_{\alpha } } \\
%&= \sum_{\alpha\models k} (-1)^{l(\alpha )} \sum_{\sigma } \pat_{\sigma} \binom{\sigma}{ \pat_{\pi^{(1)}_{\alpha }} ,  \dots  , \pat_{\pi^{(l(\alpha ))}_{\alpha } } } \\
%&=  \sum_{\sigma } \pat_{\sigma} \sum_{\alpha\models k} (-1)^{l(\alpha )} \binom{\sigma}{ \pat_{\pi^{(1)}_{\alpha }} ,  \dots  , \pat_{\pi^{(l(\alpha ))}_{\alpha } } } =  \sum_{\sigma } \pat_{\sigma} \bigl[\!\begin{smallmatrix} \sigma  \\ \pi_1, \dots ,\pi_k \end{smallmatrix}\!\bigr] \, ,
%\end{split}
%\end{equation}
%where in the last equality we used \cref{lm:interlacingcefsformula}.
%\end{proof}
%
%In order to establish \cref{lm:interlacingcefsformula}, we proceed as follows:
%For a composition $\alpha$, we introduce a notion of $\alpha$-QSS of a permutation $\sigma$.
%We see that any interlacing QSS of $\sigma$ cannot be \say{extended} to a non-trivial $\alpha$-QSS of a permutation $\sigma$.
%Finally, we see that in \cref{lm:posetsumcancelation}, all the QSS of $\sigma$ that can be extended will cancel its contribution to the interlacing quasi-shuffle coefficient.
%
%\begin{defin}
%Fix a permutation $\sigma $ and a composition $\alpha $.
%We say that $\III$ is an $\alpha$-QSS of $\sigma$ from $\pi_1, \dots ,\pi_k$ if $\III $ is a QSS of $\sigma $ from $\pi^{(1)}_{\alpha }, \dots ,\pi^{(l(\alpha ))}_{\alpha } $.
%When the permutations $\pi_1, \dots ,\pi_k$ are clear from context, we simply write that $\III $ is an $\alpha$-QSS of $\sigma$.
%
%Assume that $\III$ is an $\alpha$-QSS of $\sigma$, then we can construct a canonical QSS of $\sigma $, which we write  $\JJJ = \sigma(\III)$, as follows:
%For each $i= 1 , \dots , l(\alpha )$, let $J^{(i)}_1, \dots , J^{(i)}_{\alpha_i} $ be the unique choice of sets such that 
%\begin{itemize}
%\item $\biguplus_{j=1}^{\alpha_i} J_j^{(i)} = I_i$;
%
%\item $J_j^{(i)}< J_{j+1}^{(i)} $ for any $j=1 , \dots , \alpha_i-1 $;
%
%\item $\sigma|_{J_j^{(i)}} = \pi_{\mathbf{CS}(\alpha)_{i-1} + j}$ for any $j=1 , \dots , \alpha_i $.
%\end{itemize}
%
%This is possible, because $\sigma|_{I_i} = \pi_{\alpha }^{(i)} = \pi_{\mathbf{CS}(\alpha )_{i-1}+1} \oplus \dots \oplus \pi_{\mathbf{CS}(\alpha )_{i}} $.
%Then, by letting $\JJJ = (J^{(1)}_1, \dots , J^{(1)}_{\alpha_1},  J^{(2)}_1, \dots ) $ we obtain a QSS of $\sigma $.
%
%
%
%Finally, fix a permutation $\sigma $, and let $\JJJ $ be a QSS of $\sigma $.
%Then, define 
%$$\mathcal I^{\sigma, \pi }_{\JJJ} = \{\alpha \models k \, \,  | \, \,  \exists \, \III \text{ QSS of } \sigma \text{ s.t. } \sigma(\III ) = \JJJ  \}\, . $$
%\end{defin}
%
%
%
%\begin{lm}
%Fix a permutation $\pi$ and $\sigma $, and let $\pi_1, \dots ,\pi_k$ be as above.
%Given $\JJJ$ a QSS of $\sigma$, and $\alpha$ a composition of $k$, then there is at most one $\III$ that is an $\alpha$-QSS of $\sigma$ such that $\sigma(\III ) = \JJJ$.
%
%Furthermore, $\JJJ $ is a $(1, \dots , 1)$-QSS of $\sigma$ and $\sigma(\JJJ ) = \JJJ$.
%\end{lm}
%
%\begin{proof}
%That $\JJJ $ is a $(1, \dots , 1)$-QSS of $\sigma$ and $\sigma(\JJJ ) = \JJJ$ is immediate by definition.
%The fact that there is at most one such $\III $, follows from the fact that the procedure $\sigma ( \III ) = \JJJ $ is invertible if we know $\alpha $: if $\sigma ( \III ) = \JJJ $, then $\III $ results from $\JJJ =(J_1, \dots , J_k)$ by taking the union of consecutive sets $J_i$ according to $\alpha $.
%Thus it follows that such QSS $\III $ is unique.
%\end{proof}
%
%\begin{rem}
%Having established the uniqueness, one can wonder if we can also guarantee the existence.
%That is in fact not the case, because the procedure of \say{taking the union of consecutive sets $J_i$ according to $\alpha $} does not guarantee that this union will satisy the properties of QSS, namely that $\sigma|_{I_i}= \pi^{(i)}_{\alpha }$.
%
%In fact, we will see now that if $\JJJ $ is an interlacing QSS of $\sigma $, then there is no such $\alpha $ and such $\III$, except the trivial choice $\III= \JJJ$ and $\alpha = (1, \dots , 1)$.
%\end{rem}
%
%The following lemma is crucial in establishing \cref{lm:interlacingcefsformula}.
%
%\begin{lm}\label{lm:noninterlacingcrit}
%Fix a permutation $\pi$ and $\sigma $, and let $\pi_1, \dots ,\pi_k$ be as above.
%Fix $\JJJ$ a QSS of $\sigma $.
%Then $|\mathcal I^{\sigma, \pi }_{\JJJ} | = 1 $ if and only if $\JJJ $ is interlacing QSS of $\sigma$.
%\end{lm}
%
%\begin{proof}
%Assume that $\JJJ = (J_1, \dots , J_k ) $ is non-interlacing QSS of $\sigma$.
%Then, there is some $i= 1, \dots , k-1 $ such that $J_i< J_{i+1}$ and $\sigma(J_i) < \sigma (J_{i+1})$.
%We claim that $\alpha = (\underbrace{1, \dots ,1 }_{i-1 \text{ times}}, 2, \underbrace{1, \dots ,1 }_{k-i-1 \text{ times}}) \in \mathcal I^{\sigma, \pi }_{\JJJ}$.
%
%
%In fact, we have that $\III = (J_1, \dots J_{i-1}, J_i\cup J_{i+1}, J_{i+2}, \dots , J_k ) $ is precisely the desired $\alpha$-QSS, since we have by construction that $\sigma (\III ) = \JJJ$.
%Further, the non-interlacing condition gives us that $\sigma|_{ J_i\cup J_{i+1}} = \pi_i\oplus\pi_{i+1}$.
%
%Thus $|\mathcal I^{\sigma, \pi }_{\JJJ} | \neq 1 $, as desired.
%
%\todo[inline]{from this part until the YVY I still don't like how it looks}
%
%On the other hand, if $\JJJ $ is interlacing, suppose that there is some non-trivial $\alpha \in \mathcal I^{\sigma, \pi }_{\JJJ}$, and consider $\III $ its corresponding $\alpha$-QSS of $\sigma$.
%Then $\JJJ = \sigma (\III )$, and this contradicts the assumption that $\JJJ $ is interlacing by construction of $\sigma (\III)$.
%\end{proof}
%
%\begin{lm}\label{lm:principalideal}
%The set $\mathcal I^{\sigma, \pi }_{\JJJ} $ is an ideal with a unique minimum.
%\end{lm}
%
%Because we know that $(1, \cdots , 1) \in \mathcal I^{\sigma, \pi }_{\JJJ} $, this means that $\mathcal I^{\sigma, \pi }_{\JJJ} $  is in fact an interval.
%
%\begin{proof}
%Let $P=\{i \in [k-1]| J_i < J_{i+1} \text{ and } \sigma(J_i) < \sigma( J_{i+1}\}\}$.
%We claim that $\mathcal I^{\sigma, \pi }_{\JJJ} = \{ \alpha \models [k-1] | \mathbf{CS}(\alpha ) \cup P = [k-1] \}$.
%That is, $\mathbf{CS}^{-1}([k-1]\setminus P ) $ is the unique minimal element in $\mathcal I^{\sigma, \pi }_{\JJJ} $
%
%First, let $\alpha \geq \mathbf{CS}^{-1}([k-1]\setminus P ) $.
%We observe that it is straightforward to construct an $\alpha$-QSS of $\sigma$, say $\III $, such that $\sigma(\III) = \JJJ$.
%The definition of $P$ gives us that we indeed obtain an $\alpha$-QSS of $\sigma$.
%
%On the other hand, suppose that $\JJJ = \sigma (\III )$, where $\III$ is an $\alpha$-QSS of $\sigma$ such that $\mathbf{CS}(\alpha ) \cup P \neq [k-1] $.
%Say $j\in [k-1] $ is such that $j \not \in\mathbf{CS}(\alpha ) \cup P$.
%Then either $J_j \not< J_{j+1}$ or $\sigma(J_j) \not< \sigma (J_{j+1})$.
%
%However, from $j\not\in \mathbf{CS}(\alpha )$, we know that there is some $i\in \{1, \dots , l(\alpha ) \}$ such that $J_j, J_{j+1}\subseteq I_i$.
%From $\JJJ = \sigma (\III )$ we know that $J_j, J_{j+1}$ are obtained from the $\oplus$-decomposition of $\sigma|_{I_i}$.
%This contradicts the assumption that either $J_j \not< J_{j+1}$ or $\sigma(J_j) \not< \sigma (J_{j+1})$.
%\end{proof}
%
%\begin{lm}\label{lm:posetsumcancelation}
%Fix a permutation $\pi$ and $\sigma $, and let $\pi_1, \dots ,\pi_k$ be as above.
%Fix $\JJJ$ a QSS of $\sigma $.
%Then
%$$\sum_{\alpha \in \mathcal I^{\sigma, \pi }_{\JJJ}}(-1)^{ l(\alpha )} = \begin{cases}
%(-1)^k, \text{ if } |\mathcal I^{\sigma, \pi }_{\JJJ} | = 1, \\
%0, \text{ otherwise }
%\end{cases} \, . $$
%\end{lm}
%
%\begin{proof}
%The case where $|\mathcal I^{\sigma, \pi }_{\JJJ} | = 1$ is trivial, because in this case we have $\mathcal I^{\sigma, \pi }_{\JJJ} = \{\underbrace{(1, \dots ,1 )}_{k \text{ times}}\}$.
%Otherwise, from \cref{lm:principalideal} we have that $\mathcal I^{\sigma, \pi }_{\JJJ} $ is an ideal with a unique minimum that contains $\{\underbrace{(1, \dots ,1 )}_{k \text{ times}}\}$.
%Thus, $\mathbf{CS} ( \mathcal I^{\sigma, \pi }_{\JJJ}  ) $ is an ideal with a unique minimum, so there is a set $M \subsetneq [k-1] $ such that 
%$$\mathbf{CS} ( \mathcal I^{\sigma, \pi }_{\JJJ}  ) = \{K \subseteq [k-1] |  M \subseteq K \} \, . $$
%
%Thus, we have that
%\begin{equation}
%\begin{split}
%\sum_{\alpha \in \mathcal I^{\sigma, \pi }_{\JJJ}}(-1)^{ l(\alpha )} =& \sum_{\substack{M\subseteq \mathbf{CS} (\alpha )\\ \mathbf{CS} (\alpha ) \subseteq [k-1]}} (-1)^{l(\alpha )}\\
%=& \sum_{\substack{M\subseteq K\\K \subseteq [k-1]}} (-1)^{|K |+1} = 0\, , 
%\end{split}
%\end{equation}
%where the last equality is a simple binomial identity.
%\end{proof}
%
%With this we can finally prove \cref{lm:interlacingcefsformula}, which concludes the proof of the cancellation-free formula of the antipode of $\mathcal A (\mathtt{Per})$.
%
%\begin{proof}[Proof of \cref{lm:interlacingcefsformula}]
%
%We use \eqref{eq:qscoefasQSS} to develop the sum at hand, yielding:
%\begin{equation}
%\begin{split}
%\sum_{\alpha\models n} (-1)^{l(\alpha )} &\binom{\sigma }{\pi^{(1)}_{\alpha }, \dots ,\pi^{(l(\alpha ))}_{\alpha }} =  \sum_{\alpha\models n}  (-1)^{l(\alpha )} \sum_{\III \text{ is } \alpha-\text{QSS of $\sigma$}} 1  \\
% =&  \sum_{\alpha\models n} (-1)^{l(\alpha )} \sum_{\JJJ \text{ is QSS of $\sigma$}} \mathbb{1}[\exists \, \, \III \, \alpha-\text{QSS s.t. } \sigma(\III) = \JJJ]   \\
% =& \sum_{\JJJ \text{ is QSS of $\sigma$}} \sum_{\substack{\alpha\models n\\ \alpha \in \mathcal I^{\sigma, \pi}_{\JJJ}}}  (-1)^{l(\alpha )}  \\
%  =& \sum_{\substack{\JJJ \text{ is QSS of $\sigma$}\\ |\mathcal I^{\sigma, \pi}_{\JJJ}| = 1}} (-1)^k\, , \\
%\end{split}
%\end{equation}
%where in the last sum we used \cref{lm:posetsumcancelation}.
%
%Thus, we are to compute
%$$|\{ \JJJ \text{ is QSS of $\sigma$ s.t.} |\mathcal I^{\sigma, \pi}_{\JJJ}| = 1 \} |\, , $$
%which is precisely $ \bigl[\!\begin{smallmatrix} \sigma  \\ \pi_1, \dots ,\pi_k \end{smallmatrix}\!\bigr]$, according to \cref{lm:noninterlacingcrit}.
%This concludes the proof.
%\end{proof}
%


\


\section{Reciprocity results\label{sec:reciprocity}}

In this section, let $\mathtt{R}$ be a connected associative species with restrictions.
On this setting, we define a new polynomial invariant associated to any object in $\mathtt{R}$, the \textbf{multiple occurrences polynomial}, or MOP for short.
The MOP construction in this section is rather general and corresponds to an infinite family of polynomial maps $\mathcal{A}(\mathtt{R}) \to \mathbb{K}[x]$, one for each choice of object $\mathrm{z} \in \mathtt{R}$.

After the definition of the MOP, we present a motivation for this construction.
Our main theorem is that this construction is in fact a Hopf algebra morphism, and we leave the proof of this result to the end of the section.
We also display an application of the cancellation free formula above, \cref{thm:antipode_perms} to produce a reciprocity result, following the steps in \cite{humpert2012incidence}, on the permutation pattern Hopf algebra.


\begin{defin}[Multiple occurrence polynomial on a species with restrictions $\mathtt{R}$]\label{MOPDef}
Fix an object $\mathrm{z}$ of the connected associative species with restrictions $\mathtt{R}$.
We construct a map $\chi^{\mathrm{Z}}: \mathcal{A}(\mathtt{R}) \to \mathbb{K}[x]$.
For that, fix a positive integer $x$ and an object $\mathrm{y}$ in $\mathtt{R}$, $\chi^{\mathrm{Z}}$ has
$$\chi^{\mathrm{z}}(\pat_{\mathrm{y}})(x) \coloneqq \pat_{\mathrm{y}}(\mathrm{z}^{\star x})\, , $$
where $\mathrm{z}^{\star x}$ denotes the $\oplus$-product of $\mathrm{z}$ with itself $x$ times.
\end{defin}

That this indeed defines a polynomial, and that this is a Hopf algebra morphism, is the content of the main theorem below \cref{thm:polynomiality}.
Let us now see what values this invariant takes.
For notation simplicity we write $\chi^{\mathrm{x}}_{\mathtt{y}} \coloneqq \chi^{\mathrm{x}}(\pat_{\mathtt{y}})$.

\begin{smpl}
We compute this invariant on particular objects for $\mathtt{R} = \mathtt{Per}$, defined above in \cref{sec:speciespermutation}.
Take $\mathtt{z} = 1$, the unique permutation of size one.
It is easy to see that $\chi^{\mathrm{z}}_{\mathrm{y}}(x) = 0$ whenever $\mathrm{y}$ is not an increasing permutation, and $\chi^{\mathrm{z}}(\pat_{\mathrm{z}^{\star k}})(x) = \binom{x}{k}$, a polynomial of degree $k$.
In fact, we can observe something much more general: if $\pi, \tau$ have disjoint sets of $\oplus$-indecomposable factors, then $\chi^{\tau}_{\pi} = 0$

Let's now take $\mathtt{z} = 21$, the unique indecomposable permutation of size two.
In this case, $\ker \chi^{\mathrm{z}} = \spn \{ \pat_{\pi} | \text{$\oplus$-factorization of $\pi$ contains a permutation $\tau$ with $|\tau| > 2$} \}$ from the observation above.
If $\pi $ decomposes into $k_1$ copies of $\tau_1 = 1$ and $k_{21}$ copies of $\tau_2 = 21$, then $\chi^{\mathrm{z}}_{\pi}(x) = 2^{k_1} \binom{x}{k_1 + k_{21}}$.
\end{smpl}

The MOP encapsulates the complexity of the species with restrictions quite well.
One can easily observe that, for the species with restrictions $\mathtt{R} = \mathtt{Gr}$ on graphs, a general formula for $\chi^{\mathrm{z}}_{\mathrm{y}}(x)$ can be found for indecomposable $\mathrm{y}$, depending only on $\pat_{\mathrm{y}}(\mathrm{z})$.
This is not so simple for the permutation case.
On $\mathtt{R} = \mathtt{MPer}$, the species with restrictions of marked permutations, the story is even more complicated as one does not have a unique factorization into indecomposables, making the description of all possible factorizations more complex.
In this way, we ask the following question.

\begin{conj}
    For any marked permutations $\mathtt{y}, \mathtt{z}$, the polynomial $\chi^{\mathrm{z}}_{\mathrm{y}}$ is a binomial multiplied by some constant, that is it maps $x \mapsto a \binom{x}{b}$ for $a, b$ independent of $x$.
\end{conj}

\begin{obs}
    Fixing $x = 1$ we get a multiplicative map $\mathcal{A}(\mathtt{R}) \to \mathbb{K}$.
    This map is in fact a \textbf{character}, that is, a unital algebra homomorphism to the field $\mathbb{K}$.
    
    In \cite{aguiar2006combinatorial}, a Hopf algebra morphism was constructed from any connected graded Hopf algebra to $QSym$ using only a character.
    The astute reader may note that there is a classical \textit{specialisation} map $QSym \to \mathbb{K}[x]$, so one should wonder if this $\chi^{\mathrm{z}}$ arises from lifting a character $\mathcal{A}(\mathtt{R}) \to \mathbb{K}$ to a map $QSym \to \mathbb{K}[x]$, followed by composing with said specialisation.
    
    Indeed, the evaluation map $\mathcal{A}(\mathtt{R}) \to \mathbb{K}$ defined by $\pat_{\mathrm{y}} \mapsto \pat_{\mathrm{y}}(\mathrm{z})$ is this character.
    We leave the proof of this fact to the interested reader, because it is mostly tedious unwravelling of definitions that closely relates to the proof of \cref{lm:polynomial} below.
\end{obs}

We now apply \cref{thm:antipode_perms} to extract a reciprocity result for permutations.

\begin{smpl}
    Let us get back to the case $\mathrm{z} = 21$.
    Because the antipode on polynomials simply flips the sign of the variable, we have
$$\chi^{21}_{\pi}(-x) = S(\chi^{21}(\pat_{\pi})(x)) =\chi^{21}(S(\pat_{\pi}))(x) $$

Setting $x = 1$ and using \cref{thm:antipode_perms}, we get
$$\chi^{21}_{\pi}(-1) = S(\pat_{\pi})(21) = (-1)^n \sum_{\sigma} 
    \begin{bmatrix}
    \sigma \\ \pi_1, \dots, \pi_n
    \end{bmatrix}\pat_{\sigma}(21) $$

    This allows us to quickly compute simple cases for $\pi = 12$, for instance $\chi^{21}_{\pi}(-1) = 1 \times \pat_{21}(21) = 1$.
\end{smpl}



\subsection{Proof of polynomiality and Hopf morphism}

We present and prove the main theorem of this section.

\begin{thm}[MOPs in pattern algebras]\label{thm:polynomiality}
Let $\mathtt{R}$ be a connected associative species with restrictions and $\mathrm{x}$ an object.
The map $\chi^{\mathrm{x}}$ takes patterns to polynomials and is a Hopf algebra morphism $\mathcal A(\mathtt{R}) \to \mathbb{K}[x]$.
\end{thm}

We recall that for each object $\yy$ in a connected associative species with restrictions, any factorization into $\ast$-indecomposable elements has the same multiset of factors and, therefore, the same size, which we call $\ell(\yy)$.

\begin{thm}\label{lm:polynomial}
The invariant $\chi^{\xx}_{\yy}(n)$ is a polynomial in $n$.
The degree of this polynomial is at most $\ell(\yy)$, the number of $\ast$-indecomposable factors of $\yy$ (with repetition).
\end{thm}


\begin{proof}
We act by induction on $|\yy|$.
If $|\yy| = 0$, recall that $\mathtt{R}$ is connected, so $\chi^{\xx}_{\yy}(n) = 1$ for all $n$, which is a polynomial of degree zero.

For the induction hypothesis, observe that
$$\Delta \pat_{\yy} = 1\otimes \pat_{\yy} + \pat_{\yy} \otimes 1 + \sum_{\substack{a \ast b = y\\\ell(a), \ell(b) < \ell(\yy)}} \pat_a \otimes \pat_b \, ,$$
thus we have for $n > 1$. The key insight is that differences of consecutive values are polynomials of lower degree:

\begin{equation*}
    \begin{split}
        \pat_{\yy}(\xx^{\ast n}) &= \pat_{\yy}(\xx^{\ast n-1}) + \pat_{\yy}(\xx) + \sum_{\substack{a \ast b = y\\|a|, |b| < |\yy|}} \pat_a (\xx) \otimes \pat_b (\xx^{\ast n-1}) \\
         \pat_{\yy}(\xx^{\ast n}) &- \pat_{\yy}(\xx^{\ast n-1}) = \pat_{\yy}(\xx) + \sum_{\substack{a \ast b = y\\|a|, |b| < |\yy|}} \pat_a (\xx) \chi^{\xx}_b ( n-1)
    \end{split}
\end{equation*}

The right hand side is, by induction hypothesis, a polynomial of degree at most $\ell(\yy) -1$.
The left hand side is $\chi^{\xx}_{\yy}(n) - \chi^{\xx}_{\yy}(n-1)$, which shows that $\chi^{\xx}_{\yy}(n)$ has degree at most $\ell(\yy)$.
\end{proof}


\begin{lm}\label{lm:HA_morphism}
The map $\pat_{\yy} \mapsto \chi^{\xx}_{\yy}$ is a Hopf algebra morphism.
\end{lm}

\begin{proof}
   Observe the product is preserved precisely because the product on $\mathcal{A}(\mathtt{R})$ is the pointwise product: that is, if $\pat_{\yy_1} \pat_{\yy_2} = \sum_{\zz} \binom{\zz}{\yy_1, \yy_2}_{\mathtt{R}} \pat_{\zz}$, then passing in the argument $x^{\ast n}$ of both sides yields a true equality $\chi^{\xx}_{\yy_1}(n) \chi^{\xx}_{\yy_2}(n) = \sum_{\zz} \binom{\zz}{\yy_1, \yy_2}_{\mathtt{R}} \chi^{\xx}_{\zz}(n)  $.

    For the coproduct, we want to show that $\Delta \chi^{\xx}_{\yy} = \chi^{\xx} \otimes \chi^{\xx}(\Delta \pat_{\yy})$.
    It is enough to show that $\Delta \chi^{\xx}_{\yy} (m \otimes n) = \chi^{\xx} \otimes \chi^{\xx}(\Delta \pat_{\yy})(m\otimes n) $ for any non-negative integers $m, n$.

    Write $\chi^{\xx}_{\yy} = \sum_{k\geq 0 } a_k x^k$.
    On the left hand side, we get $\Delta \chi^{\xx}_{\yy} (m \otimes n) = \sum_{k \geq 0} a_k \sum_{j=0}^k \binom{k}{j} x^j\otimes x^{k-j} (m \otimes n) = \sum_{k \geq 0} a_k (m+n)^k = \chi^{\xx}_{\yy}(m+n) $.

    On the right hand side we have 
    
    
    \begin{equation*}
        \begin{split}
        (\chi^{\xx} \otimes \chi^{\xx})\Delta \pat_{\yy}(m\otimes n) =& \sum_{a \ast b = \yy} \chi^{\xx}_a(m) \otimes \chi^{\xx}_a(n)\\
        =& \sum_{a \ast b = \yy} \pat_a(x^{\ast m})\pat_b(x^{\ast n})\\
        =& \sum_{a \ast b = \yy} \pat_a\otimes \pat_b ( x^{\ast m} \otimes x^{\ast n})\\
        =& \Delta \pat_{\yy}(x^{\ast n} \ast x^{\ast b}) = \Delta \pat_{\yy}(x^{\ast m+n}) = \chi^{\xx}_{\yy}(n+m)\, .
        \end{split}
    \end{equation*}
    This is exactly what we obtained earlier, so this map preserves coproducts.
\end{proof}


\begin{proof}[Proof of \cref{thm:polynomiality}]
This is a consequence of \cref{lm:polynomial} and \cref{lm:HA_morphism}.
\end{proof}




\section*{Glossary of Key Terms and Abbreviations}

This glossary provides brief definitions of frequently used terms and abbreviations throughout the paper.

\begin{itemize}

\item \textbf{QSS} (Quasi-Shuffle Signature): A tuple $\III = (I_1, \dots, I_n)$ of subsets whose union covers the ground set of a permutation (or more generally, an object in a species with restrictions), where each subset represents a pattern occurrence. Unlike standard shuffles, the sets may overlap. See \cref{sec:intro}.

\item \textbf{MOP} (Multiple Occurrences Polynomial): A polynomial invariant associated to objects in a species with restrictions, defined by evaluating pattern elements at repeated copies of a fixed object. This invariant is a Hopf algebra morphism. See \cref{MOPDef}.

\item \textbf{NCF} (Non-Commuting Factorization): A property of species with restrictions where every element has a unique factorization into indecomposable elements. This property is crucial for obtaining cancellation-free antipode formulas. See \cref{sec:formula_general}.

\item \textbf{Cancellation-free formula}: An antipode formula expressed as a sum where each term contributes its full value without intermediate cancellations. Formally, a formula where the combinatorial objects can be partitioned such that non-fixed-point pairs cancel completely. See introduction of this paper.

\item \textbf{Grouping-free formula}: An antipode formula where each combinatorial object contributes independently, without requiring aggregation or grouping of multiple objects for simplification. See introduction of this paper.

\item \textbf{Interlacing QSS}: A restricted subclass of quasi-shuffle signatures where the sets do not maintain a consistent ordering across both position and value orders. This constraint is essential for cancellation-free antipode formulas. See \cref{sec:intro}.

\item \textbf{Species with restrictions}: A contravariant functor from the category of finite sets with injections to the category of sets, equipped with restriction maps that allow systematic decomposition of objects. This structure unifies the construction of all pattern Hopf algebras discussed in this paper.

\item \textbf{Pattern Hopf algebra}: A Hopf algebra constructed from combinatorial objects via species with restrictions, where the product counts pattern occurrences (via quasi-shuffles) and the coproduct respects the decomposition of objects.

\item \textbf{Sign-reversing involution method}: A combinatorial technique for obtaining cancellation-free antipode formulas using involutions that reverse the signs of non-fixed-point elements in a sum, allowing terms to cancel completely. See \cref{sec:antipode_computing}.

\end{itemize}

\section*{Acknowledgments}

Both authors would like to thank Valentin F\'eray and Christophe Reutenauer, for helping this project to start.
Both authors are also grateful for the efforts from the referee assigned to this paper, as many mistakes and unclear statements were helpfully pointed out.
The first author is also grateful to the SNF grant P2ZHP2 191301 and the Max Planck Institute, while the second one is grateful to the Austrian Science Fund FWF, grant I 5788 PAGCAP.


%\
%
%\section{Remarks before submission}
%
%\begin{itemize}
%    \item The pattern Hopf algebras on permutations and packed words can be obtained by the universal constructions discussed in \cite{AM2010}.
%    \item Mention works of P. Tamaroff and J. Kock.
%    \item Discuss marked permutations.
%\end{itemize}


\bibliographystyle{alpha}
\bibliography{Bibliography}



\end{document}
